\section{Structure of generalised symmetric pairs}
    \subsection{Setup}
        Fix a root-decomposable Lie algebra:
            $$(\g, \h, \n^{\pm}, \omega)$$
        along with a weighted automorphism:
            $$\vartheta \in \Aut(\g, \h)$$
        As these data are now fixed, and for the sake of avoiding notational clutter, let us suppress the dependence of root systems etc. on the underlying weighted Lie algebra $(\g, \h)$.

        \begin{convention}[Generalised eigenspaces]
            If $V$ is a finite-dimensional vector space and:
                $$T \in \End(V)$$
            then let us write $\Spec T$ for the set of eigenvalues of $T$, and for each $\lambda \in \Spec T$, let us write:
                $$V_{T[\lambda]} \subset V$$
            for the generalised eigenspace associated to said eigenvalue $\lambda$.
        \end{convention}

    \subsection{Symmetric pairs}
        Recall that any involutive endomorphism $\vartheta \in \End(V)$ on a vector space $V$ has eigenvalues $\pm 1$ (and hence is automatically an automorphism too)\footnote{As long as the characteristic of the ground field is not $2$, but this is not a concern for us.}, and hence induces the following eigenspace decomposition:
            $$V = V_{\vartheta[1]} \oplus V_{\vartheta[-1]}$$
        Through this, we infer that $\a$ is a Lie algebra, $\vartheta \in \Inv_{\Lie\Alg}(\a)$ , and $V := \a$ is the adjoint $\a$-module, then we can decompose the Lie algebra $\a$ in the following manner:
            \begin{equation} \label{equation: symmetric_space_decomposition}
                \a = \a_{\vartheta[1]} \oplus \a_{\vartheta[-1]}
            \end{equation}
        This is commonly called the \say{Cartan decomposition} or the \say{symmetric pair decomposition}, and we note that:
            $$\a_{\vartheta[1]} = \a^{\vartheta}$$
        i.e. the eigenspace of weight $1$ coincides - as a Lie subalgebra of $\a$ - with the fixed-point Lie subalgebra $\a^{\vartheta} := \{x \in \a \mid \vartheta(x) = x\}$. Note also that:
            $$\a_{\vartheta[-1]} = \a^{-\vartheta}$$
        with $-\vartheta$ being regarded as an anti-automorphism of $\a$.

    \subsection{Quasi-symmetric pairs} \label{subsection: quasi_symmetric_pairs}
        Consider the case when the weighted automorphism $\vartheta \in \Aut(\g, \h)$ is of finite order.
    
        We begin with the following observation, which motivates the notion of \say{quasi-symmetric pairs}.
        \begin{lemma}[Root partitions by pseudo-involutions] \label{lemma: root_partitions_by_pseudo_involutions}
            Any pseudo-involution $\vartheta \in \PseudoInv(\g, \h)$ gives rise to the following partition of the root system $\rootsystem$:
                \begin{equation} \label{equation: root_partitions_by_pseudo_involutions}
                    \rootsystem = \rootsystem^{\vartheta^*} \sqcup \rootsystem^{-\vartheta^*}
                \end{equation}
            wherein we have written $\rootsystem^{\vartheta^*} := \rootsystem \cap \h^*_{\vartheta^*[\pm 1]}$ for the subsets consisting of roots that are eigenvectors of the eigenvalues $\pm 1$ of $\vartheta^*|_{\h^*}$, respectively.
        \end{lemma}
            \begin{proof}
                By definition \ref{def: weighted_automorphisms_of_weighted_lie_algebras}, $\vartheta|_{\h}$ is in particular a linear involution of $\h$, and hence $\vartheta^*|_{\h^*}$ is a linear involution of $\h^*$. This means that the only possible eigenvalues of $\vartheta^*$ are $\pm 1$. At the same time, lemma \ref{lemma: weighted_automorphisms_permute_weights} tells us that the set of $\h$-weights (i.e. $0$ and the roots of $\g$, so the set of weights in question is $\{0\} \cup \rootsystem$) are permuted by $\vartheta$, for this is first and foremost a weighted automorphism (cf. definition \ref{def: weighted_automorphisms_of_weighted_lie_algebras}); specifically, the automorphism on $\rootsystem$ is given by $\vartheta^*$. Since eigenspaces of different eigenvalues intersect at $\{0\}$, and as this is not considered a root by convention, we thus get the partition \eqref{equation: root_partitions_by_pseudo_involutions} simply by setting $\rootsystem^{\vartheta^*} := \rootsystem \cap \h^*_{\vartheta^*[\pm 1]}$.
            \end{proof}
        \begin{remark}
            We caution the reader, though, that a similar partition is not available for the subset of simple roots, due to the fact that simple roots are strictly positive by definition.
        \end{remark}
        In light of lemma \ref{lemma: root_partitions_by_pseudo_involutions}, we would argue that - in searching for an analogue of the notion of fixed-point subalgebra for $\vartheta$ within $\g$, it is more natural to consider the action of the weightd automorphism on \textit{all} root spaces, not just the simple ones. Namely, under the action of $\vartheta$, the root space decomposition $\g = \h \oplus \bigoplus_{\alpha \in \rootsystem} \g_{\alpha}$ becomes:
            $$
                \begin{aligned}
                    \vartheta(\g) & = \vartheta(\h) \oplus \vartheta\left( \bigoplus_{\alpha \in \rootsystem} \g_{\alpha} \right)
                    \\
                    & = (\h^{\vartheta} \oplus \h^{-\vartheta}) \oplus \left( \bigoplus_{\alpha \in \rootsystem^{\vartheta^*}} \g_{\alpha} \right) \oplus \left( \bigoplus_{\alpha \in \rootsystem^{-\vartheta^*}} \vartheta(\g_{\alpha}) \right)
                    \\
                    & = (\h^{\vartheta} \oplus \h^{-\vartheta}) \oplus \left( \bigoplus_{\alpha \in \rootsystem^{\vartheta^*}} \g_{\alpha} \right) \oplus \left( \bigoplus_{\alpha \in \rootsystem^{-\vartheta^*}} \g_{\vartheta^*(\alpha)} \right)
                    \\
                    & = (\h^{\vartheta} \oplus \h^{-\vartheta}) \oplus \left( \bigoplus_{\alpha \in \rootsystem^{\vartheta^*}} \g_{\alpha} \right) \oplus \left( \bigoplus_{\alpha \in \rootsystem^{-\vartheta^*}} \g_{-\alpha} \right)
                \end{aligned}
            $$

    \subsection{Pseudo-symmetric pairs} \label{subsection: pseudo_symmetric_pairs}
        In this subsection, we consider the case when:
            $$\vartheta \in \PseudoInv(\g, \h)$$
        i.e. when $\vartheta$ is a pseudo-involution. Such a weighted automorphism determines a quasi-symmetric pair in the sense of subsection \ref{subsection: quasi_symmetric_pairs} when it is of finite order, so let us focus on pseudo-involutions that are potentially infinite-order. In light of lemma \ref{lemma: weighted_automorphisms_permute_weights} and in order to circumvent the technical difficulty, whereby the eigenvalues of $\vartheta$ may not be a root of unity now, the following generalisation of the notion of fixed-point subalgebras was proposed in \cite[Definition 1.2]{regelskis_vlaar_kac_moody_pseudo_symmetric_pairs}. 
        \begin{definition}[Pseudo-fixed-points subalgebras] \label{def: pseudo_fixed_point_subalgebras}
            Let $(\g, \h)$ be a weighted Lie algebra, and fix some pseudo-involution $\vartheta \in \PseudoInv(\g, \h)$. A \textbf{pseudo-fixed-point subalgebra} of $\g$ is then a Lie subalgebra $\k$ such that:
            \begin{itemize}
                \item $\k \cap \h = \h^{\vartheta}$, and
                \item $\dim\left( \k \cap ( \g_{\alpha} + \vartheta(\g_{\alpha}) ) \right) = \dim \g_{\alpha}$ for each non-zero weight $\alpha \in \rootsystem(\g, \h)$.
            \end{itemize}
        \end{definition}
        \begin{remark}
            Before moving on, let us give a partial explanation of the motivation behind the second condition in definition \ref{def: pseudo_fixed_point_subalgebras}; we trust that the reader have found the first condition to be self-explanatory. From lemma \ref{lemma: weighted_automorphisms_permute_weights}, we know that:
                $$\vartheta(\g_{\alpha}) = \g_{\vartheta^*(\alpha)}$$
            and so:
                $$\vartheta( \g_{\alpha} + \vartheta(\g_{\alpha}) ) = \vartheta( \g_{\alpha} + \g_{\vartheta^*(\alpha)} ) = \g_{\vartheta^*(\alpha)} + \g_{\alpha} = \vartheta(\g_{\alpha}) + \g_{\alpha}$$
            wherein the last equality holds because $\vartheta^*|_{\h^*}^2 = \id_{\h^*}$ as a consequence of the assumption that $\vartheta|_{\h}^2 = \id_{\h}$ (cf. definition \ref{def: weighted_automorphisms_of_weighted_lie_algebras}). We see thus that the subspaces $\g_{\alpha} + \vartheta(\g_{\alpha}) \subset \g$ are fixed by $\vartheta$.
            
            \todo[inline]{Motivation for the definition of pseudo-fixed-point subalgebras.}
        \end{remark}
        \begin{remark}
            To be strict about technicalities, we must point out that our definition \eqref{def: pseudo_fixed_point_subalgebras} and \cite[Definition 1.2]{regelskis_vlaar_kac_moody_pseudo_symmetric_pairs} do not coincide in full generality, though they do tend to in practice. For them, $\h$ needs to be maximal amongst the Lie subalgebras of $\g$ that act semi-simply via the adjoint representation. For us, at least at the moment, in contrast, $\h$ merely has to act semi-simply via derivations on $\g$, which is a much weaker requirement. Later on, we will explain the need for $\h$ to be maximal in the sense above.
            
            We would also like to remark that definition \ref{def: pseudo_fixed_point_subalgebras} (as well as \cite[Definition 1.2]{regelskis_vlaar_kac_moody_pseudo_symmetric_pairs}) makes \textit{no mention of a possibility of uniqueness for pseudo-fixed-point subalgebras}.
        \end{remark}
    
        In what follows, we let $\k$ be the Lie subalgebra of $\g$ generated by the set of \textbf{$\vartheta$-pseudo-fixed root vectors}:
            \begin{equation} \label{equation: pseudo_fixed_point_subalgebra_generators}
                \begin{gathered}
                    \h^{\vartheta} \cup \{ b_{\alpha} \}_{\alpha \in \rootsystem}
                    \\
                    b_{\alpha} := \frac12 \delta_{\alpha > 0} \left( e_{\alpha} + \vartheta(e_{\alpha}) \right) \in \g_{\alpha} + \vartheta(\g_{\alpha}) \quad, \quad \alpha \in \rootsystem
                \end{gathered}
            \end{equation}
        wherein, for a statement $P$, we write:
            $$
                \delta_P =
                \begin{cases}
                    $1$ & \text{if $P$ is true}
                    \\
                    $0$ & \text{if $P$ is false}
                \end{cases}
            $$
        \begin{remark}
            The generators $b_{\alpha}$ in equation \eqref{equation: pseudo_fixed_point_subalgebra_generators} are not quite the same as the ones that appear in \cite{regelskis_vlaar_finite_QSPs_via_generalised_satake_diagrams}. The difference is not substantial, amounting to a renormalisation whenever $\alpha$ is not fixed by $\vartheta^*$; when a root $\alpha$ is fixed by $\vartheta^*$, this renormalisation yields us:
                $$b_{\alpha} = \frac12 \delta_{\alpha > 0} \left( e_{\alpha} + \vartheta(e_{\alpha}) \right) = \frac12 \delta_{\alpha > 0} \left( e_{\alpha} + e_{\alpha} \right) = \delta_{\alpha > 0} e_{\alpha}$$
            Thus, the renormalisation in \eqref{equation: pseudo_fixed_point_subalgebra_generators} saves us from having to perform an excessive amount of case-by-case computation.
        \end{remark}

        Most of the results below concerning $\k$ were worked out by Regelskis and Vlaar in \cite[Section 3]{regelskis_vlaar_kac_moody_pseudo_symmetric_pairs} in the (affine) Kac-Moody case, i.e. the case of \say{integrable} contragredient Lie algebras (cf. lemma \ref{lemma: integrable_root_decomposable_lie_algebras_are_symmetrisable_kac_moody_algebras}). Given the similarities between the Chevalley-Serre-style presentations for general contragredient Lie algebras and those for Kac-Moody algebras, what is presented below is only a mild generalisation of the results in \textit{loc. cit.}
        
        \begin{lemma} \label{lemma: underlying_vector_spaces_of_pseudo_fixed_point_subalgebras}
            The underlying vector space of $\k$ is:
                $$\k = $$
        \end{lemma}
            \begin{proof}
                
            \end{proof}
        
        For our own convenience, let us set:
            $$
                \begin{gathered}
                    e_{\alpha + \beta} := [e_{\alpha}, e_{\beta}] \in \g_{\alpha + \beta}
                    \\
                    b_{\alpha + \beta} := \frac12 \delta_{\alpha > 0} \delta_{\beta > 0} \left( e_{\alpha + \beta} + \vartheta(e_{\alpha + \beta}) \right) \in \g_{\alpha + \beta} + \vartheta(\g_{\alpha + \beta})
                \end{gathered}
                \quad, \quad \alpha, \beta \in \rootsystem
            $$

        \todo[inline]{IMPORTANT: commutators between pseudo-fixed root vectors.}
        First of all, let us observe that the generators \eqref{equation: pseudo_fixed_point_subalgebra_generators} of $\k$ satisfy the following commutation relations:
            \begin{equation} \label{equation: pseudo_fixed_point_subalgebra_relations}
                \begin{gathered}
                    \begin{aligned}
                        [k, b_{\alpha}] & = \frac12 \delta_{\alpha > 0} [ k, e_{\alpha} + \vartheta(e_{\alpha}) ]
                        \\
                        & = \frac12 \delta_{\alpha > 0} [ k, e_{\alpha} + \vartheta(e_{\alpha}) ]
                        \\
                        & = \frac12 \delta_{\alpha > 0} \left( \alpha(k) e_{\alpha} + \vartheta^*(\alpha(k)) \vartheta(e_{\alpha}) \right)
                        \\
                        & = \frac12 \delta_{\alpha > 0} \left( \alpha(k) e_{\alpha} + \alpha( \vartheta(k) ) \vartheta(e_{\alpha}) \right)
                        \\
                        & = \frac12 \delta_{\alpha > 0} \left( \alpha(k) e_{\alpha} + \alpha(k) \vartheta(e_{\alpha}) \right)
                        \\
                        & = \alpha(k) \cdot \frac12 \delta_{\alpha > 0} \left( e_{\alpha} + \vartheta(e_{\alpha}) \right)
                        \\
                        & = \alpha(k) b_{\alpha}
                    \end{aligned}
                    \quad, \quad k \in \h^{\vartheta}, \alpha \in \rootsystem
                    \\
                    \begin{aligned}
                        [b_{\alpha}, b_{\beta}] & = \frac14 \delta_{\alpha > 0} \delta_{\beta > 0} \left[ e_{\alpha} + \vartheta(e_{\alpha}), e_{\beta} + \vartheta(e_{\beta}) \right]
                        \\
                        & = \frac14 \delta_{\alpha > 0} \delta_{\beta > 0} \left( e_{\alpha + \beta} + [ \vartheta(e_{\alpha}), e_{\beta} ] + [ e_{\alpha}, \vartheta(e_{\beta}) ] + \vartheta(e_{\alpha + \beta}) \right)
                        \\
                        & = \frac12 b_{\alpha + \beta} + \frac14 \delta_{\alpha > 0} \delta_{\beta > 0} \left( [ \vartheta(e_{\alpha}), e_{\beta} ] + [ e_{\alpha}, \vartheta(e_{\beta}) ] \right)
                    \end{aligned}
                    \quad, \quad \alpha, \beta \in \rootsystem
                \end{gathered}
            \end{equation}
        which can be verified through some straightforward computations.

        \todo[inline]{Compute $[\k, \g]$ in order to get a $\k$-action $\k \tensor \g \to \g$. Dualising the action, one obtains a Lie coideal structure on pseudo-fixed point subalgebras.}
        
        \begin{proposition}[Action of pseudo-fixed-point subalgebras] \label{prop: adjoint_actions_of_pseudo_fixed_point_subalgebras}
            
        \end{proposition}
            \begin{proof}
                
            \end{proof}

        We seek conditions on $\k$ so that it is a $\vartheta$-pseudo-fixed-point subalgebra inside $\g$. Given the construction of $\k$, it is sufficient to only verify that the second condition in definition \ref{def: pseudo_fixed_point_subalgebras} is satisfied, namely that we have:
            $$\dim\left( \k \cap \left( \g_{\alpha} + \vartheta(\g_{\alpha}) \right) \right) = \dim \g_{\alpha} \quad, \quad \alpha \in \rootsystem$$
        while the first condition is an automatic consequence of the fact that $\h^{\vartheta}$ is a subset of the set of Lie generators for $\k$. To this end, let us recall first of all, from lemma \ref{lemma: weighted_automorphisms_permute_weights}, that:
            $$\vartheta(\g_{\alpha}) = \g_{\vartheta^*(\alpha)} \quad, \quad \alpha \in \rootsystem$$
        and so:
            $$
                \begin{aligned}
                    \k \cap \left( \g_{\alpha} + \vartheta(\g_{\alpha}) \right) & = \k \cap \left( \g_{\alpha} + \g_{\vartheta^*(\alpha)} \right)
                    \\
                    & =
                    \begin{cases}
                        \k \cap \g_{\alpha} & \text{if $\alpha \in \rootsystem^{\vartheta^*}$}
                        \\
                        (\k \cap \g_{\alpha}) \oplus (\k \cap \g_{-\alpha}) & \text{if $\alpha \in \rootsystem^{-\vartheta^*}$}
                    \end{cases}
                \end{aligned}
            $$
        \begin{proposition}[Constructing pseudo-fixed-point subalgebras] \label{prop: constructing_pseudo_fixed_point_subalgebras}
            
        \end{proposition}
            \begin{proof}
                
            \end{proof}
        \begin{corollary}[Pseudo-Iwasawa decompositions] \label{coro: pseudo_iwasawa_decompositions}
            
        \end{corollary}
            \begin{proof}
                
            \end{proof}

    \subsection{The Regelskis-Vlaar classification of affine Kac-Moody pseudo-symmetric pairs} \label{subsection: affine_kac_moody_pseudo_symmetric_pairs}
        In this subsection, we fix once and for all an affine\footnote{... either untwisted or twisted, in the sense of \cite[Chapter 8]{kac_infinite_dimensional_lie_algebras}.} Kac-Moody algebra:
            $$(\g, \h, \n^{\pm}, \omega)$$
        along with a pseudo-involution:
            $$\vartheta \in \PseudoInv(\g, \h)$$
    
        For pseudo-involutions of affine Kac-Moody algebras, as was originally worked out by Regelskis and Vlaar in \cite{regelskis_vlaar_kac_moody_pseudo_symmetric_pairs}, the structure of pseudo-fixed-point subalgebras is much more constrained. Namely, there is an interpretation of pseudo-involutions in terms of their actions on Borel subalgebras of affine Kac-Moody algebras, which allows us to classify pseudo-involutions in this setting. Thanks to the existence of such a classification of such pseudo-involutions $\vartheta$, the extra assumption in proposition \ref{prop: constructing_pseudo_fixed_point_subalgebras} can be replaced by a more succinct condition on the Satake diagram arising from $\vartheta$.

        Now, the classification scheme mentioned above is, broadly speaking, a consequence of the conjugacy of Borel subalgebras within affine Kac-Moody algebras. More precisely, Kac and Peterson proved in \cite{kac_peterson_infinite_flag_varieties_and_conjugacy_of_cartan_subalgebras}, by means of the conjugation action of , that when $\g$ either of finite\footnote{The finite-type case is a classical result of Borel.} or affine type, all Borel subalgebras are $\frakG$-conjugate; see also \cite{chernousov_egorov_gille_pianzola_cohomological_proof_of_peterson_kac_theorem} and \cite{chernousov_neher_pianzola_conjugacy_of_cartan_subalgebras_in_EALAs_with_non_fgc_centreless_cores}. 
    
        \begin{definition}[Integrable modules] \label{def: integrable_modules_over_triangular_decomposable_lie_algebras}
            Consider a triangular-decomposable Lie algebra $(\g, \h, \n^{\pm}, \omega)$. A $\g$-module given by $\pi: \g \to \End(V)$ is said to be \textbf{integrable} if for any $\alpha \in \rootlattice^+(\g, \h)$ and for any corresponding weight vector $e_{\alpha} \in \g_{\alpha}$, the operator $\pi(e_{\alpha}) \in \End(V)$ is locally nilpotent, which is to say that for all $v \in V$, there exists some $N_v > 0$ such that $\pi(e_{\alpha})^n \cdot v = 0$ for all $n \geq N_v$. More succinctly, we say that $V$ is integrable if $\n^+$ acts locally nilpotently on it.
        \end{definition}
        In the Kac-Moody case, this notion reduces to the same notion by the same name, as defined in \cite[Chapter 3]{kac_infinite_dimensional_lie_algebras}.
        \begin{definition}[Integrable Lie algebras] \label{def: integrable_lie_algebras}
            Consider a triangular-decomposable Lie algebra $(\g, \h, \n^{\pm}, \omega)$. If the subalgebras $\n^{\pm} \subset \g$ are locally nilpotent, then the aforementioned triangular-decomposable Lie algebra will be called \textbf{integrable}.
        \end{definition}
        It is necessarily to make this definition because in general, the \say{upper/lower-triangular} subalgebras:
            $$\n^{\pm} \subset \g$$
        of a triangular-decomposable Lie algebra $(\g, \h, \n^{\pm}, \omega)$ are \textit{not} necessarily locally nilpotent! In more combinatorial terms, in this general of a setting, it is entirely possible for the so-called \say{root strings} to be infinitely long. This is the reason for singling out the integrable Lie algebras from the larger class of triangular-decomposable ones. Somewhat surprisingly, though, integrability is a rather strong condition.
        \begin{lemma} \label{lemma: integrable_root_decomposable_lie_algebras_are_symmetrisable_kac_moody_algebras}
            Let $(\g, \h, \n^{\pm}, \omega)$ be a root-decomposable Lie algebra. This is a symmetrisable Kac-Moody algebra if and only if it is integrable in the sense of definition \ref{def: integrable_lie_algebras}.
        \end{lemma}
            \begin{proof}
                See \cite[Propsition 4.1.11]{moody_pianzola_lie_algebras_with_triangular_decompositions}.
            \end{proof}

        Given a triangular-decomposable Lie algebra $(\g, \h, \n^{\pm}, \omega)$, it is typical to refer to the subalgebras $\b^{\pm} := \h \oplus \n^{\pm}$ as the \textbf{positive/negative Borel subalgebras} of $\g$. More generally, Borel subalgebras of an arbitrary Lie algebra are those Lie subalgebras which are maximal amongst the solvable ones.

        \begin{proposition}[Minimal Kac-Moody groups] \label{prop: minimal_kac_moody_groups}
            \todo[inline]{Minimal pro-algebraic groups associated to symmetrisable Kac-Moody algebras. See \cite{perrin_kac_moody_algebras}.}
        \end{proposition}

        \begin{definition}[Weighted automorphisms of types I and II] \label{def: weighted_automorphisms_of_types_I_and_II}
            Let $(\g, \h, \n^{\pm}, \omega)$ be a symmetrisable Kac-Moody algebra, and let $\frakG$ be the associated minimal pro-algebraic group. Let $\Ad: \frakG \to \Aut_{\Grp}( \Aut(\g, \h) )$ be the action by conjugation. Given a weighted automorphism $\vartheta \in \Aut(\g, \h)$, if $\vartheta \in \Ad(\frakG) \cdot \omega$, then we will say that it is of type II. Otherwise, it is said to be of type I.
        \end{definition}
        \begin{remark}[Weighted automorphisms of finite-type Kac-Moody algebras] \label{remark: finite_type_kac_moody_algebras_only_have_type_II_weighted_automorphisms}
            We keep the notations as in definition \ref{def: weighted_automorphisms_of_types_I_and_II}, and we suppose that $\g$ is of finite type (i.e. it is a finite-dimensional simple Lie algebra). In this case, it is known that:
                $$\Ad(\frakG) \cdot \omega = \Aut(\g, \h)$$
            and therefore, all weighted automorphisms are of type II. We interpret this fact as a reaffirmation of the fact from \cite[Chapter 8]{kac_infinite_dimensional_lie_algebras} that finite-type Kac-Moody algebras do not admit twisted forms, and will elaborate on this perspective in theorem \ref{theorem: twisted_forms_and_coideal_subalgebras_of_kac_moody_lie_bialgebras}. 
        \end{remark}