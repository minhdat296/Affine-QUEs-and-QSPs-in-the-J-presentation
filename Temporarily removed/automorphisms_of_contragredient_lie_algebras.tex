\section{Automorphisms of contragredient Lie algebras}
    \subsection{Weighted Lie algebras, triangular-decomposable, and contragredient Lie algebras}
        Ideally, we would like to develop a theory of Manin triples $(\a, \a^+, \a^-)$ that are somehow twisted by general Lie algebra automorphisms $\vartheta \in \Aut_{\Lie\Alg}(\a)$, but at present, we are unfortunately at a loss for how to write down a description of such twisted Manin triples. Guided by the theory of quantum symmetric pairs that arise from so-called \say{pseudo-involutions} on affine Kac-Moody algebras (see e.g. \cite{regelskis_vlaar_reflection_matrices_coideal_subalgebras}, \cite{kolb_kac_moody_QSPs}, and references therein), we have chosen to restrict our consideration down to the class of Lie algebras that admit so-called \say{triangular decompositions}. The structure and representation theory of this class of Lie algebras has been studied extensively, notably in \cite{moody_pianzola_lie_algebras_with_triangular_decompositions}, and thanks to the fact that they admit triangular decompositions, one can speak of Borel subalgebras; the automorphisms that we shall consider shall be the ones that either stabilise or permute pairs of Borel subalgebras.
        
        Let us begin by recalling the construction of triangular-decomposable\footnote{In \textit{loc. cit.}, such Lie algebras are called \say{Lie algebras admitting triangular decompositions}, but we find the terminology somewhat cumbersome.} Lie algebras from \cite[Section 2.1]{moody_pianzola_lie_algebras_with_triangular_decompositions}. Their construction starts with what we shall refer to as their \say{weighing subalgebras}: this is a pair:
            $$(\h, V)$$
        consisting of a non-zero abelian Lie algebra $\h \not = 0$ and a choice of a representation $\h \to \gl(V)$. Then, for any linear functional $\lambda \in \h^*$, a \say{weight vector} of weight $\lambda$ shall be an element $v \in V$ that is a simultaneous eigenvector of all the scalars in the set $\{ \lambda(h) \}_{h \in \h}$, i.e.:
            $$h \cdot v = \lambda(h) v \quad, \quad h \in \h$$
        For any $\lambda \in \h^*$, the vector subspace of $V$ spanned by weight-$\lambda$ vectors, i.e.:
            $$V_{\lambda} := \{ v \in V \mid \forall h \in \h: h \cdot v = \lambda(h) v \}$$
        is called the \say{subspace of weight $\lambda$}. If $V = \sum_{\lambda \in \h^*} V_{\lambda}$ as vector subspaces of $V$, then we shall say that $V$ admits a \say{weight space decomposition} with respect to $\h$; \textit{a priori}, this sum is necessarily direct if it exists, i.e.:
            $$V = \bigoplus_{\lambda \in \h^*} V_{\lambda}$$
        (see \cite[Section 2.1, Proposition 1]{moody_pianzola_lie_algebras_with_triangular_decompositions}). In order to avoid redundancy, let us refer to only the functionals $\lambda \in \h^*$ such that $V_{\lambda} \not = 0$ as \say{weights} of $V$, and the subset of $\h^*$ consisting of weights of $V$ is denoted by\footnote{$\Pi$ for \say{poid}.} $\weight(V, \h)$.

        Next, consider another Lie algebra $\g$, on which our weighing algebra $\h$ acts via derivations, i.e. there exists a Lie algebra homomorphism $\h \to \der(\g)$, such that there exists a weight space decomposition:
            $$\g = \bigoplus_{\alpha \in \h^*} \g_{\alpha}$$
        It is then possible to show that:
            $$[\g_{\alpha}, \g_{\beta}] \subseteq \g_{\alpha + \beta} \quad, \quad \alpha, \beta \in \h^*$$
        and consequently, the Lie algebra $\g$ is graded by the abelian group:
            $$\rootlattice(\g, \h) := \Z\weight(\g, \h)$$
        commonly called the \say{root lattice} of $\g$. Conversely, it is also possible to show that if there exists a set of weights $P \subset \weight(\g, \h)$ and a set of weight vectors $\{ e_{\alpha} \}_{\alpha \in P}$ such that $\g$ is generated as a Lie algebra by $\h \cup \{ e_{\alpha} \}_{\alpha \in P}$, then $\g = \bigoplus_{\alpha \in \h^*} \g_{\alpha}$ wherein $e_{\alpha} \in \g_{\alpha}$ for all $\alpha \in P$. For details on these facts, see \cite[Subsection 2.1, Proposition 2]{moody_pianzola_lie_algebras_with_triangular_decompositions}. Therefore, it makes sense to refer to a Lie algebra $\g$ generated by weight vectors and together with a weighing algebra $\h$ in the manner above as a Lie algebra \say{weighted} by $\h$. Let us record this.
        \begin{definition}[Weighted Lie algebras] \label{def: weighted_lie_algebras}
            A \textbf{weighted Lie algebra} is a pair:
                $$(\g, \h)$$
            consisting of a Lie algebra $\g$ and an abelian Lie algebra $\h$, which acts semi-simply on $\g$ by derivations.
        \end{definition}
        \begin{remark}
            For a Lie algebra $\g$ weighted by $\h$, we have that:
                $$\h \subseteq \g_0$$
            if $\h$ acts by inner derivations. Equality occurs if $\h$ is maximal amongst the abelian Lie subalgebras of $\g$ whose adjoint actions on $\g$ is semi-simple.
        \end{remark}
        \begin{lemma}[Direct sums of weighted Lie algebras] \label{lemma: direct_sums_of_weighted_lie_algebras}
            Let $(\g_1, \h_1)$ and $(\g_2, \h_2)$ be weighted Lie algebras via $\rho_1: \h_1 \to \der(\g_1)$ and $\rho_2: \h_2 \to \der(\g_2)$ respectively. Then, $(\g_1 \oplus \g_2, \h_1 \oplus \h_2)$ will also be a weighted Lie algebra, namely via $\rho_1 \oplus \rho_2: \h_1 \oplus \h_2 \to \der(\g_1 \oplus \g_2)$.
        \end{lemma}
            \begin{proof}
                Clear as stated.
            \end{proof}

        \begin{definition}[Triangular-decomposable Lie algebras] \label{def: triangular_decomposable_lie_algebras}
            A \textbf{triangular-decomposable Lie algebra} is a weighted Lie algebra $(\g, \h)$ satisfying the following additional conditions.
            \begin{itemize}
                \item There exists non-zero Lie subalgebras $\n^{\pm} \subset \g$ such that $\g \cong \n^- \oplus \h \oplus \n^+$ as Lie algebras.
                \item There exists an involution $\omega: \g \xrightarrow[]{\cong} -\g$ (commonly called the \textbf{Chevalley involution}) such that $\omega(\n^{\pm}) = -\n^{\mp}$ and $\omega|_{\h} = \id_{\h}$.
                \item $\n^{\pm}$ are $\h$-submodules of $\g$, i.e. $[\h, \n^{\pm}] \subseteq \n^{\pm}$. The Lie subalgebras $\n^{\pm}$ thus admit the induced weight space decompositions $\n^{\pm} = \bigoplus_{\alpha \in \weight(\g, \h)} \n^{\pm} \cap \g_{\alpha}$, and let us require furthermore that:
                    $$\rootlattice^{\pm}(\g) := \{ \alpha \in \weight(\g, \h) \setminus \{0\} \mid \n^{\pm} \cap \g_{\alpha} \not = 0 \}$$
                are free (additive) sub-semigroups of $\rootlattice(\g, \h) := \Z\weight(\g, \h)$.
                \item Finally, we require that there is a semi-group basis\footnote{... which exists because we are pre-supposing that $\rootlattice^+(\g, \h)$ is free.} $\simpleroots(\g, \h) \subset \rootlattice^+(\g, \h)$ which is also a basis for $\h^*$.
            \end{itemize}
            A triangular-decomposable Lie algebra as above shall be denoted as a quintuple:
                $$(\g, \h, \n^+, \n^-, \omega)$$
            (though we shall usually abbreviate the signs and write $(\g, \h, \n^{\pm}, \omega)$ instead).
        \end{definition}
        \begin{convention}[Root systems]
            In this context, it is customary to refer to the non-zero weights\footnote{Recall that we only consider elements $\alpha \in \h^*$ such that $\g_{\alpha} \not = 0$ as weights.} of the set:
                $$\rootsystem(\g, \h) := \weight(\g, \h) \setminus \{0\}$$
            as \textbf{roots}. If $\g$ is equipped with a symmetric, non-degenerate, and invariant bilinear form $(\cdot, \cdot)_{\g}$ (cf. definition \ref{def: root_decomposable_lie_algebras}), then the triple:
                $$( \h, \rootsystem(\g, \h), (\cdot, \cdot)_{\g} )$$
            is known as the \textbf{root system} of $(\g, \h, \n^{\pm}, \omega)$, though it is typical to just refer to the set $\rootsystem(\g, \h)$. Weight spaces of the adjoint representation are typically called \textbf{root spaces}. The elements of the subsets:
                $$\rootsystem^{\pm}(\g, \h) := \rootsystem(\g, \h) \cap \rootlattice^{\pm}(\g, \h)$$
            are called \textbf{positive/negative roots}, respectively, and the elements of a chosen semi-group basis $\simpleroots(\g, \h) \subset \rootlattice^+(\g, \h)$ are called \textbf{simple roots} (in particular, this means that simple roots are positive).
        \end{convention}
        \begin{remark}
            Note that by requiring that $0 \not \in \rootlattice^{\pm}(\g)$, we have that:
                $$\rootlattice^-(\g, \h) \cap \rootlattice^+(\g, \h) = \varnothing$$

            Also, there is no guarantee in general that the root spaces $\g_{\alpha} \subset \g$ are finite-dimensional. Triangular-decomposable Lie algebras with this property are said to be \textbf{regular}. Note that this property needs not imply that the Lie subalgebras $\n^{\pm}$ are nilpotent: for instance, extended Kac-Moody algebras (given by \eqref{equation: extended_kac_moody_relations}; cf. also \cite[Theorem 1.2]{kac_infinite_dimensional_lie_algebras}) are regular in the above sense, but their generators are not subjected to the Serre relations.
        \end{remark}
        \begin{example}
            By their very construction (e.g. as explained in \cite[Theorem 1.2]{kac_infinite_dimensional_lie_algebras}), Kac-Moody algebras are triangular-decomposable. Via the Serre relations, one sees also that symmetrisable Kac-Moody algebras are actually triangular-decomposable.
            
            Finite-dimensional but non-Kac-Moody examples of triangular-decomposable Lie algebras include reductive Lie algebras, while infinite-dimensional non-Kac-Moody examples include Heisenberg algebras and Virasoro algebras, extended affine Lie algebras (EALAs) in the sense of \cite{neher_lectures_on_EALAs}, as well as the higher-nullity analogues of the Heisenberg and Virasoro algebras that arise via EALAs. Triangular-decomposable Lie algebras form a very large class.

            On the other hand, solvable Lie algebras for instance - and thus also the Lie algebras that are nilpotent, and especially the abelian ones - are \textit{not} triangular decomposable.
        \end{example}
        \begin{example}
            Weighted Lie algebras need not be triangular-decomposable, even when the weighing algebra acts by inner derivations. For instance, if $(\g, \h, \n^{\pm}, \omega)$ is triangular-decomposable Lie algebra, then by definition \ref{def: triangular_decomposable_lie_algebras} the two Borel subalgebras $\b^{\pm} := \h \oplus \n^{\pm}$ are weighted by the adjoint action of $\h$, but they are not triangular-decomposable with respect to the same $\h$-action.
        \end{example}

        \begin{definition}[Contragredient Lie algebras] \label{def: contragredient_lie_algebras}
            Let $(\g, \h, \n^{\pm}, \omega)$ be a triangular-decomposable Lie algebra. This is said to be \textbf{contragredient} if the following conditions are satisfied.
            \begin{itemize}
                \item Let $\simpleroots(\g, \h)$ be a set of simple roots\footnote{... whose existence is stipulated by definition \ref{def: triangular_decomposable_lie_algebras}.}. Then, for each $\alpha \in \simpleroots(\g, \h)$, the Lie subalgebra $\sl_2^{(\alpha)} \subset \g$ generated by the simple root spaces $\g_{\pm \alpha}$ is isomorphic to $\sl_2$.
                \item One can choose a a set of root vectors $\{ e_{\alpha} \}_{\alpha \in \rootsystem(\g, \h)}$ such that the set $\h \cup \{ e_{\alpha} \}_{\alpha \in \rootsystem(\g, \h)}$ generates the Lie algebra $\g$.
                \item The sum $\sum_{\alpha \in \simpleroots(\g, \h)} [\g_{\alpha}, \g_{-\alpha}]$ is direct.
            \end{itemize}
        \end{definition}
        It is rather convenient to have the following fact at our disposal.
        \begin{lemma}[Simple root spaces are $1$-dimensional] \label{lemma: simple_root_spaces_are_1_dimensional}
            Let $(\g, \h, \n^{\pm}, \omega)$ be a contragredient Lie algebra, and let $\simpleroots(\g, \h)$ be a choice of simple roots for it. Then, for every $\alpha \in \simpleroots(\g, \h)$, we have that:
                $$\dim \g_{\pm \alpha} = 1$$
        \end{lemma}
            \begin{proof}
                See \cite[Section 4.1, p. 311]{moody_pianzola_lie_algebras_with_triangular_decompositions}.
            \end{proof}
        \begin{corollary}[$\sl_2$-triples] \label{coro: sl_2_triples}
            With notations as in lemma \ref{lemma: simple_root_spaces_are_1_dimensional}, it is possible to choose root vectors $e_{\pm \alpha} \in \g_{\pm \alpha}$ for each $\alpha \in \simpleroots(\g, \h)$ so that one has a Lie algebra isomorphism:
                \begin{equation} \label{equation: sl_2_triples}
                    \sl_2^{(\alpha)} \xrightarrow[]{\cong} \sl_2
                \end{equation}
            given by $e_{\pm \alpha} \mapsto e^{\pm}$ and $\alpha^{\vee} := [e_{\alpha}, e_{-\alpha}] \mapsto h$, with $e^+ := \begin{pmatrix} 0 & 1 \\ 0 & 0 \end{pmatrix}, e^- := \begin{pmatrix} 0 & 0 \\ 1 & 0 \end{pmatrix}$, and $h := \begin{pmatrix} 1 & 0 \\ 0 & -1 \end{pmatrix}$ making .
        \end{corollary}
            \begin{proof}
                This merely requires us to choose the root vectors $e_{\pm \alpha} \in \g_{\pm \alpha}$ so that:
                    $$[\alpha^{\vee}, e_{\pm \alpha}] = \pm 2 e_{\pm \alpha}$$
            \end{proof}

        \todo[inline]{Chevalley-Serre presentation for contragredient Lie algebras}
        
        \begin{definition}[Root-decomposable Lie algebras] \label{def: root_decomposable_lie_algebras}
            A contragredient Lie algebra $(\g, \h, \n^{\pm}, \omega)$ is said to be \textbf{root-decomposable} if it is equipped with a non-degenerate, symmetric, and invariant bilinear form $(\cdot, \cdot)_{\g}$ that is $\omega$-invariant, i.e.:
                \begin{equation} \label{equation: cartan_invariant_pairing_on_root_decomposable_lie_algebras}
                    (\omega(x), y)_{\g} = (x, \omega(y))_{\g} \quad, \quad x, y \in \g
                \end{equation}
        \end{definition}
        \begin{example}[Symmetrisable Kac-Moody algebras are root-decomposable] \label{example: symmetrisable_kac_moody_algebras_are_root_decomposable}
            From \cite[Theorem 1.2 and Lemma 3.2]{kac_infinite_dimensional_lie_algebras}, we know that general Kac-Moody algebras are contragredient in the sense of definition \ref{def: contragredient_lie_algebras} by construction. Those Kac-Moody algebras whose Cartan matrices are symmetrisable, moreover, are root-decomposable.
            
            To see why, suppose firstly that $\g$ is the Kac-Moody algebra associated to a symmetrisable $n \x n$ Cartan matrix:
                $$C = DA$$
            (cf. equation \eqref{equation: cartan_matrix_entries}), generated by the set:
                $$\h \cup \{e_i^{\pm}\}_{1\leq i \leq n}$$
            (cf. equation \ref{equation: kac_moody_generators}) subjected to the relations \eqref{equation: extended_kac_moody_relations} and \eqref{equation: kac_moody_serre_relations}. First of all, it is known that there is a non-degenerate, symmetric, and invariant bilinear form $(\cdot, \cdot)_{\g}$ given as in equations \eqref{equation: kac_moody_pairing_on_cartan_subalgebras} and \eqref{equation: kac_moody_pairing}. Secondly, it is known (see e.g. \cite[Theorem 1.2]{kac_infinite_dimensional_lie_algebras}) that there is a Chevalley involution on $\g$ given by:
                \begin{equation} \label{equation: symmetrisable_kac_moody_chevalley_involution}
                    \begin{gathered}
                        \omega(e_i^{\pm}) = -e_i^{\mp} \quad, \quad 1 \leq i \leq n
                        \\
                        \omega(h) = -h \quad, \quad h \in \h
                    \end{gathered}
                \end{equation}
            We can then make the following computations:
                \begin{equation} \label{equation: kac_moody_pairing_cartan_equivariance}
                    \begin{gathered}
                        \left( \omega(e_i^{\pm}), e_j^{\pm} \right)_{\g} = \left( -e_i^{\mp}, e_j^{\pm} \right)_{\g} = -\delta_{i, j} D_{i, i}^{-1} = \left( e_i^{\pm}, -e_j^{\mp} \right)_{\g} = \left( e_i^{\pm}, \omega(e_j^{\pm}) \right)_{\g} \quad, \quad 1 \leq i, j \leq n
                        \\
                        \left( \omega(h), h' \right)_{\g} = \left( -h, h' \right)_{\g} = \left( h, -h' \right)_{\g} = \left( h, \omega(h') \right)_{\g} \quad, \quad h, h' \in \h
                    \end{gathered}
                \end{equation}
            We have thus verified that symmetrisable Kac-Moody algebras are also root-decomposable in the sense of definition \ref{def: root_decomposable_lie_algebras}.
        \end{example}

        \begin{remark}[Isotropic subalgebras of root-decomposable Lie algebras] \label{remark: isotropic_subalgebras_of_root_decomopsable_lie_algebras}
            According to \cite[Proposition 4.4.2]{moody_pianzola_lie_algebras_with_triangular_decompositions}, the following are true for a contragredient Lie algebra $(\g, \h, \n^{\pm}, \omega)$ equipped with a symmetric and invariant bilinear form $(\cdot, \cdot)_{\g}$.
            \begin{enumerate}
                \item For all $\alpha, \beta \in \weight(\g, \h)$, one has that:
                    $$( \g_{\alpha}, \g_{\beta} )_{\g} = 0 \iff \alpha + \beta \not = 0$$
                i.e. when regarded as a symmetric $2$-tensor, $(\cdot, \cdot)_{\g}$ is of total degree $0$ with respect to the $\rootlattice(\g, \h)$-grading on $\Sym^2(\g)^{\g}$ induced by the one on $\g$.
                \item $(\cdot, \cdot)_{\g}$ is non-degenerate if and only if for all $\alpha \in \weight(\g, \h)$, the restriction $(\cdot, \cdot)_{\g}|_{\g_{-\alpha} \x \g_{\alpha}}$ is non-degenerate.
            \end{enumerate}

            Combining these facts with the conditions for root-decomposable Lie algebras from definition \ref{def: root_decomposable_lie_algebras}, one sees that for such Lie algebras - say $(\g, \h, \n^{\pm}, \omega)$ - one has that:
            \begin{enumerate}
                \item $(\n^{\pm}, \n^{\pm})_{\g} = 0$, and
                \item that the restriction $(\cdot, \cdot)_{\g}|_{\h \x \h}$ is non-degenerate (note that we need $\g_0 = \h$ for this to be true, hence the first condition in definition \ref{def: root_decomposable_lie_algebras}).
            \end{enumerate}
            This is a rather important fact, as it allows us to construct Manin triples of the form:
                $$(\a, \eta^+(\b^+), \eta^-(\b^-))$$
            that generalise the construction of extended Kac-Moody Manin triples as in subsection \ref{subsection: setup_standard_kac_moody_lie_bialgebras}, with $\b^{\pm} := \h \oplus \n^{\pm}$ being embedded \textit{non-trivially} into $\a := \b^+ \oplus \b^-$ via $\eta^{\pm}$.
        \end{remark}

    \subsection{Weighted automorphisms}
        \begin{definition}[Weighted automorphisms and pseudo-involutions] \label{def: weighted_automorphisms_of_weighted_lie_algebras}
            Consider a weighted Lie algebra $(\g, \h)$. Next, consider a Lie algebra automorphism $\vartheta \in \Aut_{\Lie\Alg}(\g)$ under which $\h$ is stable, i.e. it is such that $\vartheta|_{\hbar} \in \Aut_{\Lie\Alg}(\h)$. In such a situation, we say that this is an automorphism of $\g$ \textbf{weighted} by $\h$.
            
            A weighted automorphism $\vartheta \in \Aut(\g, \h)$ that is an involution, i.e. $\vartheta^2 = \id_{\g}$, is said to be a \textbf{weighted involution}. If we only have that $\vartheta|_{\h}^2 = \id_{\h}$, then $\vartheta$ will be called a \textbf{pseudo-involution}. 
        \end{definition}
        It is clear that weight automorphisms, pseudo-involutions, weighted involutions, and involutions of a given weighted Lie algebra $\g$ form subgroups of $\Aut_{\Lie\Alg}(\g)$. We denote these subgroups, respectively, by:
            $$\Aut(\g, \h) \quad \supset \quad \PseudoInv(\g, \h) \quad \supset \quad \Inv(\g, \h) \quad \subset \quad \Inv_{\Lie\Alg}(\g)$$
        and we note that $\Inv(\g, \h) := \Inv_{\Lie\Alg}(\g) \cap \Aut(\g, \h)$ just by definition.
        \begin{remark}
            The notion of pseudo-involution above coincides with \cite[Definition 1.1]{regelskis_vlaar_kac_moody_pseudo_symmetric_pairs}. Additionally, note that the Chevalley involution $\omega \in \Aut(\g, \h)$ of a given triangular-decomposable Lie algebra $(\g, \h, \n^{\pm}, \omega)$ is in fact a weighted involution by definition, and the notion below is given \textit{relatively} to this choice of an involution on $\g$.
        \end{remark}
        
        It is worth noting, that in the following sense, one ought to think of the notion of weighted automorphisms as an inherently combinatorial one.
        \begin{lemma}[Weighted automorphisms permute weights] \label{lemma: weighted_automorphisms_permute_weights}
            Let $(\g, \h)$ be a weighted Lie algebra, say with weight space decomposition $\g = \bigoplus_{\alpha \in \weight(\g, \h)} \g_{\alpha}$, and consider a weighted automorphism $\vartheta \in \Aut(\g, \h)$. Then:
                $$\vartheta(\g) = \bigoplus_{\alpha \in \weight(\g, \h)} \g_{\vartheta^*(\alpha)}$$
            and hence:
                $$\vartheta^* \in \Aut_{\Sets}( \weight(\g, \h) )$$
        \end{lemma}
            \begin{proof}
                First of all, for an arbitrary $\alpha \in \weight(\g, \h)$ and $x \in \g_{\alpha}$, let us compute the weight of $\vartheta(x)$. To this end, consider the following, for any $h \in \h$:
                    $$[h, \vartheta(x)] = \vartheta\left( [\vartheta^{-1}(h), x] \right) = \vartheta\left( \alpha( \vartheta^{-1}(h) ) \cdot x \right) = \alpha( \vartheta^{-1}(h) ) \vartheta(x)$$
                wherein the last equality holds because we are assuming that $\vartheta|_{\h} \in \Aut_{\Lie\Alg}(\h)$, and hence $\vartheta^{-1}(h) \in \h$ in particular. The above tells us that:
                    $$\vartheta(x) \in \g_{\alpha \circ \vartheta}$$
                
                It remains to show that $\alpha \circ \vartheta \in \h^*$ is a weight. For this, note firstly that:
                    $$\alpha \circ \vartheta = \vartheta^*(\alpha)$$
                wherein $\vartheta^*: \h^* \to \h^*$ is the dual of $\vartheta$. From this, we infer that:
                    $$\vartheta(x) \in \g_{\vartheta^*(\alpha)}$$
                Now, $\vartheta \in \Aut_{\Lie\Alg}(\g)$ is assumed to satisfy $\vartheta(\h) = \h$, so $\vartheta^*(\h^*) = \h^*$ as well. We now know that $\vartheta^*: \h^* \to \h^*$ is a linear isomorphism, and hence injective in particular. Therefore, $\vartheta^*(\alpha) = 0$ if and only if $\alpha = 0$. Since we know that $\h \subseteq \g_0$, we therefore see that $\g_{\vartheta^*(\alpha)}$ is a weight space for all $\alpha \in \weight(\g, \h)$. We are also assuming that $\g = \bigoplus_{\alpha \in \weight(\g, \h)} \g_{\alpha}$, so by applying $\vartheta$ to both sides, we shall yield:
                    $$\g = \vartheta(\g) = \bigoplus_{\alpha \in \weight(\g, \h)} \g_{\vartheta^*(\alpha)}$$
                This then tells us that there is an automorphism of $\weight(\g, \h)$ given by $\vartheta^*|_{\weight(\g, \h)}$.
            \end{proof}