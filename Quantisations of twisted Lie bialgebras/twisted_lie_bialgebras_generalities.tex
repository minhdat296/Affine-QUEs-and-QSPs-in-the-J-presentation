\section{Twisting Lie-bialgebraic structures}
    \subsection{Twisted Manin triples and twisted topological Lie bialgebras} \label{subsection: twisted_manin_triples}
        For a recollection of basic facts about Manin triples, we refer the reader to subsection \ref{subsection: manin_triples}. We freely employ the notation algebraic space therein.

        Inspired by the theory of symmetric spaces, and following \cite{belliard_crampe_coideal_subalgebras_from_twisted_manin_triples}, we begin with the following definition.
        \begin{definition}[Twisted Manin triple] \label{def: twisted_manin_triples}
            Let:
                $$(\a, \a^+, \a^-)$$
            be a Manin triple (cf. definition \ref{def: manin_triples}) and consider a Lie algebra automorphism:
                $$\vartheta \in \Aut_{\LA}(\a)$$
            \begin{itemize}
                \item We say that $\vartheta$ is a \textbf{polar} (respectively, \textbf{anti-polar}) as a Manin triple automorphism if $\vartheta(\a^{\pm}) = \a^{\pm}$ (respectively, if $\vartheta(\a^{\pm}) = \a^{\mp}$).
                \item We say that $\vartheta$ \textbf{twists} the Manin triple above \textbf{(anti-)invariantly} if:
                    $$(\vartheta(x), y)_{\a} = \pm (x, \vartheta(y))_{\a}$$
                with the \say{$+$} sign corresponding to the invariant case, while the \say{$-$} sign corresponds to the anti-invariant case.
            \end{itemize}
        \end{definition}
        \begin{remark}
            Even when $\vartheta$ is a Lie algebra involution, our definition remains more general than \cite[Definition 2.1]{belliard_crampe_coideal_subalgebras_from_twisted_manin_triples}. Part of the reason for not insisting that the automorphism $\vartheta$ respects the polarisation of the Manin triple, i.e. not requiring that $\vartheta(\a^{\pm}) = \a^{\pm}$, is because we are interested also in examples of automorphisms such that $\vartheta(\a^{\pm}) = \a^{\mp}$. However, if we insist that $\vartheta$ respects the polarisation, then lemma \ref{lemma: twisted_manin_triples_and_twisted_topological_lie_bialgebras} in particular will only hold in a more restricted sense, which creates unnecessary inflexibility in our opinion.

            Interestingly, we will see through lemma \ref{lemma: manin_triple_twists_and_duality}, that an anti-invariant twist of a Manin triple $(\a, \a^+, \a^-)$ by an automorphism $\vartheta \in \Aut_{\LA}(\a)$ produces a topological Lie coideal subalgebra structure on the fixed-point subalgebra $\k := \a^{\vartheta}$ if and only if $\vartheta$ is a Lie algebra involution. That said, we are able to extend the results from \cite{belliard_crampe_coideal_subalgebras_from_twisted_manin_triples} to the case wherein $\vartheta$ is an \textit{anti-polar} involution that twists anti-invariantly. Note also that there are also involutions that twist \textit{invariantly}; see lemma \ref{lemma: manin_triple_twists_and_duality} for details.
        \end{remark}

        Recall that by duality, any Manin triple $(\a, \a^+, \a^-)$ gives rise to a topological Lie bialgebra structure $\delta: \a \to \a \hattensor \a$ by means of:
            $$\left( \delta(x), y \tensor z \right)_{\a \hattensor \a} = \left( x, [y, z] \right)_{\a} \quad, \quad x, y, z \in \a$$
        wherein $(\cdot, \cdot)_{\a \hattensor \a}$ is the factor-wise extension of the bilinear form $(\cdot, \cdot)_{\a}$ to $\a \hattensor \a$ (cf. equation \eqref{equation: lie_cobrackets_by_duality}). If $(\a, \a^+, \a^-)$ is twisted by a Lie algebra automorphism $\vartheta \in \Aut_{\LA}(\a)$, then:
            \begin{equation} \label{equation: twisting_lie_bialgebras_by_duality}
                \begin{aligned}
                    \pm \left( x, \vartheta([y, z]) \right)_{\a} & = \pm \left( x, [\vartheta(y), \vartheta(z)] \right)_{\a}
                    \\
                    & = \left( \vartheta(x), [y, z] \right)_{\a}
                    \\
                    & = \left( \delta( \vartheta(x), y \tensor z ) \right)_{\a \hattensor \a}
                \end{aligned}
            \end{equation}
        This prompts the following definition.
        \begin{definition}[Twisted topological Lie bialgebras] \label{def: twisted_topological_lie_bialgebras}
            Let $(\a, \delta)$ be a topological Lie bialgebra, and consider a Lie algebra automorphism:
                $$\vartheta \in \Aut_{\LA}(\a)$$
            This is an \textbf{(anti-)invariant twist} of the Lie bialgebra structure $\delta$ if:
                $$\delta \circ \vartheta = \pm \vartheta^{\tensor 2} \circ \delta$$
            and like in definition \ref{def: twisted_manin_triples}, the \say{$+$} sign corresponding to the invariant case, while the \say{$-$} sign corresponds to the anti-invariant case.
        \end{definition}
        \begin{remark}
            An (anti-)invariant twist of a topological Lie bialgebra $(\a, \delta)$ by $\vartheta \in \Aut_{\LA}(\a)$ is the same as a topological Lie bialgebra (anti-)automorphism. However, it is more convenient to work with twists, especially in the anti-invariant case. This is because the codomain of an anti-automorphism of $\a$ is $\a^{\op}$, while the codomain of an automorphism of $\a$ is $\a$, so we are of the opinion, that in order to avoid notational clutter, it is better to work with the notion of twists as in definition \ref{def: twisted_topological_lie_bialgebras}.
        \end{remark}
        For book-keeping purposes, let us record the following technical lemma. See also remark \ref{remark: manin_triple_twists_via_currying} below.
        \begin{lemma}[Twisted Manin triples and twisted topological Lie bialgebras] \label{lemma: twisted_manin_triples_and_twisted_topological_lie_bialgebras}
            Let $(\a, \a^+, \a^-)$ be a Manin triple and let $\delta: \a \to \a \hattensor \a$ be the Lie bialgebra structure that it defines. Let $\vartheta \in \Aut_{\LA}(\a)$ be a Lie algebra automorphism. Then, $\vartheta$ twists the Manin triple $(\a, \a^+, \a^-)$ invariantly (respectively, anti-invariantly) if and only if it twists $\delta$ invariantly (respectively, anti-invariantly).
        \end{lemma}
            \begin{proof}
                This is clear by consideration of equation \eqref{equation: twisting_lie_bialgebras_by_duality} along with the correspondence \eqref{equation: manin_triple_lie_bialgebra_correspondence} between Manin triples and topological Lie bialgebras.
            \end{proof}

        The central question that we would like to address in this subsection is the following. An answer will be given by theorem \ref{theorem: twisted_lie_bialgebraic_structures}.
        \begin{question} \label{question: twisted_topological_lie_bialgebras}
            If $\vartheta \in \Aut_{\LA}(\a)$ is a twist of a general topological Lie bialgebra $(\a, \delta)$, then what sort of Lie-bialgebra structure:
                $$\delta^{\vartheta}$$
            will it induce onto the fixed-point subalgebra $\k := \a^{\vartheta}$ with ?
        \end{question}
        
        To provide an answer to question \ref{question: twisted_topological_lie_bialgebras}, we begin with an easy but useful lemma about the eigenspaces of a Lie algebra automorphism.
        \begin{lemma}[Symmetric space decompositions] \label{lemma: symmetric_space_decompositions}
            Let $\a$ be a Lie algebra being acted on by an automorphism $\vartheta \in \Aut_{\LA}(\a)$ and let:
                $$\a := \k \oplus \bigoplus_{ \mu \in \weight(\vartheta), \mu \not = 1 } \p_{\mu}$$
            be the eigenspace decomposition\footnote{See convention \ref{conv: endomorphisms} for an explanation of the notations.} induced by $\vartheta$; also, let $\p := \bigoplus_{ \mu \in \weight(\vartheta), \mu \not = 1 } \p_{\mu}$. Then, the following commutation relations hold for all $\mu, \nu \in \weight(\vartheta)$:
                \begin{equation} \label{equation: symmetric_space_relations}
                    [\p_{\mu}, \p_{\nu}] \subseteq \p_{\mu \nu}
                \end{equation}
            and consequently, we have that:
                $$[\k, \p] \subseteq \p \quad, \quad [\k, \k] \subseteq \k$$
        \end{lemma}
            \begin{proof}
                For $x \in \p_{\mu}$ and $y \in \p_{\nu}$, consider:
                    $$\vartheta( [x, y] ) = [ \vartheta(x), \vartheta(y) ] = [\mu \cdot x, \nu \cdot y] = \mu \nu \cdot [x, y]$$
                This tells us that $[x, y] \in \p_{\mu \nu}$, and so we have that $[\p_{\mu}, \p_{\nu}] \subseteq \p_{\mu \nu}$, as claimed. That $[\k, \p] \subseteq \p$ can be proven as follows:
                    $$[\k, \p] = \left[ \k, \bigoplus_{ \mu \in \weight(\vartheta), \mu \not = 1 } \p_{\mu} \right] = \sum_{ \mu \in \weight(\vartheta), \mu \not = 1 } [\k, \p_{\mu}] = \sum_{ \mu \in \weight(\vartheta), \mu \not = 1 } \p_{\mu} \subseteq \bigoplus_{ \mu \in \weight(\vartheta), \mu \not = 1 } \p_{\mu} =: \p$$
                while that $[\k, \k] \subseteq \k$ then follows from setting $\mu = \nu = 1$.
            \end{proof}
        \begin{remark}
            It is possible that for $\mu, \nu \in \weight(\vartheta)$, we may have $\mu \nu = 1$ even when $\mu, \nu \not = 1$. For instance, when $\vartheta$ is an involution, this occurs when $\mu = \nu = -1$. This means that in general, $\p$ is not a Lie subalgebra of $\a$, in contrast with $\k$ which is a Lie subalgebra (thanks to the relation $[\k, \k] \subseteq \k$).
        \end{remark}
        
        Next, let:
            $$(\a, \a^+, \a^-)$$
        be a Manin triple twisted - either anti-invariantly or invariantly - in the sense of definition \ref{def: twisted_manin_triples} by a Lie algebra automorphism $\vartheta \in \Aut_{\LA}(\a)$, and let us abbreviate:
            \begin{equation} \label{equation: polarised_symmetric_space_decompositions}
                \k^{\pm} := \k \cap \a^{\pm} \quad, \quad \p^{\pm} := \p \cap \a^{\pm}
            \end{equation}
        The relations \eqref{equation: symmetric_space_relations} then induces the following relations:
            \begin{equation} \label{equation: polarised_symmetric_space_relations}
                [\k^{\pm}, \p^{\pm}] \subseteq \p^{\pm} \quad, \quad [\k^{\pm}, \k^{\pm}] \subseteq \k^{\pm} 
            \end{equation}
        from which we infer in particular that while the subspaces $\k^{\pm}$ are Lie subalgebras of $\a^{\pm}$, the subspaces $\p^{\pm}$ are not.
        \begin{itemize}
            \item From the fact that $[\k^{\mp}, \p^{\mp}] \subseteq \p^{\mp}$, we see that $\p^{\mp}$ has the structure of a $\k^{\mp}$-module via the adjoint action. We regard this module structure as the linear map:
                \begin{equation} \label{equation: action_of_fixed_point_subalgebras_on_unfixed_points}
                    \rho^{\mp}: \k^{\mp} \tensor \p^{\mp} \to \p^{\mp}
                \end{equation}
            given by\footnote{For $\rho^-$, the additional minus sign is necessary for keeping track of the fact that $\a- \cong (\a^+)^{*, \op, \cop}$ as Lie bialgebras.}:
                $$\rho^{\mp}(x \tensor y) := \mp [x, y]_{\a^{\mp}} \quad, \quad x \in \k^{\mp}, y \in \p^{\mp}$$
            Dualising the map \eqref{equation: action_of_fixed_point_subalgebras_on_unfixed_points} then yields a linear map $(\rho^{\mp})^*: (\p^{\mp})^* \to (\k^{\mp} \tensor \p^{\mp})^*$. Should we have that $(\k^{\mp})^* \hattensor (\p^{\mp})^* \subseteq (\k^{\mp} \tensor \p^{\mp})^*$, then $\rho^*$ would extend to a topological $(\a^{\mp})^*$-comodule structure on $(\p^{\mp})^*$, which we (abusively) denote by the same symbol:
                $$(\rho^{\mp})^*: (\p^{\mp})^* \to (\a^{\mp})^* \hattensor (\p^{\mp})^*$$
            \item Likewise, by dualising the Lie bracket on $\k^{\mp}$, one obtains a linear map:
                \begin{equation} \label{equation: dual_brackets_on_fixed_point_subalgebras}
                    [\cdot, \cdot]_{\k^{\mp}}^*: (\k^{\mp})^* \to (\k^{\mp} \tensor \k^{\mp})^*
                \end{equation}
            which would define a topological Lie bialgebra structure on $(\k^{\mp})^*$ should we have that $(\k^{\mp})^* \hattensor (\k^{\mp})^* \subseteq (\k^{\mp} \tensor \k^{\mp})^*$.
        \end{itemize}
        To those ends, recall from definition \ref{def: manin_triples} that for Manin triples $(\a, \a^+, \a^-)$, we assume that the dual brackets $\delta^{\pm} := [\cdot, \cdot]_{\a^{\mp}}^*: (\a^{\mp})^* \to (\a^{\mp} \tensor \a^{\mp})^*$ factor through $(\a^{\mp})^* \hattensor (\a^{\mp})^*$. We see thus that equation \eqref{equation: action_of_fixed_point_subalgebras_on_unfixed_points} defines a topological $(\a^{\mp})^*$-comodule structure on $(\p^{\mp})^*$:
            \begin{equation} \label{equation: coactions_on_dual_of_unfixed_points}
                (\rho^{\mp})^*: (\p^{\mp})^* \to (\a^{\mp})^* \hattensor (\p^{\mp})^*
            \end{equation}
        while equation \eqref{equation: dual_brackets_on_fixed_point_subalgebras} yields us a topological Lie bialgebra structure on $(\k^{\mp})^*$:
            \begin{equation} \label{equation: cobrackets_on_fixed_points}
                [\cdot, \cdot]_{\k^{\mp}}^*: (\k^{\mp})^* \to (\k^{\mp})^* \hattensor (\k^{\mp})^*
            \end{equation}
        
        Our next task is to compute the dual spaces $(\p^{\mp})^*$ and $(\k^{\mp})^*$, and we shall see that this depends on whether the Manin triple being considered is twisted anti-invariantly or invariantly.
        
        \begin{lemma}[Manin triple twists and duality] \label{lemma: manin_triple_twists_and_duality}
            Let $(\a, \a^+, \a^-)$ be a Manin triple which is twisted in the sense of definition \ref{def: twisted_manin_triples} by a Lie algebra automorphism $\vartheta \in \Aut_{\LA}(\a)$, and let $\k^{\pm}, \p^{\pm}$ be as in \eqref{equation: polarised_symmetric_space_decompositions}. Then, via the isomorphisms \eqref{equation: manin_triple_currying}, we can make the following identifications of $(\p^{\mp})^*$ and $(\k^{\mp})^*$.
            \begin{enumerate}
                \item $\vartheta$ twists $(\a, \a^+, \a^-)$ anti-invariantly if and only if:
                    \begin{equation} \label{equation: anti_invariant_twist_duality}
                        (\p^{\mp})^* \cong \bigoplus_{ \mu \in \weight(\vartheta), \mu \not = -1 } \p_{\mu}^{\pm} \quad, \quad (\k^{\mp})^* \cong \p_{-1}^{\pm}
                    \end{equation}
                \item $\vartheta$ twists $(\a, \a^+, \a^-)$ invariantly if and only if:
                    \begin{equation} \label{equation: invariant_twist_duality}
                        (\p^{\mp})^* \cong \p^{\pm} \quad, \quad (\k^{\mp})^* \cong \k^{\pm}
                    \end{equation}
            \end{enumerate}
            with both cases holding regardless of whether $\vartheta$ is polar or anti-polar.
        \end{lemma}
        \begin{remark} \label{remark: manin_triple_twists_via_currying}
            Let us take a slight detour and discuss a re-interpretation of definition \ref{def: twisted_manin_triples} that will be useful for the proof of lemma \ref{lemma: manin_triple_twists_and_duality}.
        
            Recall that the data defining a Manin triple $(\a, \a^+, \a^-)$ (cf. definition \ref{def: manin_triples}) consists in particular of isomorphisms of vector spaces\footnote{Recall from definition \ref{def: manin_triples} that the dual spaces $(\a^{\mp})^*$ are equipped with the weak topology, and $\a = \a^+ \oplus \a^-$ is equipped with the product topology.}:
                $$\beta^{\pm}: \a^{\pm} \xrightarrow[]{\cong} (\a^{\mp})^*$$
            given by:
                $$\beta^{\pm}(x) := (x, \cdot)_{\a} \quad, \quad x \in \a^{\pm}$$
            (cf. equation \eqref{equation: manin_triple_currying}). Using these isomorphisms, we can rephrase definition \ref{def: twisted_manin_triples} as follows: a polar Lie algebra automorphism $\vartheta \in \Aut_{\LA}(\a)$ is a twist of a Manin triple $(\a, \a^+, \a^-)$ if and only if either of the following diagrams in the category of topological vector spaces commutes:
                \begin{equation} \label{diagram: manin_triple_polar_twist_currying}
                    \begin{tikzcd}
                    	{\a^+} & {\a^+} \\
                    	{(\a^-)^*} & {(\a^-)^*}
                    	\arrow["\vartheta|_{\a^+}", from=1-1, to=1-2]
                    	\arrow["{\beta^+}"', from=1-1, to=2-1]
                    	\arrow["{\beta^+}", from=1-2, to=2-2]
                    	\arrow["{\pm \vartheta|_{\a^-}^*}", from=2-1, to=2-2]
                    \end{tikzcd}
                \end{equation}
            with \say{$-\vartheta^*$} corresponding to the anti-invariant case while \say{$\vartheta^*$} corresponding to the invariant case.

            Twisting by an anti-polar automorphism has a completely analogous re-interpretation in terms of the following commutative diagram:
                \begin{equation} \label{diagram: manin_triple_anti_polar_twist_currying}
                    \begin{tikzcd}
                    	{\a^+} & {\a^-} \\
                    	{(\a^-)^*} & {(\a^+)^*}
                    	\arrow["\vartheta|_{\a^+}", from=1-1, to=1-2]
                    	\arrow["{\beta^+}"', from=1-1, to=2-1]
                    	\arrow["{\beta^-}", from=1-2, to=2-2]
                    	\arrow["{\pm \vartheta|_{\a^+}^*}", from=2-1, to=2-2]
                    \end{tikzcd}
                \end{equation}
        \end{remark}
            \begin{proof}[Proof of lemma \ref{lemma: manin_triple_twists_and_duality}]
                The polar and anti-polar cases are completely analogous to one another, so let us discuss only the former.

                Through the fact that $\k^{\pm} = \ker(\vartheta|_{\a^{\pm}} - 1)$ and through diagram \eqref{diagram: manin_triple_polar_twist_currying}, we see that the following diagram commutes:
                    \begin{equation} \label{diagram: manin_triple_twists_and_duality}
                        \begin{tikzcd}
                        	0 & {\k^+} & {\a^+} & {\a^+} & {\p^+} & 0 \\
                        	0 & {\beta^+(\k^+)} & {(\a^-)^*} & {(\a^-)^*} & {\beta^+(\p^+)} & 0
                        	\arrow[from=1-1, to=1-2]
                        	\arrow["\ker", from=1-2, to=1-3]
                        	\arrow["{\beta^+}"', dashed, from=1-2, to=2-2]
                        	\arrow["{\vartheta|_{\a^+} - 1}", from=1-3, to=1-4]
                        	\arrow["{\beta^+}"', from=1-3, to=2-3]
                        	\arrow["\coker", from=1-4, to=1-5]
                        	\arrow["{\beta^+}", from=1-4, to=2-4]
                        	\arrow[from=1-5, to=1-6]
                        	\arrow["{\beta^+}", dashed, from=1-5, to=2-5]
                        	\arrow[from=2-1, to=2-2]
                        	\arrow[from=2-2, to=2-3]
                        	\arrow["{\pm \vartheta|_{\a^-}^* - 1}", from=2-3, to=2-4]
                        	\arrow[from=2-4, to=2-5]
                        	\arrow[from=2-5, to=2-6]
                        \end{tikzcd}
                    \end{equation}
                Since the vertical arrows are isomorphisms (see remark \ref{remark: manin_triple_twists_via_currying} for more details), the bottom row of diagram \eqref{diagram: manin_triple_twists_and_duality} is also an exact sequence, and we see thus that:
                    $$
                        \begin{gathered}
                            \beta^+(\k^+) \cong \ker(\pm \vartheta|_{\a^-}^* - 1)
                            \\
                            \beta^+(\p^+) \cong \coker(\pm \vartheta|_{\a^-}^* - 1)
                        \end{gathered}
                    $$
                Using the setup above, let us now prove the stated claims.
                \begin{enumerate}
                    \item Assume first of all that $\vartheta$ twists anti-invariantly. In this case, the middle arrow in the bottom row in diagram \eqref{diagram: manin_triple_twists_and_duality} is:
                        $$-\vartheta|_{\a^-}^* - 1 = -( \vartheta|_{\a^-}^* + 1 )$$
                    (cf. diagram \eqref{diagram: manin_triple_polar_twist_currying}). From this, we see that:
                        $$\beta^+(\k^+) \cong \ker(-\vartheta|_{\a^-}^* - 1) = \ker(\vartheta|_{\a^-}^* + 1) = \ker(\vartheta|_{\a^-} + 1)^* = (\p_{-1}^-)^*$$
                    As $\beta^+: \a^+ \to (\a^-)^*$ is an isomorphism, we have $\beta^+(\k^+) \oplus \beta^+(\p^+) = \beta^+(\k^+ \oplus \p^+) = \beta^+(\a^+)$, and hence:
                        $$\beta^+(\p^+) = \frac{\beta^+(\a^+)}{\beta^+(\k^+)} = \beta^+\left( \bigoplus_{ \mu \in \weight(\vartheta), \mu \not = -1 } \p_{\mu}^+ \right) \cong \left( \bigoplus_{ \mu \in \weight(\vartheta), \mu \not = -1 } \p_{\mu}^- \right)^*$$

                    Conversely, assume that $\beta^+(\k^+) = (\p^-_{-1})^*$ and $\beta^+(\p^+) = \left( \bigoplus_{ \mu \in \weight(\vartheta), \mu \not = -1 } \p_{\mu}^- \right)^*$. This means that $\beta^+(\k^+) = \ker(-\vartheta|_{\a^-}^* - 1) = \ker( \vartheta|_{\a^-}^* + 1 )$, and hence the middle arrow in the bottom row in diagram \eqref{diagram: manin_triple_twists_and_duality} to be:
                        $$-\vartheta|_{\a^-}^* - 1 = -( \vartheta|_{\a^-}^* + 1 )$$
                    According to diagram \eqref{equation: manin_triple_currying}, $\vartheta$ must then twist the Manin triple $(\a, \a^+, \a^-)$ anti-invariantly.
                    \item Now, assume that $\vartheta$ twists invariantly. Then, because $(\vartheta|_{\a^+} - 1)^* = \vartheta|_{\a^+}^* - 1$, the lower row in diagram \eqref{diagram: manin_triple_twists_and_duality} coincides with the following diagram:
                        \begin{equation}
                            \begin{tikzcd}
                            	0 & {(\p^-)^*} & {(\a^-)^*} & {(\a^-)^*} & {(\p^-)^*} & 0
                            	\arrow[from=1-1, to=1-2]
                            	\arrow["\ker", from=1-2, to=1-3]
                            	\arrow["{(\vartheta|_{\a^-} - 1)^*}", from=1-3, to=1-4]
                            	\arrow["\coker", from=1-4, to=1-5]
                            	\arrow[from=1-5, to=1-6]
                            \end{tikzcd}
                        \end{equation}
                    This is the exact sequence obtained by dualising the top row in diagram \eqref{diagram: manin_triple_twists_and_duality}, i.e. by applying the functor $(-)^* := \Hom(-, \bbk)$ to it, and then appealing to the fact that this functor is exact. Putting everything together, we see that:
                        $$
                            \begin{gathered}
                                \beta^+(\k^+) \cong \ker(\vartheta|_{\a^-}^* - 1) \cong (\k^-)^*
                                \\
                                \beta^+(\p^+) \cong \coker(\vartheta|_{\a^-}^* - 1) \cong (\p^-)^*
                            \end{gathered}
                        $$
                        
                    Conversely, assume that $\beta^+(\p^+) = (\p^-)^*$ and $\beta^+(\k^+) = (\k^-)^*$. Since $(\k^-)^* = \ker(\vartheta|_{\a^+}^* - 1)^*$, this forces the middle arrow in the bottom row in diagram \eqref{diagram: manin_triple_twists_and_duality} to be:
                        $$\vartheta|_{\a^-}^* - 1$$
                    According to diagram \eqref{equation: manin_triple_currying}, $\vartheta$ must then twist the Manin triple $(\a, \a^+, \a^-)$ invariantly.
                \end{enumerate}
            \end{proof}
        \begin{remark}
            Interestingly, the duality patterns for both the polar and anti-polar cases turn out to be the same. Therefore, there are no \textit{abstract} qualitative differences between these two scenarios. However, the fixed-point subalgebras can behave quite differently between the two cases.
        \end{remark}
        
        Combining lemma \ref{lemma: manin_triple_twists_and_duality} with the observations leading up to it, we obtain the following rough classification of Lie-bialgebraic twists in the sense of definition \ref{def: twisted_topological_lie_bialgebras}.
        \begin{theorem}[Twisted Lie bialgebraic structures] \label{theorem: twisted_lie_bialgebraic_structures}
            We keep the notations from lemma \ref{lemma: manin_triple_twists_and_duality}, and let:
                $$\delta^{\pm}: \a^{\pm} \to \a^{\pm} \hattensor \a^{\pm}$$
            be the topological Lie algebra structures defined by the Manin triple $(\a, \a^+, \a^-)$ via equation \eqref{equation: lie_cobrackets_by_duality}. Also, for a Lie algebra automorphism:
                $$\vartheta \in \Aut_{\LA}(\a)$$
            let us write:
                \begin{equation} \label{equation: twisted_lie_cobrackets}
                    (\delta^{\pm})^{\vartheta} :=
                    \begin{cases}
                        \delta^{\pm} \circ \vartheta|_{\a^{\pm}} & \text{if and only if $\vartheta$ twists $(\a, \delta)$ anti-invariantly}
                        \\
                        [\cdot, \cdot]_{\k^{\mp}}^* & \text{if and only if $\vartheta$ twists $(\a, \delta)$ invariantly}
                    \end{cases}
                \end{equation}
            \begin{enumerate}
                \item $(\k^{\pm}, (\delta^{\pm})^{\vartheta})$ are topological Lie coideal subalgebras of $(\a^{\pm}, \delta^{\pm})$ if and only if $\vartheta \in \Aut_{\LA}(\a)$ is an involution that twists the topological Lie bialgebras $(\a, \delta)$ anti-invariantly.
                \item $(\k^{\pm}, (\delta^{\pm})^{\vartheta})$ are new topological Lie bialgebras if and only if $\vartheta \in \Aut_{\LA}(\a)$ twists the topological Lie bialgebras $(\a, \delta)$ invariantly. In this case, $\vartheta$ needs not be an involution.
            \end{enumerate}
        \end{theorem}
            \begin{proof}
                \begin{enumerate}
                    \item Combine lemma \ref{lemma: manin_triple_twists_and_duality} with equation \eqref{equation: coactions_on_dual_of_unfixed_points}.
                    \item Combine lemma \ref{lemma: manin_triple_twists_and_duality} with equation \eqref{equation: cobrackets_on_fixed_points}.
                \end{enumerate}
            \end{proof}
        We would like to note that $(\delta^{\pm})^{\vartheta} := [\cdot, \cdot]_{\k^{\mp}}^*$ is usually not a Lie sub-bialgebra structure of $\delta^{\pm} := [\cdot, \cdot]_{\a^{\mp}}^*$, which is a map $(\k^{\mp})^* \to (\k^{\mp})^* \hattensor (\k^{\mp})^* + (\p^{\mp})^* \hattensor (\p^{\mp})^*$ instead. In fact, the following criterion prohibits $(\k^{\pm}, \delta^{\pm})$ from being a topological Lie sub-bialgebra of $(\a^{\pm}, \delta^{\pm})$.
        \begin{lemma}[Lie sub-bialgebra criterion] \label{lemma: lie_sub_bialgebra_criterion}
            Suppose that $(\a^+, \delta^+)$ is a topological Lie bialgebra defined by a Manin triple $(\a, \a^+, \a^-)$, and that $\k^+ \subset \a^+$ is a Lie subalgebra. Then, the following statements are equivalent.
            \begin{enumerate}
                \item $(\k^+, \delta^+)$ is a topological Lie sub-bialgebra.
                \item $(\k^+)^{\perp} := \{ \phi \in (\k^+)^* \subset (\a^+)^* \mid \forall x \in \k^+: \phi(x) = ( x, \phi )_{\a} = 0 \}$ is a Lie ideal of $(\a^+)^*$.
            \end{enumerate}
        \end{lemma}
            \begin{proof}
                We will freely employ the identification $\a^- \cong (\a^+)^*$, as stipulated by definition \ref{def: manin_triples}. Let us also recall equation \ref{equation: lie_cobrackets_by_duality}, which tells us that the Lie cobracket $\delta^+$ is determined by:
                    \begin{equation} \label{equation: lie_sub_bialgebra_criterion}
                        ( \delta^+(x), \phi \tensor \psi )_{\a \hattensor \a} = ( x, [\phi, \psi]_{\a^-} )_{\a} \quad, \quad x \in \a^+, \phi, \psi \in \a^-
                    \end{equation}
                \begin{enumerate}
                    \item Suppose first of all that \url{1} holds true, i.e. that $\delta^+(\k^+) \subseteq \k^+ \hattensor \k^+$. If we now take $x \in \k^+$ and suppose that the LHS is zero, then $\phi \tensor \psi \in (\k^+ \hattensor \k^+)^{\perp}$. As such, let us consider the following:
                        \begin{equation} \label{equation: orthogonal_complement_of_tensor_square}
                            \begin{aligned}
                                (\k^+ \hattensor \k^+)^{\perp} & := \{ \Phi \in (\k^+ \hattensor \k^+)^* \mid \forall x \tensor y \in \k^+ \hattensor \k^+: \Phi(x \tensor y) = (x \tensor y, \Phi)_{\a \hattensor \a} = 0 \}
                                \\
                                & = \span \{ \phi \tensor \psi \in (\k^+ \hattensor \k^+)^* \mid \forall x \tensor y \in \k^+ \hattensor \k^+: (\phi \tensor \psi)(x \tensor y) = (x \tensor y, \phi \tensor \psi)_{\a \hattensor \a} = 0 \}
                                \\
                                & = (\k^+)^{\perp} \hattensor \a^- + \a^- \hattensor (\k^+)^{\perp}
                            \end{aligned}
                        \end{equation}
                    with the last equality holding because $(x \tensor y, \phi \tensor \psi)_{\a \hattensor \a} = (x, \phi)_{\a} (y, \psi)_{\a}$, so the term vanishes if either factor does. Now, the LHS of equation \eqref{equation: lie_sub_bialgebra_criterion} is zero means that the RHS is also zero. Thus, we must simultaneously have that $[\phi, \psi]_{\a^-} \in (\k^+)^{\perp}$ whenever $x \in \k^+$, if we are to have $(x, [\phi, \psi]_{\a^-})_{\a} = 0$. But since $\phi \tensor \psi \in (\k^+)^{\perp} \hattensor \a^- + \a^- \hattensor (\k^+)^{\perp}$, we see thus that $(\k^+)^{\perp}$ must be a Lie ideal of $\a^-$, as claimed. 
                    \item Conversely, assume that \url{2} is true. If the two sids of equation \ref{equation: lie_sub_bialgebra_criterion} vanish, then $[\phi, \psi]_{\a^-} \in (\k^+)^{\perp}$ whenever $x \in \k^+$. Since $(\k^+)^{\perp} \subset \a^-$ is a Lie ideal, we see thus that $\phi \tensor \psi \in (\k^+)^{\perp} \hattensor \a^- + \a^- \hattensor (\k^+)^{\perp}$. Equation \eqref{equation: orthogonal_complement_of_tensor_square} tells us that $(\k^+)^{\perp} \hattensor \a^- + \a^- \hattensor (\k^+)^{\perp} = (\k^+ \hattensor \k^+)^{\perp}$, so indeed, we must have $\delta^+(x) \in \k^+ \hattensor \k^+$ if the LHS of equation \eqref{equation: lie_sub_bialgebra_criterion} is to bee zero as well. Thus, $(\k^+, \delta^+)$ is a topological Lie sub-bialgebra of $(\a^+, \delta^+)$.
                \end{enumerate}
            \end{proof}
        In our setting, if $(\a, \delta)$ is twisted invariantly by some automorphism $\vartheta \in \Aut_{\LA}(\a)$, then through lemma \ref{lemma: manin_triple_twists_and_duality}, we see that $(\k^+)^{\perp} = \p^-$. However, through the relations \ref{equation: symmetric_space_relations}, which implies in particular that $[\p^-, \p^-] \not \subseteq \p^-$ (when $\vartheta$ is an involution, we even have that $[\p, \p] \subseteq \k$), we see that $\p^-$ is not a Lie ideal of $\a^-$. Therefore, $(\k^+, \delta^+)$ fails the criterion to be a topological Lie sub-bialgebra of $(\a^+, \delta^+)$ given in lemma \ref{lemma: lie_sub_bialgebra_criterion}.

        Finally, we note that the operation of twisting is compatible with the construction of classical doubles in the sense of Drinfeld. The precise statement can be phrased as a corollary to theorem \ref{theorem: twisted_lie_bialgebraic_structures} as follows.
        \begin{corollary}[Compatibility between twists and classical doubles] \label{coro: twisting_classical_doubles}
            Let us keep the notations from theorem \ref{theorem: twisted_lie_bialgebraic_structures}. Also, let:
                \begin{equation} \label{equation: twisting_classical_doubles}
                    \delta^{\vartheta} := (\delta^+)^{\delta} \oplus (-\delta^-)^{\vartheta}
                \end{equation}
            \begin{enumerate}
                \item $(\k, \delta^{\vartheta})$ is a topological Lie coideal subalgebra of $(\a, \delta)$ if and only if $\vartheta \in \Aut_{\LA}(\a)$ is an involution that twists the topological Lie bialgebra $(\a, \delta)$ anti-invariantly.
                \item $(\k, \delta^{\vartheta})$ is another topological Lie bialgebra if and only if $\vartheta \in \Aut_{\LA}(\a)$ twists the topological Lie bialgebra $(\a, \delta)$ invariantly. In this case, $\vartheta$ needs not be an involution.
                
                Furthermore, we have:
                    $$\k \cong \Dr(\k^+) \cong \Dr(\k^-)^{\op, \cop}$$
                which then implies via the correspondence \eqref{equation: manin_triple_lie_bialgebra_correspondence}, that the following is a Manin triple:
                    $$(\k, \k^+, \k^-)$$
            \end{enumerate}
        \end{corollary}
            \begin{proof}
                \begin{enumerate}
                    \item By lemma \ref{lemma: manin_triple_twists_and_duality}, what follows is true if and only if $\vartheta \in \Aut_{\LA}(\a)$ is an involution that twists $(\a, \delta)$ anti-invariantly.
                    
                    Taking the direct sum of the two maps \eqref{equation: coactions_on_dual_of_unfixed_points} yields us a topological $\a^*$-comodule structure:
                        $$\rho^*: \p^* \to \a^* \hattensor \p^*$$
                    (topologically dual to the action $\rho: \k \hattensor \p \to \p$ given by $x \tensor y \mapsto [x, y]_{\a}$), and this comodule structure is to fit into the following commutative diagram, in which the unlabelled arrows are the canonical ones:
                        $$
                            \begin{tikzcd}
                            	{\left( (\a^-)^* \hattensor (\p^-)^* \right) \oplus \left( (\a^+)^{*, \op, \cop} \hattensor (\p^+)^* \right)} && {\left( (\a^-)^* \hattensor (\p^-)^* \right) \oplus \left( (\p^+)^* \hattensor (\a^+)^* \right)} \\
                            	{(\p^-)^* \oplus (\p^+)^*} && {\left( \a^* \hattensor (\p^-)^* \right) \oplus \left( (\p^+)^* \hattensor \a^* \right)} \\
                            	{(\p^- \oplus \p^+)^*} && {\left( \a^* \hattensor (\p^- \oplus \p^+)^* \right) \oplus \left( (\p^- \oplus \p^+)^* \hattensor \a^* \right)} \\
                            	{\p^*} && {\left( \a^* \hattensor \p^* \right) \oplus \left( \p^* \hattensor \a^* \right)}
                            	\arrow["\cong", from=1-1, to=1-3]
                            	\arrow[hook, from=1-3, to=2-3]
                            	\arrow["{(\rho^-)^* \oplus (\rho^+)^*}", from=2-1, to=1-1]
                            	\arrow["\cong"', from=2-1, to=3-1]
                            	\arrow[hook, from=2-3, to=3-3]
                            	\arrow["\cong"', from=3-1, to=4-1]
                            	\arrow["\cong", from=3-3, to=4-3]
                            	\arrow["{\rho^*}", from=4-1, to=4-3]
                            \end{tikzcd}
                        $$
                    Note also, that we have made use of the fact that there is an isomorphism of topological Lie bialgebras:
                        $$\a \cong \Dr(\a^+) := \a^+ \oplus ( \a^+ )^{*, \op, \cop} \cong \a^+ \oplus (\a^-)^{\op, \cop}$$
                    (see remark \ref{remark: embeddings_into_classical_doubles} for more details). Then, by identifying $\a^* \cong \a$ as topological Lie bialgebras and $(\p^{\mp})^* \cong \k^{\pm}$ (using lemma \ref{lemma: manin_triple_twists_and_duality}) and hence $\p^* \cong \k$ (since $\k = \k^+ \oplus \k^-$ and $\p = \p^+ \oplus \p^-$), we will get a topological $\a$-comodule structure:
                        $$\rho^*: \k \to \a \hattensor \k$$
                    Lastly, because $\k$ is a Lie subalgebra of $\a$ (cf. equation \eqref{equation: symmetric_space_relations}), $\rho^*$ is actually a coideal subalgebra structure coinciding with $\delta^{\vartheta}$.
                    \item Reasoning as above, we can show that $(\k, \delta^{\vartheta})$ is a topological Lie bialgebra if and only if $\vartheta \in \Aut_{\LA}(\a)$ twists $(\a, \delta)$ invariantly.
                    
                    Then, to prove that:
                        $$\k \cong \Dr(\k^+) \cong \Dr(\k^-)^{\op, \cop}$$
                    we need merely recall that there is an isomorphism of topological Lie bialgebras:
                        $$\a \cong \Dr(\a^+) := \a^+ \oplus ( \a^+ )^{*, \op, \cop} \cong \a^+ \oplus (\a^-)^{\op, \cop}$$
                    (again, see remark \ref{remark: embeddings_into_classical_doubles} for more details), and then that $\k^{\pm}$, respectively, are Lie subalgebras of $\a^{\pm}$ (this follows from the relations \eqref{equation: symmetric_space_relations}).
                \end{enumerate}
            \end{proof}
            
        \begin{remark}[Pseudo-fixed-point topological Lie coideal subalgebras ?] \label{remark: notations_for_twisted_lie_bialgebraic_structures}
            Later on, we would also like to investigate the evaluation of the twisted structure $\delta^{\vartheta}$ on Lie subalgebras of $\a$ which, despite being defined by the automorphism $\vartheta$, are not fixed-point subalgebras. Instead, we may consider the values of $\delta^{\vartheta}$ on the so-called \say{pseudo-fixed-point subalgebras}, which have been studied in \cite{regelskis_vlaar_finite_QSPs_via_generalised_satake_diagrams} and \cite{regelskis_vlaar_kac_moody_pseudo_symmetric_pairs} as combinatorially motivated generalisations of the notion of \say{quantum symmetric pairs} (also called \say{$\i$quantum groups}) that had been previously studied by Letzter, Kolb, Stokman, Noumi, Sugitani, and many others (see e.g. \cite{letzter_coideal_subalgebras_and_QSPs} and \cite{kolb_kac_moody_QSPs}). Note also that such generalisations have only been made within the Kac-Moody setting, and to our knowledge, no Yangian or elliptic analogues of the notion of pseudo-fixed-point subalgebras have ever been studied.
        \end{remark}
        
        In conclusion, there are two possibilities when a Lie bialgebra $(\a, \delta)$ is twisted in the sense of definition \ref{def: twisted_topological_lie_bialgebras}. Namely, either the fixed-point subalgebra $\k \subseteq \a$ gives rise to a topological Lie coideal subalgebra:
            $$\delta^{\vartheta}: \k \to \a \hattensor \k + \k \hattensor \a$$
        if and only if the twist is anti-invariant and by an involution, or another Lie bialgebra structure:
            $$\delta^{\vartheta}: \k \to \k \hattensor \k$$
        if and only if the twist is invariant (by any automorphism). Our next order of business shall be to explicitly compute the expressions:
            $$\delta^{\vartheta}(x) \quad, \quad x \in \k$$
        Only coboundary topological Lie bialgebras (and in particular, the quasi-triangular ones) are of relevance to questions surrounding quantisation, so henceforth, let us focus on twists of such topological Lie bialgebras.

    \subsection{Cohomological interpretation of invariant twists}
        Now, let us shift perspective slightly. Namely, in this subsection, invariant twists are to be regarded instead as Lie bialgebra automorphisms. While this is true simply by definition, what this allows for is a possibility for interpreting invariant twists as nonabelian cocycles, as demonstrated through \cite[Proposition 2.4 and Corollary 2.5]{alsaody_pianzola_classification_of_lie_bialgebras_by_nonabelian_galois_cohomology_preprint_version}. In section \ref{section: invariant_twists_and_belavin_drinfeld_classification}, these results will help us make the case, that the procedure of invariant twisting respects the Belavin-Drinfeld classification of solutions to CYBEs.

        First, let us recall the following idea about sheaf cohomology with coefficients in objects of a potentially nonabelian category, e.g. the category $\Grp$ of groups, which goes back to \cite[Exposé XIII]{sga1}; for a comprehensive and detailed treatment of nonabelian cohomology, we refer the reader to the eponymous \cite{giraud_cohomologie_nonabelienne}. For our purposes, the following shall suffice. If $S := \Spec A$ is a base scheme, then let us write:
            $$S_{\fppf}$$
        for the fppf site of $S$, i.e. the category $\Sch_{/S}$ of schemes over $S$ equipped with the fppf Grothendieck topology. Then, for a sheaf of groups $\scrG$, let us write:
            $$H^1_{\fppf}(S, \scrG)$$
        for the first nonabelian sheaf cohomology with coefficients in $\scrG$. In particular we are interested $\scrG := \Aut_{\LBA}(\a, \delta)$ being the group of topological Lie bialgebra automorphism of a topological Lie bialgebra $(\a, \delta)$, which is \textit{a priori} is an affine formal group scheme.\todo{Motivation for the use of nonabelian cohomology.}

        \begin{definition}[Fibred categories] \label{def: fibred_categories}
            
        \end{definition}

        \begin{definition}[Descent data] \label{def: descent_data}
            
        \end{definition}

        \begin{definition}[Twisted forms of algebraic structures] \label{def: twisted_forms_of_algebraic_structures}
            
        \end{definition}

        \begin{lemma}[Descent data and twisted forms for topological Lie bialgebras] \label{lemma: descent_data_and_twisted_forms_for_topological_lie_bialgebras}
            
        \end{lemma}
            \begin{proof}
                
            \end{proof}
        \begin{proposition}[Nonabelian $H^1_{\fppf}$ and twisted forms of topological Lie bialgebras] \label{prop: nonabelian_H^1_and_twisted_forms_of_topological_lie_bialgebras}
            
        \end{proposition}
            \begin{proof}
                
            \end{proof}
        \begin{remark}
            
        \end{remark}

    \subsection{Cohomological interpretation of anti-invariant twists}