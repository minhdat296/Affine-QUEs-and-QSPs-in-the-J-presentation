\section{Twisting Lie-bialgebraic structures}
    \subsection{Twisted Manin triples and twisted topological Lie bialgebras} \label{subsection: twisted_manin_triples}
        For a recollection of basic facts about Manin triples, we refer the reader to subsection \ref{subsection: manin_triples}. We freely employ the notation scheme therein.

        Inspired by the theory of symmetric spaces, and following \cite{belliard_crampe_coideal_subalgebras_from_twisted_manin_triples}, we begin with the following definition.
        \begin{definition}[Twisted Manin triple] \label{def: twisted_manin_triples}
            Let:
                $$(\a, \a^+, \a^-)$$
            be a Manin triple (cf. definition \ref{def: manin_triples}) and consider a Lie algebra automorphism:
                $$\vartheta \in \Aut_{\LA}(\a)$$
            \begin{itemize}
                \item We say that $\vartheta$ is a \textbf{polar} (respectively, \textbf{anti-polar}) as a Manin triple automorphism if $\vartheta(\a^{\pm}) = \a^{\pm}$ (respectively, if $\vartheta(\a^{\pm}) = \a^{\mp}$).
                \item We say that $\vartheta$ \textbf{twists} the Manin triple above \textbf{(anti-)invariantly} if:
                    $$(\vartheta(x), y)_{\a} = \pm (x, \vartheta(y))_{\a}$$
                with the \say{$+$} sign corresponding to the invariant case, while the \say{$-$} sign corresponds to the anti-invariant case.
            \end{itemize}
        \end{definition}
        \begin{remark}
            Even when $\vartheta$ is a Lie algebra involution, our definition remains more general than \cite[Definition 2.1]{belliard_crampe_coideal_subalgebras_from_twisted_manin_triples}. Part of the reason for not insisting that the automorphism $\vartheta$ respects the polarisation of the Manin triple, i.e. not requiring that $\vartheta(\a^{\pm}) = \a^{\pm}$, is because we are interested also in examples of automorphisms such that $\vartheta(\a^{\pm}) = \a^{\mp}$. However, if we insist that $\vartheta$ respects the polarisation, then lemma \ref{lemma: twisted_manin_triples_and_twisted_topological_lie_bialgebras} in particular will only hold in a more restricted sense, which creates unnecessary inflexibility in our opinion.

            Interestingly, we will see through lemma \ref{lemma: manin_triple_twists_and_duality}, that an anti-invariant twist of a Manin triple $(\a, \a^+, \a^-)$ by an automorphism $\vartheta \in \Aut_{\LA}(\a)$ produces a Lie coideal subalgebra structure on the fixed-point subalgebra $\k := \a^{\vartheta}$ if and only if $\vartheta$ is a Lie algebra involution. That said, we are able to extend the results from \cite{belliard_crampe_coideal_subalgebras_from_twisted_manin_triples} to the case wherein $\vartheta$ is an \textit{anti-polar} involution that twists anti-invariantly. Note also that there are also involutions that twist \textit{invariantly}; see lemma \ref{lemma: manin_triple_twists_and_duality} for details.
        \end{remark}

        Recall that by duality, any Manin triple $(\a, \a^+, \a^-)$ gives rise to a topological Lie bialgebra structure $\delta: \a \to \a \hattensor \a$ by means of:
            $$\left( \delta(x), y \tensor z \right)_{\a \hattensor \a} = \left( x, [y, z] \right)_{\a} \quad, \quad x, y, z \in \a$$
        wherein $(\cdot, \cdot)_{\a \hattensor \a}$ is the factor-wise extension of the bilinear form $(\cdot, \cdot)_{\a}$ to $\a \hattensor \a$ (cf. equation \eqref{equation: lie_cobrackets_by_duality}). If $(\a, \a^+, \a^-)$ is twisted by a Lie algebra automorphism $\vartheta \in \Aut_{\LA}(\a)$, then:
            \begin{equation} \label{equation: twisting_lie_bialgebras_by_duality}
                \begin{aligned}
                    \pm \left( x, \vartheta([y, z]) \right)_{\a} & = \pm \left( x, [\vartheta(y), \vartheta(z)] \right)_{\a}
                    \\
                    & = \left( \vartheta(x), [y, z] \right)_{\a}
                    \\
                    & = \left( \delta( \vartheta(x), y \tensor z ) \right)_{\a \hattensor \a}
                \end{aligned}
            \end{equation}
        This prompts the following definition.
        \begin{definition}[Twisted topological Lie bialgebras] \label{def: twisted_topological_lie_bialgebras}
            Let $(\a, \delta)$ be a topological Lie bialgebra, and consider a Lie algebra automorphism\footnote{... which may not be a Lie bialgebra automorphism!}:
                $$\vartheta \in \Aut_{\LA}(\a)$$
            This is an \textbf{(anti-)invariant twist} of the Lie bialgebra structure $\delta$ if:
                $$\delta \circ \vartheta = \pm \vartheta^{\tensor 2} \circ \delta$$
            and like in definition \ref{def: twisted_manin_triples}, the \say{$+$} sign corresponding to the invariant case, while the \say{$-$} sign corresponds to the anti-invariant case.
        \end{definition}
        \begin{remark}
            An (anti-)invariant twist of a topological Lie bialgebra $(\a, \delta)$ by $\vartheta \in \Aut_{\LA}(\a)$ is the same as a topological Lie bialgebra (anti-)automorphism. However, it is more convenient to work with twists, especially in the anti-invariant case. This is because the codomain of an anti-automorphism of $\a$ is $\a^{\op}$, while the codomain of an automorphism of $\a$ is $\a$, so we are of the opinion that in order to avoid notational clutter, it is better to work with the notion of twists as in definition \ref{def: twisted_topological_lie_bialgebras}.
        \end{remark}
        For book-keeping purposes, let us record the following technical lemma.
        \begin{lemma}[Twisted Manin triples and twisted topological Lie bialgebras] \label{lemma: twisted_manin_triples_and_twisted_topological_lie_bialgebras}
            Let $(\a, \a^+, \a^-)$ be a Manin triple and let $\delta: \a \to \a \hattensor \a$ be the Lie bialgebra structure that it defines. Let $\vartheta \in \Aut_{\LA}(\a)$ be a Lie algebra automorphism. Then, $\vartheta$ twists the Manin triple $(\a, \a^+, \a^-)$ invariantly (respectively, anti-invariantly) if and only if it twists $\delta$ invariantly (respectively, anti-invariantly).
        \end{lemma}
            \begin{proof}
                This is clear by consideration of equation \eqref{equation: twisting_lie_bialgebras_by_duality} along with the correspondence \eqref{equation: manin_triple_lie_bialgebra_correspondence} between Manin triples and topological Lie bialgebras.
            \end{proof}

        The central question that we would like to address in this subsection is the following. An answer will be given by theorem \ref{theorem: twisted_lie_bialgebraic_structures}.
        \begin{question} \label{question: twisted_topological_lie_bialgebras}
            If $\vartheta \in \Aut_{\LA}(\a)$ is a twist of a general topological Lie bialgebra $(\a, \delta)$, then what sort of Lie-bialgebra structure:
                $$\delta^{\vartheta}$$
            will it induce onto the fixed-point subalgebra $\k := \a^{\vartheta}$ with ?
        \end{question}
        
        To provide an answer to question \ref{question: twisted_topological_lie_bialgebras}, we begin with an easy but useful lemma about the eigenspaces of a Lie algebra automorphism.
        \begin{lemma}[Symmetric space decompositions] \label{lemma: symmetric_space_decompositions}
            Let $\a$ be a Lie algebra being acted on by an automorphism $\vartheta \in \Aut_{\LA}(\a)$ and let:
                $$\a := \k \oplus \bigoplus_{ \mu \in \weight(\vartheta), \mu \not = 1 } \p_{\mu}$$
            be the eigenspace decomposition\footnote{See convention \ref{conv: endomorphisms} for an explanation of the notations.} induced by $\vartheta$; also, let $\p := \bigoplus_{ \mu \in \weight(\vartheta), \mu \not = 1 } \p_{\mu}$. Then, the following commutation relations hold for all $\mu, \nu \in \weight(\vartheta)$:
                \begin{equation} \label{equation: symmetric_space_relations}
                    [\p_{\mu}, \p_{\nu}] \subseteq \p_{\mu \nu}
                \end{equation}
            and consequently, we have that:
                $$[\k, \p] \subseteq \p \quad, \quad [\k, \k] \subseteq \k$$
        \end{lemma}
            \begin{proof}
                For $x \in \p_{\mu}$ and $y \in \p_{\nu}$, consider:
                    $$\vartheta( [x, y] ) = [ \vartheta(x), \vartheta(y) ] = [\mu \cdot x, \nu \cdot y] = \mu \nu \cdot [x, y]$$
                This tells us that $[x, y] \in \p_{\mu \nu}$, and so we have that $[\p_{\mu}, \p_{\nu}] \subseteq \p_{\mu \nu}$, as claimed. That $[\k, \p] \subseteq \p$ can be proven as follows:
                    $$[\k, \p] = \left[ \k, \bigoplus_{ \mu \in \weight(\vartheta), \mu \not = 1 } \p_{\mu} \right] = \sum_{ \mu \in \weight(\vartheta), \mu \not = 1 } [\k, \p_{\mu}] = \sum_{ \mu \in \weight(\vartheta), \mu \not = 1 } \p_{\mu} \subseteq \bigoplus_{ \mu \in \weight(\vartheta), \mu \not = 1 } \p_{\mu} =: \p$$
                while that $[\k, \k] \subseteq \k$ then follows from setting $\mu = \nu = 1$.
            \end{proof}
        \begin{remark}
            It is possible that for $\mu, \nu \in \weight(\vartheta)$, we may have $\mu \nu = 1$ even when $\mu, \nu \not = 1$. For instance, when $\vartheta$ is an involution, this occurs when $\mu = \nu = -1$. This means that in general, $\p$ is not a Lie subalgebra of $\a$, in contrast with $\k$ which is a Lie subalgebra (thanks to the relation $[\k, \k] \subseteq \k$).
        \end{remark}
        
        Next, let:
            $$(\a, \a^+, \a^-)$$
        be a Manin triple twisted - either anti-invariantly or invariantly - in the sense of definition \ref{def: twisted_manin_triples} by a Lie algebra automorphism $\vartheta \in \Aut_{\LA}(\a)$, and let us abbreviate:
            \begin{equation} \label{equation: polarised_symmetric_space_decompositions}
                \k^{\pm} := \k \cap \a^{\pm} \quad, \quad \p^{\pm} := \p \cap \a^{\pm}
            \end{equation}
        The relations \eqref{equation: symmetric_space_relations} then induces the following relations:
            \begin{equation} \label{equation: polarised_symmetric_space_relations}
                [\k^{\pm}, \p^{\pm}] \subseteq \p^{\pm} \quad, \quad [\k^{\pm}, \k^{\pm}] \subseteq \k^{\pm} 
            \end{equation}
        from which we infer in particular that while the subspaces $\k^{\pm}$ are Lie subalgebras of $\a^{\pm}$, the subspaces $\p^{\pm}$ are not.
        \begin{itemize}
            \item From the fact that $[\k^{\mp}, \p^{\mp}] \subseteq \p^{\mp}$, we see that $\p^{\mp}$ has the structure of a $\k^{\mp}$-module via the adjoint action. We regard this module structure as the linear map:
                \begin{equation} \label{equation: action_of_fixed_point_subalgebras_on_unfixed_points}
                    \rho^{\mp}: \k^{\mp} \tensor \p^{\mp} \to \p^{\mp}
                \end{equation}
            given by:
                $$\rho^{\mp}(x \tensor y) := [x, y]_{\a^{\mp}} \quad, \quad x \in \k^{\mp}, y \in \p^{\mp}$$
            Dualising the map \eqref{equation: action_of_fixed_point_subalgebras_on_unfixed_points} then yields a linear map:
                $$(\rho^{\mp})^*: (\p^{\mp})^* \to (\k^{\mp} \tensor \p^{\mp})^*$$
            and should we have that $(\k^{\mp})^* \hattensor (\p^{\mp})^* \subseteq (\k^{\mp} \tensor \p^{\mp})^*$, then $\rho^*$ would be a topological $(\k^{\mp})^*$-comodule structure on $(\p^{\mp})^*$. 
            \item Likewise, by dualising the Lie bracket on $\k^{\mp}$, one obtains a linear map:
                \begin{equation} \label{equation: dual_brackets_on_fixed_point_subalgebras}
                    [\cdot, \cdot]_{\k^{\mp}}^*: (\k^{\mp})^* \to (\k^{\mp} \tensor \k^{\mp})^*
                \end{equation}
            which would define a topological Lie bialgebra structure on $(\k^{\mp})^*$ should we have that $(\k^{\mp})^* \hattensor (\k^{\mp})^* \subseteq (\k^{\mp} \tensor \k^{\mp})^*$.
        \end{itemize}
        To those ends, recall from definition \ref{def: manin_triples} that for Manin triples $(\a, \a^+, \a^-)$, we assume that the dual brackets $\delta^{\pm} := [\cdot, \cdot]_{\a^{\mp}}^*: (\a^{\mp})^* \to (\a^{\mp} \tensor \a^{\mp})^*$ factor through $(\a^{\mp})^* \hattensor (\a^{\mp})^*$. We see thus that equation \eqref{equation: action_of_fixed_point_subalgebras_on_unfixed_points} defines a topological $(\k^{\mp})^*$-comodule structure on $(\p^{\mp})^*$:
            \begin{equation} \label{equation: coactions_on_unfixed_points}
                (\rho^{\mp})^*: (\p^{\mp})^* \to (\k^{\mp})^* \hattensor (\p^{\mp})^*
            \end{equation}
        while equation \eqref{equation: dual_brackets_on_fixed_point_subalgebras} yields us a topological Lie bialgebra structure on $(\k^{\mp})^*$:
            \begin{equation} \label{equation: cobrackets_on_fixed_points}
                [\cdot, \cdot]_{\k^{\mp}}^*: (\k^{\mp})^* \to (\k^{\mp})^* \hattensor (\k^{\mp})^*
            \end{equation}
        \begin{remark}
            \todo[inline]{$[\cdot, \cdot]_{\k^{\mp}}^*$ is usually not a Lie sub-bialgebra structure of $[\cdot, \cdot]_{\a^{\mp}}^*$.}
        \end{remark}
        
        Our next task is to compute the dual spaces $(\p^{\mp})^*$ and $(\k^{\mp})^*$, and we shall see that this depends on whether the Manin triple being considered is twisted anti-invariantly or invariantly.
        \begin{remark} \label{remark: manin_triple_twists_via_currying}
            Let us take a slight detour and discuss a re-interpretation of definition \ref{def: twisted_manin_triples} that will be useful for the proof of lemma \ref{lemma: manin_triple_twists_and_duality}.
        
            Recall that the data defining a Manin triple $(\a, \a^+, \a^-)$ (cf. definition \ref{def: manin_triples}) consists in particular of isomorphisms of vector spaces\footnote{Recall from definition \ref{def: manin_triples} that the dual spaces $(\a^{\mp})^*$ are equipped with the weak topology, and $\a = \a^+ \oplus \a^-$ is equipped with the product topology.}:
                $$\beta^{\pm}: \a^{\pm} \xrightarrow[]{\cong} (\a^{\mp})^*$$
            given by:
                $$\beta^{\pm}(x) := (x, \cdot)_{\a} \quad, \quad x \in \a^{\pm}$$
            (cf. equation \eqref{equation: manin_triple_currying}). Using these isomorphisms, we can rephrase definition \ref{def: twisted_manin_triples} as follows: a polar Lie algebra automorphism $\vartheta \in \Aut_{\LA}(\a)$ is a twist of a Manin triple $(\a, \a^+, \a^-)$ if and only if either of the following diagrams in the category of topological vector spaces commutes:
                \begin{equation} \label{diagram: manin_triple_twist_currying}
                    \begin{tikzcd}
                    	{\a^+} & {\a^+} \\
                    	{(\a^-)^*} & {(\a^-)^*}
                    	\arrow["\vartheta|_{\a^+}", from=1-1, to=1-2]
                    	\arrow["{\beta^+}"', from=1-1, to=2-1]
                    	\arrow["{\beta^+}", from=1-2, to=2-2]
                    	\arrow["{\pm \vartheta|_{\a^-}^*}", from=2-1, to=2-2]
                    \end{tikzcd}
                \end{equation}
            with \say{$-\vartheta^*$} corresponding to the anti-invariant case while \say{$\vartheta^*$} corresponding to the invariant case.

            Twisting by an anti-polar automorphism has a completely analogous re-interpretation.
        \end{remark}
        
        \begin{lemma}[Manin triple twists and duality] \label{lemma: manin_triple_twists_and_duality}
            Let $(\a, \a^+, \a^-)$ be a Manin triple which is twisted in the sense of definition \ref{def: twisted_manin_triples} by a Lie algebra automorphism $\vartheta \in \Aut_{\LA}(\a)$, and let $\k^{\pm}, \p^{\pm}$ be as in \eqref{equation: polarised_symmetric_space_decompositions}. Then, via the isomorphisms \eqref{equation: manin_triple_currying}, we can make the following identifications of $(\p^{\mp})^*$ and $(\k^{\mp})^*$.
            \begin{enumerate}
                \item $\vartheta$ twists $(\a, \a^+, \a^-)$ anti-invariantly if and only if:
                    \begin{equation} \label{equation: anti_invariant_twist_duality}
                        \begin{cases}
                            (\p^{\mp})^* \cong \left( \bigoplus_{ \mu \in \weight(\vartheta), \mu \not = -1 } \p_{\mu}^{\pm} \right)^* \quad, \quad (\k^{\mp})^* \cong \p_{-1}^{\pm} & \text{when $\vartheta$ is polar}
                            \\
                            (\p^{\mp})^* \cong \left( \bigoplus_{ \mu \in \weight(\vartheta), \mu \not = -1 } \p_{\mu}^{\mp} \right)^* \quad, \quad (\k^{\mp})^* \cong \p_{-1}^{\mp} & \text{when $\vartheta$ is anti-polar}
                        \end{cases}
                    \end{equation}
                \item $\vartheta$ twists $(\a, \a^+, \a^-)$ invariantly if and only if:
                    \begin{equation} \label{equation: invariant_twist_duality}
                        \begin{cases}
                            (\p^{\mp})^* \cong \p^{\pm} \quad, \quad (\k^{\mp})^* \cong \k^{\pm} & \text{when $\vartheta$ is polar}
                            \\
                            (\p^{\mp})^* \cong \p^{\mp} \quad, \quad (\k^{\mp})^* \cong \k^{\mp} & \text{when $\vartheta$ is anti-polar}
                        \end{cases}
                    \end{equation}
                In this case, $\vartheta$ needs not be an involution.
            \end{enumerate}
        \end{lemma}
            \begin{proof}
                The polar and anti-polar cases are completely analogous to one another, so let us discuss only the former.

                Through the fact that $\k^{\pm} = \ker(\vartheta|_{\a^{\pm}} - 1)$ and through diagram \eqref{diagram: manin_triple_twist_currying}, we see that the following diagram commutes:
                    \begin{equation} \label{diagram: manin_triple_twists_and_duality}
                        \begin{tikzcd}
                        	0 & {\k^+} & {\a^+} & {\a^+} & {\p^+} & 0 \\
                        	0 & {\beta^+(\k^+)} & {(\a^-)^*} & {(\a^-)^*} & {\beta^+(\p^+)} & 0
                        	\arrow[from=1-1, to=1-2]
                        	\arrow["\ker", from=1-2, to=1-3]
                        	\arrow["{\beta^+}"', dashed, from=1-2, to=2-2]
                        	\arrow["{\vartheta|_{\a^+} - 1}", from=1-3, to=1-4]
                        	\arrow["{\beta^+}"', from=1-3, to=2-3]
                        	\arrow["\coker", from=1-4, to=1-5]
                        	\arrow["{\beta^+}", from=1-4, to=2-4]
                        	\arrow[from=1-5, to=1-6]
                        	\arrow["{\beta^+}", dashed, from=1-5, to=2-5]
                        	\arrow[from=2-1, to=2-2]
                        	\arrow[from=2-2, to=2-3]
                        	\arrow["{\pm \vartheta|_{\a^-}^* - 1}", from=2-3, to=2-4]
                        	\arrow[from=2-4, to=2-5]
                        	\arrow[from=2-5, to=2-6]
                        \end{tikzcd}
                    \end{equation}
                Since the vertical arrows are isomorphisms (see remark \ref{remark: manin_triple_twists_via_currying} for more details), the bottom row of diagram \eqref{diagram: manin_triple_twists_and_duality} is also an exact sequence, and we see thus that:
                    $$
                        \begin{gathered}
                            \beta^+(\k^+) \cong \ker(\pm \vartheta|_{\a^-}^* - 1)
                            \\
                            \beta^+(\p^+) \cong \coker(\pm \vartheta|_{\a^-}^* - 1)
                        \end{gathered}
                    $$
                Using the setup above, let us now prove the stated claims.
                \begin{enumerate}
                    \item Assume first of all that $\vartheta$ twists anti-invariantly. In this case, the middle arrow in the bottom row in diagram \eqref{diagram: manin_triple_twists_and_duality} is:
                        $$-\vartheta|_{\a^-}^* - 1 = -( \vartheta|_{\a^-}^* + 1 )$$
                    (cf. diagram \eqref{diagram: manin_triple_twist_currying}). From this, we see that:
                        $$\beta^+(\k^+) \cong \ker(-\vartheta|_{\a^-}^* - 1) = \ker(\vartheta|_{\a^-}^* + 1) = \ker(\vartheta|_{\a^-} + 1)^* = (\p_{-1}^-)^*$$
                    As $\beta^+: \a^+ \to (\a^-)^*$ is an isomorphism, we have $\beta^+(\k^+) \oplus \beta^+(\p^+) = \beta^+(\k^+ \oplus \p^+) = \beta^+(\a^+)$, and hence:
                        $$\beta^+(\p^+) = \frac{\beta^+(\a^+)}{\beta^+(\k^+)} = \beta^+\left( \bigoplus_{ \mu \in \weight(\vartheta), \mu \not = -1 } \p_{\mu}^+ \right) \cong \left( \bigoplus_{ \mu \in \weight(\vartheta), \mu \not = -1 } \p_{\mu}^- \right)^*$$

                    Conversely, assume that $\beta^+(\k^+) = (\p^-_{-1})^*$ and $\beta^+(\p^+) = \left( \bigoplus_{ \mu \in \weight(\vartheta), \mu \not = -1 } \p_{\mu}^- \right)^*$. This means that $\beta^+(\k^+) = \ker(-\vartheta|_{\a^-}^* - 1) = \ker( \vartheta|_{\a^-}^* + 1 )$, and hence the middle arrow in the bottom row in diagram \eqref{diagram: manin_triple_twists_and_duality} to be:
                        $$-\vartheta|_{\a^-}^* - 1 = -( \vartheta|_{\a^-}^* + 1 )$$
                    According to diagram \eqref{equation: manin_triple_currying}, $\vartheta$ must then twist the Manin triple $(\a, \a^+, \a^-)$ anti-invariantly.
                    \item Now, assume that $\vartheta$ twists invariantly. Then, because $(\vartheta|_{\a^+} - 1)^* = \vartheta|_{\a^+}^* - 1$, the lower row in diagram \eqref{diagram: manin_triple_twists_and_duality} coincides with the following diagram:
                        \begin{equation}
                            \begin{tikzcd}
                            	0 & {(\p^-)^*} & {(\a^-)^*} & {(\a^-)^*} & {(\p^-)^*} & 0
                            	\arrow[from=1-1, to=1-2]
                            	\arrow["\ker", from=1-2, to=1-3]
                            	\arrow["{(\vartheta|_{\a^-} - 1)^*}", from=1-3, to=1-4]
                            	\arrow["\coker", from=1-4, to=1-5]
                            	\arrow[from=1-5, to=1-6]
                            \end{tikzcd}
                        \end{equation}
                    This is the exact sequence obtained by dualising the top row in diagram \eqref{diagram: manin_triple_twists_and_duality}, i.e. by applying the functor $(-)^* := \Hom(-, \bbC)$ to it, and then appealing to the fact that this functor is exact. Putting everything together, we see that:
                        $$
                            \begin{gathered}
                                \beta^+(\k^+) \cong \ker(\vartheta|_{\a^-}^* - 1) \cong (\k^-)^*
                                \\
                                \beta^+(\p^+) \cong \coker(\vartheta|_{\a^-}^* - 1) \cong (\p^-)^*
                            \end{gathered}
                        $$
                        
                    Conversely, assume that $\beta^+(\p^+) = (\p^-)^*$ and $\beta^+(\k^+) = (\k^-)^*$. Since $(\k^-)^* = \ker(\vartheta|_{\a^+}^* - 1)^*$, this forces the middle arrow in the bottom row in diagram \eqref{diagram: manin_triple_twists_and_duality} to be:
                        $$\vartheta|_{\a^-}^* - 1$$
                    According to diagram \eqref{equation: manin_triple_currying}, $\vartheta$ must then twist the Manin triple $(\a, \a^+, \a^-)$ invariantly.
                \end{enumerate}
            \end{proof}
        Combining lemma \ref{lemma: manin_triple_twists_and_duality} with the observations that precedes it, we obtain the following classification of twists of topological Lie bialgebras in the sense of definition \ref{def: twisted_topological_lie_bialgebras}.

        \todo[inline]{I was not expecting this, but it seems that an anti-invariant twist can only be done by an involution, if we are to obtain a coideal subalgebra structure on the fixed-point subalgebra out of this twisting procedure. One can get coideal subalgebras from non-involutive automorphisms, as in \cite{schrader_integrable_systems_from_classical_reflection_equations}, but these do not originate from the notion of Manin triple/Lie bialgebra twists by Belliard-Crampé from \cite{belliard_crampe_coideal_subalgebras_from_twisted_manin_triples} (the same notion used in \cite{belliard_regelskis_J_presentation_for_twisted_yangians} and \cite{guay_regelskis_twisted_yangians_for_symmetric_pairs_of_types_BCD}). I'm not yet sure what the correct generalisation of anti-invariant twists to non-involutive automorphisms ought to be. Perhaps, for a finite-order automorphism of order $N$, one can replace the $-1$ factor by an $N^{th}$ root of unity; when $N = 2$, we regard the $-1$ factor as a $2^{nd}$ root of unity.}
        
        \begin{theorem}[Twisted Lie bialgebraic structures] \label{theorem: twisted_lie_bialgebraic_structures}
            We keep the notations from lemma \ref{lemma: manin_triple_twists_and_duality}, and let:
                $$\delta^+: \a^+ \to \a^+ \hattensor \a^+$$
            be the topological Lie algebra structure defined by the Manin triple $(\a, \a^+, \a^-)$ via equation \eqref{equation: lie_cobrackets_by_duality}. Also, let:
                \begin{equation} \label{equation: twisted_lie_cobrackets}
                    (\delta^+)^{\vartheta} :=
                    \begin{cases}
                        \delta^+ \circ \vartheta|_{\a^+} & \text{if and only if $\vartheta$ twists $(\a^+, \delta^+)$ anti-invariantly}
                        \\
                        [\cdot, \cdot]_{\k^-}^* & \text{if and only if $\vartheta$ twists $(\a^+, \delta^+)$ invariantly}
                    \end{cases}
                \end{equation}
            \begin{enumerate}
                \item $(\k^+, (\delta^+)^{\vartheta})$ is a topological Lie coideal subalgebra of $(\a^+, \delta^+)$ if and only if $\vartheta \in \Aut_{\LA}(\a)$ is an involution that twists the topological Lie bialgebra $(\a^+, \delta^+)$ anti-invariantly.
                \item $(\k^+, (\delta^+)^{\vartheta})$ is another topological Lie bialgebra if and only if $\vartheta \in \Aut_{\LA}(\a)$ twists the topological Lie bialgebra $(\a^+, \delta^+)$ invariantly.
            \end{enumerate}
        \end{theorem}
            \begin{proof}
                \begin{enumerate}
                    \item Combine lemma \ref{lemma: manin_triple_twists_and_duality} with equation \eqref{equation: coactions_on_unfixed_points}.
                    \item Combine lemma \ref{lemma: manin_triple_twists_and_duality} with equation \eqref{equation: cobrackets_on_fixed_points}.
                \end{enumerate}
            \end{proof}
            
        \begin{remark}[Pseudo-fixed-point Lie coideal subalgebras ?] \label{remark: notations_for_twisted_lie_bialgebraic_structures}
            Later on, we would also like to investigate the evaluation of the twisted structure $\delta^{\vartheta}$ on Lie subalgebras of $\a$ which, despite being defined by the automorphism $\vartheta$, are not fixed-point subalgebras. Namely, in subsection \ref{subsection: pseudo_twisted_lie_bialgebras}, we shall be considering the values of $\delta^{\vartheta}$ on the so-called \say{pseudo-fixed-point subalgebras}, which have been studied in \cite{regelskis_vlaar_finite_QSPs_via_generalised_satake_diagrams} and \cite{regelskis_vlaar_kac_moody_pseudo_symmetric_pairs} as combinatorially motivated generalisations of the notion of \say{quantum symmetric pairs} (also called \say{$\i$quantum groups}) that had been previously studied by Letzter, Kolb, Stokman, Noumi, Sugitani, and many others (see e.g. \cite{letzter_coideal_subalgebras_and_QSPs} and \cite{kolb_kac_moody_QSPs}). Note also that such generalisations have only been made within the Kac-Moody setting, and to our knowledge, no Yangian or elliptic analogues of the notion of pseudo-fixed-point subalgebras have ever been studied.
        \end{remark}
        
        In conclusion, there are two possibilities when a Lie bialgebra $(\a, \delta)$ is twisted in the sense of definition \ref{def: twisted_topological_lie_bialgebras}. Namely, either the fixed-point subalgebra $\k \subseteq \a$ gives rise to a Lie coideal subalgebra:
            $$\delta^{\vartheta}: \k \to \a \hattensor \k + \k \hattensor \a$$
        if and only if the twist is anti-invariant and by an involution, or another Lie bialgebra structure:
            $$\delta^{\vartheta}: \k \to \k \hattensor \k$$
        if and only if the twist is invariant (by any automorphism). Our next order of business shall be to explicitly compute the expressions:
            $$\delta^{\vartheta}(x) \quad, \quad x \in \k$$
        Only coboundary topological Lie bialgebras (and in particular, the quasi-triangular ones) are of relevance to questions surrounding quantisation, so henceforth, let us focus on twists of such topological Lie bialgebras.

    \subsection{Twists of coboundary topological Lie bialgebras: generalities} \label{subsection: twisted_coboundary_topological_lie_bialgebras_generalities}
        From now on, we shall only work with coboundary topological Lie bialgebras. Apart from their relevance to the subject of quantisation, this is also out of necessity, since our techniques make use of classical r-matrices in essential ways. Inspired by the approach taken in \cite{schrader_integrable_systems_from_classical_reflection_equations}, we would like to give criteria for a coboundary topological Lie bialgebra being twisted either anti-invariantly or invariantly in terms of invariance of certain $2$-tensors derived from classical r-matrices. These criteria will be stated in lemma \ref{lemma: invariance_criteria_for_coboundary_topological_lie_bialgebra_twists}.

        Let us fix once and for all a coboundary topological Lie bialgebra:
            $$(\a, \delta, \calr)$$
        defined by a Manin triple $(\a, \a^+, \a^-)$. \textit{A priori}, its Lie cobracket can be given by:
            $$\delta = [ \Box, \calr ]$$
        wherein $\Box(x) := x \tensor 1 + 1 \tensor x$ for all $x \in \a$, and $\calr \in \a^+ \hattensor \a^- \subset \a \hattensor \a$ (commonly called a \say{classical r-matrix}) is such that $\delta = d\calr$, with $d$ being the Chevalley-Eilenberg cochain differential. For more details, we refer the reader to subsection \ref{subsection: coboundary_and_(quasi)_triangular_topological_lie_bialgebras}, though we would like to remind the reader of one fact in particular, which is that classical r-matrices are \say{unitary}, meaning that:
            $$\calr_{1, 2} = -\calr_{2, 1}$$
        which implies in particular that $\calr \in \bigwedge^2 \a$.
        
        The questions that we are concerned with in this subsection are the following.
        \begin{question} \label{question: twisted_coboundary_topological_lie_bialgebras}
            Let $\vartheta \in \Aut_{\LA}(\a)$.
            \begin{enumerate}
                \item 
                \item 
            \end{enumerate}
        \end{question}
        \begin{question} \label{question: twisted_classical_r_matrices}
            When the coboundary topological Lie bialgebra $(\a, \delta, \calr)$ is \textit{quasi-triangular}, i.e. when:
                $$\calr \in \bigwedge^2 \a$$
            is a solution to the classical Yang-Baxter equation (CYBE) \eqref{equation: CYBEs}, what is the effect of twisting on the classical r-matrix $\calr$ ?
        \end{question}
        \begin{question} \label{question: explicit_expressions_for_twisted_cobrackets}
            Let $\vartheta \in \Aut_{\LA}(\a)$ twist $(\a, \delta, \calr)$ in the sense of definition \ref{def: twisted_topological_lie_bialgebras}, and let $\delta^{\vartheta}$ be as in theorem \ref{theorem: twisted_lie_bialgebraic_structures}. How are the expressions:
                $$\delta^{\vartheta}(x) \quad, \quad x \in \k$$
            given, explicitly ?
        \end{question}
        Partial answers to question \ref{question: twisted_coboundary_topological_lie_bialgebras}\footnote{Actually, the paper \cite{schrader_integrable_systems_from_classical_reflection_equations} addressed a slightly more general version of question \ref{question: twisted_coboundary_topological_lie_bialgebras}. There, the author considered the possibility of Lie coideal structures on the fixed-point subalgebras of general finite-order automorphisms of a (genuine) Lie bialgebra.} and question \ref{question: twisted_classical_r_matrices} had already been obtained previously, particularly in \cite{schrader_integrable_systems_from_classical_reflection_equations}. There, the author came to the conclusion that for \say{genuine} coboundary Lie bialgebras $(\a, \delta, \calr)$ that are twisted anti-invariantly by a Lie algebra involution $\vartheta \in \Aut_{\LA}(\a)$, the resulting Lie-bialgebraic structure on the fixed-point subalgebra $\k := \a^{\vartheta}$ shall be a \textit{Lie coideal subalgebra} of the original Lie bialgebra, which agrees with our theorem \ref{theorem: twisted_lie_bialgebraic_structures}. When $(\a, \delta, \calr)$ is furthermore quasi-triangular, the author also alluded to the fact, that after being anti-invariantly twisted (necessarily by an involution; see theorem \ref{theorem: twisted_lie_bialgebraic_structures}), the classical r-matrix $\calr$ ends up giving rise to solutions:
            $$\calk$$
        to a\footnote{We would like to remind the reader that, for a given classical r-matrix, there can be many associated bCYBEs.} boundary classical Yang-Baxter equation (bCYBE)\footnote{These are referred to as \say{classical reflection equations} in \cite{schrader_integrable_systems_from_classical_reflection_equations}, though we prefer the term \say{boundary classical Yang-Baxter equations} as it reminds us of the fact that solutions to bCYBEs are analogues of solutions to CYBEs in the presence of boundary conditions (e.g. closed spin chains).}; this solution shall be referred to as a \say{classical k-matrix}. We shall review and extend these results to the topological setting in subsection \ref{subsection: anti_invariantly_twisted_coboundary_topological_lie_bialgebras}, and we shall make explicit certain ideas alluded to in \cite{schrader_integrable_systems_from_classical_reflection_equations} that we believe can be elaborated upon, such as the connection between anti-invariant twists and solutions to bCYBEs. Then, in subsection \ref{subsection: invariantly_twisted_coboundary_topological_lie_bialgebras}, we shall investigate invariant twists. The analyses of the two cases do share a common starting point, though, which shall be described immediately below in the current subsection \ref{subsection: twisted_coboundary_topological_lie_bialgebras_generalities}.

        Additionally, it seems that explicit expressions for:
            $$\delta^{\vartheta}(x) \quad, \quad x \in \k$$
        have never yet been computed in either case (the result from \cite{schrader_integrable_systems_from_classical_reflection_equations} is merely a qualitative one), which is our reason for posing question \ref{question: explicit_expressions_for_twisted_cobrackets}. Such formulae will be of use to us in section \ref{section: examples_of_twisted_lie_bialgebras}, wherein we consider particular examples of twisted Lie bialgebras.

        \begin{lemma}[Invariance criteria for twistability] \label{lemma: invariance_criteria_for_coboundary_topological_lie_bialgebra_twists}
            
        \end{lemma}
            \begin{proof}
                
            \end{proof}

    \subsection{Anti-invariant twists and solutions to bCYBEs} \label{subsection: anti_invariantly_twisted_coboundary_topological_lie_bialgebras}
        \todo[inline]{Anti-invariant twists. These are Lie coideal subalgebras.}
        
        \todo[inline]{Anti-invariant twists of classical r-matrices. These are solutions to bCYBEs.}

        \todo[inline]{
            On \cite[p. 18]{schrader_integrable_systems_from_classical_reflection_equations}, it was mentioned that the quantity $C_{\vartheta}(\calr) := (\vartheta - 1)^{\tensor 2}(\calr)$ is to be thought of as the classical limit of a certain reflection equation. However, this claim has only been justified when $\a = \Loop \gl_2 := \gl_2 \tensor \bbC[t^{\pm 1}]$ is acted upon by the involution $\vartheta \in \Aut_{\LA}(\Loop \gl_2)$ given by $\vartheta( x \tensor t ) := x \tensor t^{-1}$. From these data, there arises a coideal subalgebra of $\calU_q(\Loop \gl_2)$ that can be realised as the reflection algebra:
                $$\calB_q(\calK)$$
            associated to a solution:
                $$\calK(z)$$
            of the bQYBE constructed using the quantum R-matrix of $\calU_q(\Loop \gl_2)$. It was then argued (though without much detail) that the classical limit of $\calK(z)$ ought to be related $\vartheta$, thus making the construction of $C_{\vartheta}(\calr)$ somewhat \textit{post hoc}. Therefore, my goal for this subsection is to justify the consideration of $C_{\vartheta}(\calr)$. I do think, though, that it is natural to guess that $\vartheta$ would play the role of the classical limit of $\calK(z)$, since the latter is at least supposed to be an invertible intertwiner of $\calB_q(\calK)$-modules $V$ in the sense that there exist $\calB_q(\calK)$-module isomorphisms:
                $$\calK(z)_V: V(z) \xrightarrow[]{\cong} V(z^{-1})$$
            When $V$ is a classical representation, this ought to descend to an automorphism of $V$, and $\vartheta$ therefore is to be thought of as a "universal classical k-matrix".
        }

    \subsection{Invariant twists and solutions to CYBEs} \label{subsection: invariantly_twisted_coboundary_topological_lie_bialgebras}
        \todo[inline]{Invariant twists}

        \todo[inline]{Invariant twists of classical r-matrices}

    \subsection{Twisting graded Lie bialgebras by graded automorphisms}
        \todo[inline]{Explain how things simplify when a graded Lie bialgebra (with finite-dimensional components) is twisted by a graded automorphism. In particular, we can remove the extra assumption that the automorphism must have eigenvalues in this setting.}