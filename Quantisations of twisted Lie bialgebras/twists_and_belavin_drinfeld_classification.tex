\section{Invariant twists and the Belavin-Drinfeld classification} \label{section: invariant_twists_and_belavin_drinfeld_classification}
    In this section, we investigate whether our twisting procedure respects the Belavin-Drinfeld classification of Lie bialgebra structures on Kac-Moody algebras of finite and affine types.
    
    Throughout the section, let us work over $\bbk$ that is an algebraically closed field of characteristic $0$. Let $\g$ be a Kac-Moody algebra of an affine type in Kac's classification from \cite[Chapter 4]{kac_infinite_dimensional_lie_algebras}. Such a Lie algebra has an \say{underlying} finite-type Kac-Moody algebra, which we shall denote by $\bar{\g}$; together with a finite-order Lie algebra automorphism:
        $$\sigma \in \Aut_{\LA}( \Loop \bar{\g} )$$
    of the loop algebra $\Loop \bar{\g} := \bar{\g} \tensor \bbk[t^{\pm 1}]$, the finite-type Kac-Moody algebra $\bar{\g}$ determines the affine Kac-Moody algebra $\g$ up to isomorphisms (see \cite[Chapters 7 and 8]{kac_infinite_dimensional_lie_algebras} for more details). 
    
    The question that we would like to answer is the following.
    \begin{question} \label{question: invariant_twists_and_belavin_drinfeld_classification}
        Is it true, that an elliptic (respectively, trigonometric or rational) topological Lie bialgebra structure on $\Loop \bar{\g}$ will remain elliptic (respectively, trigonometric or rational) after being \textit{invariantly} twisted ?
    \end{question}
    
    \begin{remark}
        As an anti-invariant twist of a topological Lie-bialgebra is not another topological Lie bialgebra but instead a topological Lie coideal subalgebra of the original topological Lie bialgebra structure (see theorem \ref{theorem: twisted_lie_bialgebraic_structures}), and hence does not give rise to any solution to the CYBE \eqref{equation: CYBEs}, it does not make sense to ask whether or not anti-invariant twists may respect the Belavin-Drinfeld classification.
    \end{remark}

    \subsection{The Belavin-Drinfeld Classification}
        First of all, the analogue of question \ref{question: invariant_twists_and_belavin_drinfeld_classification} for Lie bialgebra structures $\delta: \bar{\g} \to \bar{\g} \tensor \bar{\g}$ is trivial. Let us quickly explain why, and afterwards, we will focus entirely on the affine case.
        \begin{remark}
            Such structures are certain Lie $1$-cocycles $\delta \in Z^1_{\Lie}(\bar{\g}, \bar{\g} \tensor \bar{\g})$. However, by Whitehead's Lemma, we know that:
                $$H^i_{\Lie}(\bar{\g}, V) = 0 \quad, \quad i > 0$$
            for every finite-dimensional module $V$ (such as the coadjoint module $\bar{\g} \tensor \bar{\g}$) over $\bar{\g}$, as a result of this Lie algebra being finite-dimensional and simple. Therefore, any Lie bialgebra structure $\delta: \bar{\g} \to \bar{\g} \tensor \bar{\g}$ is coboundary (in the sense that $\delta \in B^1_{\Lie}(\bar{\g}, \bar{\g} \tensor \bar{\g})$) and is thus is cohomologically equivalent to $\delta + d_{\CE} r$ for any $r \in Z^0_{\Lie}(\bar{\g}, \bar{\g} \tensor \bar{\g}) \cong \Hom(\bbk, \bar{\g} \tensor \bar{\g}) \cong \bar{\g} \tensor \bar{\g}$.
        \end{remark}
    
        \todo[inline]{For the affine case, see \cite{abedin_burban_geometrisation_of_the_belavin_drinfeld_classification_1} and \cite{abedin_geometrisation_of_the_belavin_drinfeld_classification_2}. See also \cite[Theorem 3.2.4]{chari_pressley_quantum_groups} and the contents that follow, and \cite[Chapters 5 and 7]{etingof_schiffmann_lectures_on_quantum_groups}.}

    \subsection{Compatibility}