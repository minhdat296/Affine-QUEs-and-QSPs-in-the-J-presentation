\section{Invariant twists and the Belavin-Drinfeld classification}
    In this section, we investigate whether our twisting procedure respects the Belavin-Drinfeld classification. In particular, we would like to make the case, that an elliptic (respectively, trigonometric or rational) Lie bialgebra structure will remain elliptic (respectively, trigonometric or rational) after being \textit{invariantly} twisted. Additionally, as an anti-invariant twist of a Lie-bialgebra is not another Lie bialgebra but instead a Lie coideal subalgebra of the original Lie bialgebra structure (see theorem \ref{theorem: twisted_lie_bialgebraic_structures}), and hence does not give rise to any solution to the CYBE \eqref{equation: CYBEs}, it does not make sense to ask whether or not anti-invariant twists may respect the Belavin-Drinfeld classification.

    \subsection{The Belavin-Drinfeld Trichotomy}
        \todo[inline]{See \cite{abedin_burban_geometrisation_of_the_belavin_drinfeld_classification_1} and \cite{abedin_geometrisation_of_the_belavin_drinfeld_classification_2}. See also \cite[Theorem 3.2.4]{chari_pressley_quantum_groups} and the contents that follow, and \cite[Chapters 5 and 7]{etingof_schiffmann_lectures_on_quantum_groups}.}

    \subsection{\textit{Intermission}: Twists of classical r-matrices with spectral parameters} \label{subsection: twists_with_spectral_parameter}
        \todo[inline]{Explain the necessary modifications for accommodating spectral parameters.}

    \subsection{Compatibility}
        \todo[inline]{In the terminologies of \cite{abedin_geometrisation_of_the_belavin_drinfeld_classification_2}, invariant twists are special cases of gauge equivalences. Therefore, one respects the Belavin-Drinfeld classification when twisting invariantly.}