\section{Introduction}
    \subsection{Notations}
        \begin{convention}[Endomorphisms] \label{conv: endomorphisms}
            \begin{itemize}
                \item Given a vector space $V$ and a linear endomorphism $T \in \End(V)$, the set of eigenvalues of $T$ shall be denoted by:
                    $$\weight(T)$$
                and the generalised eigenspace associated to an eigenvalue $\mu \in \weight(T)$ shall be denoted by:
                    $$V_{T[\mu]}$$
                or simply:
                    $$V_{\mu}$$
                when it is understood that this is supposed to be an eigenspace of $T$.
                \item Given a Lie algebra $\a$, we consider only automorphism $\vartheta \in \Aut_{\LA}(\a)$ which induce an eigenspace decomposition of $\a$, which shall be denoted by:
                    $$\a = \k \oplus \bigoplus_{ \substack{\mu \in \weight(\vartheta)\\\mu \not = 1} } \p_{\mu}, \quad, \quad \k := \a_1$$
                (and note also that $\k$ coincides with the fixed-point subalgebra $\a^{\vartheta}$), so that we would not confuse the $\vartheta$-eigenspaces with the root spaces of $\a$. While this is not true for all automorphisms, all practical examples do in fact fall into this class.
                \item Given an algebra $A$ and an implicitly given representation $A \to \End(V)$, the operator on $V$ corresponding to an element $x \in A$ will be denoted by:
                    $$x_V$$
            \end{itemize}
        \end{convention}

        \begin{convention}[Tensor notations] \label{conv: tensor_notations}
            \begin{itemize}
                \item Given vector spaces $\{V_i\}_{1 \leq i \leq n}$ and an operator with spectral parameters:
                    $$T(z_1, ..., z_n) \in \End(V_1 \tensor ... \tensor V_n)(\!(z_1^{-1}, ..., z_n^{-1})\!)$$
                along with an auxiliary space $V_0$, then for:
                    $$\{i_0\} := \{0, ..., n\} \setminus \{i_1, ..., i_n\}$$
                we will write:
                    \begin{equation} \label{equation: tensor_factor_notation}
                        T(z_1, ..., z_n)_{i_1, ..., i_n} \in \End\left(V_0 \tensor (V_1 \tensor ... \tensor V_n)\right)(\!(z_1^{-1}, ..., z_n^{-1})\!)
                    \end{equation}
                for the operator with spectral parameter on $\bigotimes_{i = 0}^n V_i$ which acts as $\id_{V_0}$ on the $i_0^{th}$ tensor factor and as $T(z_1, ..., z_n)$ on the remaining ones.
                \item If $T \in \End(V \tensor W)$, and if there is a direct sum decomposition $V \tensor W = \bigoplus_{i_1 + i_2 = i} V_{i_1} \tensor W_{i_2}$, then the composition of $T$ with the canonical quotient map $V \tensor W \to V_{i_1} \tensor W_{i_2}$ shall be denoted by:
                    $$T_{V_{i_1}, W_{i_2}}$$
            \end{itemize}
        \end{convention}

        \begin{convention}[(Co-)opposite (co)algebras]
            \begin{itemize}
                \item If $A$ is an algebra (not even necessarily associative or unital) with multiplication:
                    $$\mu: A \tensor A \to A$$
                then its \textbf{opposite algebra} shall be denoted by:
                    $$A^{\op}$$
                This object has the same underlying vector space, but the opposite multiplication is now given by:
                    $$\mu^{\op} := \mu \circ (-)_{2, 1}$$
                wherein $(-)_{2, 1}: A \tensor A \xrightarrow[]{\cong} A \tensor A$ is the isomorphism given by $(x \tensor y)_{2, 1} := y \tensor x$.
                \item Likewise, if $C$ is a coalgebra with comultiplication:
                    $$\Delta: C \to C \tensor C$$
                then its \textbf{(co-)opposite} shall be denoted by:
                    $$C^{\cop}$$
                The co-opposite multiplication is given by:
                    $$\Delta^{\cop} := (-)_{2, 1} \circ \Delta$$
                These notions generalise in a straightforward manner to the setting of topological (co)algebras.
            \end{itemize}
        \end{convention}

        \begin{convention}[Formal distributions]
            Given any formal distribution:
                $$a(z) := \sum_{m \in \Z} a_m z^{-m} \in V[\![z^{\pm 1}]\!]$$
            with coefficients in some vector space $V$, let us write:
                $$a(z)^- := \sum_{m > 0} a_m z^{-m} \quad, \quad a(z)^+ := \sum_{m \leq 0} a_m z^{-m}$$
            with the point being that we would then have:
                $$(\del a(z))^{\pm} = \del( a(z)^{\pm} )$$
        
            In order to avoid confusion with cobrackets, which are usually denoted by $\delta$, we use:
                $$\1(z) := \sum_{m \in \Z} z^{-m} \in \bbC[\![z^{\pm 1}]\!]$$
            to mean the formal Dirac distribution. Note that:
                $$\1(z)^- = \sum_{m > 0} z^{-m} = \frac{1}{z - 1} \quad, \quad \1(z)^+ = \sum_{m \leq 0} z^{-m} = \frac{1}{1 - z}$$
        \end{convention}

    \subsection{Context}

    \subsection{Overview}