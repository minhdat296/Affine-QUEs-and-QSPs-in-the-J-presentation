\section{Introduction}
    \subsection{Notations}
        \begin{convention}[Eigenspaces] \label{conv: eigenspaces}
            \begin{itemize}
                \item Given a vector space $V$ and a linear endomorphism $T \in \End(V)$, the set of eigenvalues of $T$ shall be denoted by:
                    $$\weight(T)$$
                and the generalised eigenspace associated to an eigenvalue $\mu \in \weight(T)$ shall be denoted by:
                    $$V_{T[\mu]}$$
                or simply:
                    $$V_{\mu}$$
                when it is understood that this is supposed to be an eigenspace of $T$.
                \item Given a Lie algebra $\a$, we consider only automorphism $\vartheta \in \Aut_{\LA}(\a)$ which induce an eigenspace decomposition of $\a$, which shall be denoted by:
                    $$\a = \k \oplus \bigoplus_{ \substack{\mu \in \weight(\vartheta)\\\mu \not = 1} } \p_{\mu}, \quad, \quad \k := \a_1$$
                (and note also that $\k$ coincides with the fixed-point subalgebra $\a^{\vartheta}$), so that we would not confuse the $\vartheta$-eigenspaces with the root spaces of $\a$. While this is not true for all automorphisms, all practical examples do in fact fall into this class.
            \end{itemize}
        \end{convention}

    \subsection{Context}

    \subsection{Overview}