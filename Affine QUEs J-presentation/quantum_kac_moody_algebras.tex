\section{Quantum Kac-Moody algebras}
    Here, we recall the notion of quantum Kac-Moody algebras from \cite{etingof_kazhdan_quantisation_6}.

    \subsection{Kac-Moody algebras}
        \input{Setup files and appendices/setup_kac_moody_algebras}

    \subsection{The standard Lie bialgebra structure and its quantisation}
        \input{Setup files and appendices/setup_standard_kac_moody_lie_bialgebras}

    \subsection{Presentation by Drinfeld currents for quantum affine Kac-Moody algebras}
        Let us now focus on the cases when $\g$ is of an affine type.
        
        Thanks to the loop realisation, we can now write the Casimir tensor of $\g$ more succinctly in the following manner:
            \begin{equation} \label{equation: affine_casimir_tensor}
                \begin{aligned}
                    \calr(z, w) & = \sum_{\alpha \in \bar{\rootsystem}} x_{\alpha} z^n \tensor x_{-\alpha} w^{-n} + ( \level \tensor \del + \del \tensor \level )
                    \\
                    & = \bar{\calr} \1(zw^{-1}) + ( \level \tensor \del + \del \tensor \level )
                \end{aligned}
            \end{equation}
        wherein $x_{\pm \alpha} \in \g_{\alpha}$ are orthonormal real root vectors, $\bar{\calr}$ is the Casimir tensor of the underlying finite-type Lie algebra $\bar{\g}$, and $\1(zw^{-1}) := \sum_{n \in \Z} z^n w^{-n}$ is the formal Dirac distribution\footnote{We use this alternate notation in order to avoid confusion with Lie cobrackets.}. Using formula \eqref{equation: affine_casimir_tensor}, we can rewrite the formulae for the Lie cobrackets from \eqref{equation: standard_kac_moody_lie_bialgebra_structure} into the following:
            \begin{equation}
                \begin{gathered}
                    \delta(\level) = \delta(\del) = 0
                    \\
                    \begin{aligned}
                        \delta( x t^m ) & = [x z^m \tensor 1 + 1 \tensor x w^m, \calr(z, w)]
                        \\
                        & = [ x z^m \tensor 1 + 1 \tensor x w^m, \bar{\calr} \1(zw^{-1}) + ( \level \tensor \del + \del \tensor \level ) ]
                        \\
                        & = \left( [x \tensor 1, \bar{\calr}] z^m + [1 \tensor x, \bar{\calr}] w^m \right) \1(zw^{-1}) + m \left( x z^m \tensor \level + \level \tensor x w^m \right)
                        \\
                        & = m \left( x z^m \tensor \level + \level \tensor x w^m \right)
                    \end{aligned}
                \end{gathered}  
            \end{equation}
        for all $x \in \bar{\g}$ and $m \in \Z$, wherein the last equality holds because $\bar{\calr}$ is central in $\calU(\bar{\g})$ \textit{a priori}; more succinctly still, we can write down the following action of the Lie cobracket $\delta: \g \to \g \hattensor \g$ on currents $x(t) := \sum_{m \in \Z} x[m] t^{-m}$ (with $x[m] \in \bar{\g}$ for all $m \in \Z$) as:
            $$\delta( x(t) ) = \del x(z) \tensor \level + \level \tensor \del x(w)$$
        This gives us a Lie bialgebra structure on the loop algebra $\Loop \bar{\g}$, now regarded as the quotient $\Loop \bar{\g} \cong \g/(\bbC \level \oplus \bbC \del)$. 