Let $\b^{\pm} := \n^{\pm} \oplus \h$ be the \say{Borel subalgebras} of the Kac-Moody algebra $\g$. It is clear that these Lie subalgebras of $\g$ are not isotropic with respect to the Kac-Moody pairing $(\cdot, \cdot)_{\g}$ given by equation \eqref{equation: kac_moody_pairing}. However, one can consider instead the larger Lie algebra $\a := \b^+ \oplus \b^-$, into which $\b^{\pm}$ embed by means of the maps $\eta^{\pm}: \b^{\pm} \hookrightarrow \a$ given by:
    \begin{equation} \label{equation: borel_lie_sub_bialgebra_embeddings}
        \eta^{\pm}(x) := x \oplus ( \pm x_{\h} ) \quad, \quad x \in \b^{\pm}
    \end{equation}
wherein $x_{\h}$ is the image of $x \in \b^{\pm}$ under the canonical quotient map $\b^{\pm} \to \b^{\pm}/\n^{\pm} \cong \h$. This larger Lie algebra shall be equipped with the non-degenerate and invariant pairing given by:
    $$(\cdot, \cdot)_{\a} := (\cdot, \cdot)_{\g} - (\cdot, \cdot)_{\h}$$
in which the Lie subalgebras $\eta^{\pm}(\b^{\pm})$ are clearly isotropic with respect to $(\cdot, \cdot)_{\a}$. As such, there is a Manin triple:
    $$(\a, \eta^+(\b^+), \eta^-(\b^-))$$
from which arises the topological Lie bialgebra structures $\delta^{\pm}: \eta^{\pm}(\b^{\pm}) \to \eta^{\pm}(\b^{\pm}) \hattensor \eta^{\pm}(\b^{\pm})$ given by:
    $$\delta^{\pm} = [\Box, \calr]$$
wherein $\calr$ is the Casimir tensor. On generators, these Lie cobrackets are given by:
    \begin{equation} \label{equation: standard_kac_moody_lie_bialgebra_structure}
        \begin{gathered}
            \delta^{\pm}(h) = 0 \quad, \quad h \in \h
            \\
            \delta^{\pm}(e_i^{\pm}) = \frac12 D_{i, i} e_i^{\pm} \wedge \alpha_i^{\vee} \quad, \quad 1 \leq i \leq n
        \end{gathered}
    \end{equation}
\begin{remark}
    Note also, that by construction, we have that:
        $$\a \cong \Dr( \eta^{\pm}( \b^{\pm} ) )$$
    as Lie bialgebras.
\end{remark}
    
As a consequence of this construction, the Lie subalgebra $\h \subset \a$ is a Lie coideal on top of being a Lie ideal (this is trivial, for it is abelian), and hence the quotient:
    $$\g \cong \a/\h$$
carries a Lie bialgebra structure given by the same formulae as in \eqref{equation: standard_kac_moody_lie_bialgebra_structure}.
\begin{remark}
    In fact, it is well-known that the formulae in \eqref{equation: standard_kac_moody_lie_bialgebra_structure} lift to topological Lie bialgebra structures $\tilde{\delta}^{\pm}: \tilde{\b}^{\pm} \to \tilde{\b}^{\pm} \hattensor \tilde{\b}^{\pm}$ on the Borel subalgebras $\tilde{\b}^{\pm} := \h \oplus \tilde{\n}^{\pm}$, and hence also on the extended Kac-Moody algebra $\tilde{\g}$. This is a \textit{post hoc} construction, inspired by the construction of the Lie bialgebra structures on $\b^{\pm}$ described above, and it can also be shown, without much difficulty, that the Lie algebra $\tilde{\a} := \tilde{\b}^+ \oplus \tilde{\b}^-$ with the Lie cobracket given by $\tilde{\delta} := \tilde{\delta}^+ \oplus (-\tilde{\delta}^-)$ is isomorphic to the classical doubles $\Dr(\tilde{\b}^{\pm})$; note that the embeddings of $\tilde{\b}^{\pm}$ into this larger Lie bialgebra are given by the same formulae as in \eqref{equation: borel_lie_sub_bialgebra_embeddings}. 

    That said, the existence of such a Lie bialgebra structure on the extended Kac-Moody algebra $\tilde{\g}$ is very useful for formulating a quantisation of $\g$. See, for instance, \cite{etingof_kazhdan_quantisation_6}. 
\end{remark}