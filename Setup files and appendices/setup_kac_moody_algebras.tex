Let us recall some basic features of Kac-Moody algebras. The canonical reference is the book \cite{kac_infinite_dimensional_lie_algebras}.

Following \cite[Chapter 1]{kac_infinite_dimensional_lie_algebras}, let $C := ( C_{i, j} )_{1 \leq i, j \leq n}$ be a (generalised) Cartan matrix and choose for it a realisation:
    $$(\h, \simpleroots, \simpleroots^{\vee})$$
This consists of a vector space $\h$ of dimension $l := 2n - \rank C$ along with linearly independent subsets $\simpleroots^{\vee} := \{ \alpha_i^{\vee} \}_{1 \leq i \leq n} \subset \h$ and $\simpleroots := \{ \alpha_i \}_{1 \leq i \leq n} \subset \h^*$, whose elements are known as \say{simple coroots} and \say{simple roots}, and are such that:
    \begin{equation} \label{equation: cartan_matrix_entries}
        \alpha_i( \alpha_j^{\vee} ) = C_{i, j}
    \end{equation}
\begin{remark}
    Because the Cartan matrix $C$ is not necessarily symmetric in general, the pairing \eqref{equation: cartan_matrix_entries} between the simple coroots and simple roots - by evaluating functionals - is generally \textit{not symmetric}.
\end{remark}

We can then define a Lie algebra $\tilde{\g}$ to be the one generated by the set:
    \begin{equation} \label{equation: kac_moody_generators}
        \h \cup \{ e_i^{\pm} \}_{1 \leq i \leq n}
    \end{equation}
whose elements are subjected to the following relations:
    \begin{equation} \label{equation: extended_kac_moody_relations}
        \begin{gathered}
            [h, h'] = 0 \quad, \quad h, h' \in \h
            \\
            [h, e_i^{\pm}] = \pm \alpha_i(h) e_i^{\pm} \quad, \quad [e_i^+, e_j^-] = \delta_{i, j} \alpha_i^{\vee} \quad, \quad h \in \h, 1 \leq i \leq n
        \end{gathered}
    \end{equation}
It can be shown (see e.g. \cite[Theorem 1.2]{kac_infinite_dimensional_lie_algebras}) that $\tilde{\g}$ admits a \say{triangular decomposition}:
    \begin{equation} \label{equation: extended_kac_moody_triangular_decomposition}
        \tilde{\g} \cong \tilde{\n}^- \oplus \h \oplus \tilde{\n}^+
    \end{equation}
wherein $\tilde{\n}^{\pm}$ are the free Lie algebras generated by the sets $\{ e_i^{\pm} \}_{1 \leq i \leq n}$.
    
Next, if we let $\r \subset \tilde{\g}$ be the Lie ideal that is the sum of all ideals with zero intersection with the Lie ideal $\h \subset \tilde{\g}$, then we shall obtain the \say{Kac-Moody algebra} $\g$ associated to the previously fixed Cartan matrix as the quotient:
    $$\g := \tilde{\g}/\r$$
\textit{A priori} - and this is a somewhat non-trivial fact (see \cite[Theorem 9.11]{kac_infinite_dimensional_lie_algebras}) - the Lie algebra $\g$ is generated by the same set \eqref{equation: kac_moody_generators}\footnote{Even though it is technically an abuse of notations, we shall use the same symbols to denote the elements of \eqref{equation: kac_moody_generators} and their images under the quotient map $\tilde{\g} \to \g$.}, and in addition to the relations \eqref{equation: extended_kac_moody_relations}, the generators now satisfy also the so-called \say{Serre relations}, which take the following form in the adjoint representation of $\g$:
    \begin{equation} \label{equation: kac_moody_serre_relations}
        ( \ad( e_i^{\pm} ) )^{1 - C_{i, j}} \cdot e_j \quad, \quad 1 \leq i \not = j \leq n
    \end{equation}
or in other words, $\r$ is generated by such relations. This leads to a triangular decomposition:
    \begin{equation} \label{equation: kac_moody_triangular_decomposition}
        \g \cong \n^- \oplus \h \oplus \n^+
    \end{equation}
wherein $\n^{\pm} := \tilde{\n}^{\pm}/( \tilde{\n}^{\pm} \cap \r )$

Due to the relations $[h, e_i^{\pm}] = \pm \alpha_i(h) e_i^{\pm}$, the Lie algebra $\tilde{\g}$ has a canonical \say{root grading} by the abelian group $\rootlattice := \Z \simpleroots$ (commonly called the \say{root lattice}), taking the form of a \say{root space decomposition}:
    $$\tilde{\g} \cong \bigoplus_{\alpha \in \rootlattice} \tilde{\g}_{\alpha}$$
wherein $\tilde{\g}_{\alpha} := \{ x \in \tilde{\g} \mid \forall h \in \h: [h, x] = \alpha(h) x \}$ are the \say{root spaces}. Any lattice element $\alpha := \sum_{1 \leq i \leq n} a_i \alpha_i \in \rootlattice$ has a \say{height} $\height(\alpha) := \sum_{1 \leq i \leq n} a_i$, which allows us to define $\deg x_{\alpha} := \height(\alpha)$ for all $x_{\alpha} \in \tilde{\g}_{\alpha}$. In particular, the degrees of the generators \eqref{equation: kac_moody_generators} are $\deg e_i^{\pm} = \pm 1$ and $\deg h = 0$ (for all $h \in \h$). Through the relations \eqref{equation: kac_moody_serre_relations}, one can also see that the graded components are all finite-dimensional.

Henceforth, let us assume moreover that the Cartan matrix $C$ is symmetrisable, i.e. that there exists an invertible diagonal $n \x n$ matrix $D$ and a symmetric $n \x n$ matrix $A$ (called the symmetrisation of $C$) such that:
    \begin{equation} \label{equation: symmetrising_cartan_matrices}
        C := DA
    \end{equation}
This allows us to define a symmetric and non-degenerate bilinear form $(\cdot, \cdot)_{\h} \in \Hom( \Sym^2(\h), \bbC )$ given by:
    \begin{equation} \label{equation: kac_moody_pairing_on_cartan_subalgebras}
        ( \alpha_i^{\vee}, h )_{\h} := \delta_{i, j} D_{i, i}^{-1} \alpha_i(h) \quad, \quad h \in \h, 1 \leq i \leq n
    \end{equation}
As an aside, we note that equations \eqref{equation: kac_moody_pairing_on_cartan_subalgebras} and \eqref{equation: cartan_matrix_entries} together imply that:
    $$A_{i, j} = (\alpha_i^{\vee}, \alpha_j^{\vee})_{\h} = \delta_{i, j} D_{i, i}^{-1} \alpha_i( \alpha_j^{\vee} ) = \delta_{i, j} D_{i, i}^{-1} C_{i, j}$$
Anyhow, the bilinear form $(\cdot, \cdot)_{\h}$ constructed above induces, via an induction process\footnote{For a less \textit{ad hoc} construction of $(\cdot, \cdot)_{\g}$, we refer the reader to \cite{neher_pianzola_prelat_sepp_invariant_bilinear_forms_via_fppf_descent}.}, a symmetric, non-degenerate, and \textit{invariant} bilinear form $(\cdot, \cdot)_{\g} \in \Hom( \Sym^2(\g)^{\g}, \bbC )$ given by:
    \begin{equation} \label{equation: kac_moody_pairing}
        ( e_i^-, e_j^+ )_{\g} = \delta_{i, j} D_{i, i}^{-1} \quad, \quad 1 \leq i, j \leq n
    \end{equation}
and is uniquely determined by $(\cdot, \cdot)_{\h}$. Moreover, the bilinear form $(\cdot, \cdot)_{\g}$ extends to $\tilde{\g}$; this extension is a degenerate bilinear form whose radical is precisely $\r \subset \tilde{\g}$. Additionally, and this is a rather important fact for us, the bilinear form $(\cdot, \cdot)_{\g}$ is of total degree $0$ with respect to the grading on $\Hom( \Sym^2(\g)^{\g}, \bbC )$ by the root lattice $\rootlattice$, induced by the one on $\g$ (and this is one reason why the root spaces $\g_{\alpha} \subset \g$ being finite-dimensional is important).
\begin{convention}
    For brevity, let us refer to symmetric (and often also non-degenerate) and invariant bilinear forms as \say{invariant pairings}.
\end{convention}

\todo[inline]{Loop realisations for affine Kac-Moody algbras (untwisted and twisted).}