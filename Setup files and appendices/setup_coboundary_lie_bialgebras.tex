\todo[inline]{Coboundary Lie bialgebras}

Suppose that $(\a, \delta)$ is a topological Lie bialgebra whose cobracket is $1$-coboundary when regarded as a $1$-cocycle, which is to say that there exists some $2$-tensor:
    $$\calr \in \a \hattensor \a$$
such that:
    $$\delta = d\calr$$
wherein $d$ is the Chevalley-Eilenberg cochain differential. Such a topological Lie bialgebra is said to be \textbf{coboundary}, and $2$-tensors $\calr \in \a \hattensor \a$ such that $\delta = d\calr$ are typically called \textbf{classical r-matrices}.
\begin{remark}[Non-uniqueness of classical r-matrices] \label{remark: classical_r_matrices_non_uniqueness}
    
\end{remark}
Then, one can check - simply using the construction of the Chevalley-Eilenberg differential - that the topological Lie bialgebra structure above can be given more succinctly by the following formula:
    \begin{equation} \label{equation: coboundary_lie_cobrackets}
        \delta(x) := [\Box(x), \calr] \quad, \quad x \in \a
    \end{equation}
wherein $\Box(x) := x \tensor 1 + 1 \tensor x$ (see \cite[Section 3.2]{etingof_schiffmann_lectures_on_quantum_groups}). From this, we see that $\calr$ must also satisfy the property whereby:
    \begin{equation} \label{equation: unitary_classical_r_matrices}
        \calr_{1, 2} = -\calr_{2, 1}
    \end{equation}
which is commonly known as \textbf{unitarity}. It is then natural to pose the following question.
\begin{question}
    Which alternating $2$-tensor $\calr \in \bigwedge^2 \a$ gives rise to a topological Lie bialgebra structure $\delta: \a \to \a \hattensor \a$ by means of equation \eqref{equation: coboundary_lie_cobrackets} ?
\end{question}
Of course, one should not expect that every alternating $2$-tensor $\calr \in \bigwedge^2 \a$ would give 
\begin{proposition}[Drinfeld] \label{prop: coboundary_lie_bialgebras_and_CYBEs}
    For any $\calr \in \a \hattensor \a$, let us write:
        \begin{equation} \label{equation: classical_yang_baxter_tensor}
            \CYBE(\calr) := [\calr_{1, 2}, \calr_{1, 3}] + [\calr_{1, 2}, \calr_{2, 3}] + [\calr_{1, 3}, \calr_{2, 3}] \in \a \hattensor \a \hattensor \a
        \end{equation}
    to denote the \textbf{classical Yang-Baxter tensor}. 

    An alternating $2$-tensor:
        $$\calr \in \bigwedge^2 \a$$
    gives rise to a topological Lie bialgebra structure $\delta: \a \to \hattensor \a$ given by equation \eqref{equation: coboundary_lie_cobrackets} if and only if $\CYBE(\calr)$ is an invariant alternating element of $\a \hattensor \a \hattensor \a$, i.e.:
        $$\CYBE(\calr) \in \left( \bigwedge^3 \a \right)^{\a}$$
\end{proposition}
    \begin{proof}
        See \cite[Theorem 3.1]{etingof_schiffmann_lectures_on_quantum_groups}.
    \end{proof}