\todo[inline]{Coboundary Lie bialgebras}

\begin{definition}[Coboundary topological Lie bialgebras] \label{def: coboundary_topological_lie_bialgebras}
    
\end{definition}
\begin{remark}[Non-uniqueness of classical r-matrices] \label{remark: classical_r_matrices_non_uniqueness}
    
\end{remark}
Then, one can check - simply using the construction of the Chevalley-Eilenberg differential - that the topological Lie bialgebra structure above can be given more succinctly by the following formula:
    \begin{equation} \label{equation: coboundary_lie_cobrackets}
        \delta(x) := [\Box(x), \calr] \quad, \quad x \in \a
    \end{equation}
wherein $\Box(x) := x \tensor 1 + 1 \tensor x$ (see \cite[Section 3.2]{etingof_schiffmann_lectures_on_quantum_groups}). From this, we see that $\calr$ must also satisfy the property whereby:
    \begin{equation} \label{equation: coboundary_unitarity}
        \begin{gathered}
            \Sym(\calr) := \frac12( \calr_{1, 2} + \calr_{2, 1} ) \in (\a \hattensor \a)^{\a}
        \end{gathered}
    \end{equation}
i.e. whereby $\Sym(\calr)$ is $\a$-invariant (which is equivalent to requiring that $[\Box(x), \Sym(\calr)] = 0$ for all $x \in \a$), which can be referred to as \textbf{coboundary unitarity}. \textit{A priori}, any $2$-tensor can be decomposed into the sum of its symmetric and alternating component. In particular, one can verify that $\Alt(\calr) := \calr - \Sym(\calr) = \frac12(\calr_{1, 2} - \calr_{2, 1})$ is alternating, and obviously $\calr = \Sym(\calr) + \Alt(\calr)$ by construction. Therefore, we can work under the assumption that:
    $$\calr \in \bigwedge^2 \a \subset \a \hattensor \a$$
without any loss of generality.

It is then natural to pose the following question.
\begin{question}
    Which alternating $2$-tensor $\calr \in \bigwedge^2 \a$ gives rise to a topological Lie bialgebra structure $\delta: \a \to \a \hattensor \a$ by means of equation \eqref{equation: coboundary_lie_cobrackets} ?
\end{question}
Of course, one should not expect that every alternating $2$-tensor $\calr \in \bigwedge^2 \a$ would give rise to a Lie bialgebra structure via formula \eqref{equation: unitary_classical_r_matrices}, and this is because for a general alternating tensor $\calr \in \bigwedge^2 \a$, the map $\delta := [\Box, \calr]: \a \to \a \hattensor \a$ may not satisfy the co-Jacobi identity. For any $\calr \in \a \hattensor \a$, let us write:
    \begin{equation} \label{equation: classical_yang_baxter_tensor}
        \schouten{\calr, \calr} := [\calr_{1, 2}, \calr_{1, 3}] + [\calr_{1, 2}, \calr_{2, 3}] + [\calr_{1, 3}, \calr_{2, 3}] \in \a \hattensor \a \hattensor \a
    \end{equation}
to denote the \textbf{classical Yang-Baxter tensor}\footnote{Written in Schouten bracket notations (cf. \cite[Chapter 3]{chari_pressley_quantum_groups}).}. What Drinfeld discovered, which we recall in lemma \ref{lemma: coboundary_lie_bialgebras_and_CYBEs}, is that $\calr$ is a coboundary structure precisely when the $3$-tensor $\schouten{\calr, \calr}$ is invariant. When $\schouten{\calr, \calr} = 0$ (which is indeed invariant), we have the so-called \say{classical Yang-Baxter equation}, which when satisfied by our coboundary structure $\calr$, guarantees that the classical mechanical system described by the Poisson structure associated to $\calr$ is integrable; see \cite[Section 2.3]{chari_pressley_quantum_groups} for more details.
\begin{remark}[Schouten-Gerstenhaber brackets]
    
\end{remark}
\begin{lemma}[Drinfeld] \label{lemma: coboundary_lie_bialgebras_and_CYBEs}
    An alternating $2$-tensor $\calr \in \bigwedge^2 \a$ gives rise to a coboundary topological Lie bialgebra structure $\delta: \a \to \a \hattensor \a$ given by equation \eqref{equation: coboundary_lie_cobrackets} if and only if $\schouten{\calr, \calr}$ is an invariant alternating element of $\a \hattensor \a \hattensor \a$, i.e. if and only if $\schouten{\calr, \calr} \in \left( \bigwedge^3 \a \right)^{\a}$.
\end{lemma}
    \begin{proof}
        See \cite[Theorem 3.1]{etingof_schiffmann_lectures_on_quantum_groups}.
    \end{proof}

\begin{definition}[Quasi-triangular topological Lie bialgebras] \label{def: quasi_triangular_topological_lie_algebras}
    A coboundary topological Lie bialgebra $(\a^+, \delta^+, \calr)$ is \textbf{quasi-triangular} if and only if $\calr$ satisfies the following two properties. For such topological Lie bialgebras, $\calr$ is referred to as a \textbf{quasi-triangular structure} or a \textbf{classical r-matrix}.
    \begin{itemize}
        \item The first is the \textbf{classical Yang-Baxter equation (CYBE)}\footnote{In other words, quasi-triangular structures/classical r-matrices $\calr$ are solutions to the equation $\schouten{\calr, \calr} = 0$.}, which reads:
            \begin{equation} \label{equation: CYBEs}
                \schouten{\calr, \calr} = 0
            \end{equation}
        \item The second, which shall be referred to as \textbf{quasi-unitarity}, is the property whereby:
            \begin{equation} \label{equation: quasi_unitary_classical_r_matrices}
                \Sym(\calr) := \frac12( \calr_{1, 2} + \calr_{2, 1} ) \in \bbk \cdot \calr_{\a^+, \a^-}^0
            \end{equation}
        wherein $\calr_{\a^+, \a^-}^0$ is the canonical element of the pairing $(\cdot, \cdot)_{\Dr(\a^+)}$, i.e. the pullback of $\id_{\a}$ along the following composition $\a^+ \hattensor \a^- \hookrightarrow \a \hattensor \a \hookrightarrow \End(\a)$.
    \end{itemize}
\end{definition}
\begin{definition}[Triangular topological Lie bialgebras] \label{def: triangular_topological_lie_bialgebras}
    Suppose that we have a quasi-triangular topological Lie bialgebra $(\a^+, \delta^+, \calr)$. If, moreover, the classical r-matrix $\calr$ is \textbf{unitary}, meaning that:
        \begin{equation} \label{equation: unitary_classical_r_matrices}
            \Sym(\calr) = 0 \iff \calr_{1, 2} = -\calr_{2, 1}
        \end{equation}
    then the (coboundary) topological bialgebra $(\a^+, \delta^+, \calr^+)$ will be called \textbf{triangular}.
\end{definition}
\begin{example}[Canonical quasi-triangular structures on classical doubles] \label{example: canonical_quasi_triangular_structures_on_classical_doubles}
    Consider a topological Lie bialgebra $(\a^+, \delta^+)$ defined by a Manin triple $(\a, \a^+, \a^-)$, along with its classical double $(\Dr(\a^+), \delta_{\Dr(\a^+)})$. Recall also, that there is a canonically induced Manin triple:
        $$( \Dr(\a^+), \a^+, \a^- )$$
    whose bilinear form we shall denote by:
        $$(\cdot, \cdot)_{\Dr(\a^+)}$$
    Then, its classical double $(\Dr(\a^+), \delta_{\Dr(\a^+)})$ (cf. remark \ref{remark: embeddings_into_classical_doubles}) will always carry a canonical quasi-triangular structure. In particular, it can be shown that:
        \begin{equation} \label{equation: quasi_triangularity_of_finite_dimensional_classical_doubles}
            \delta_{\Dr(\a^+)} = [\Box, \calr_{\a^+, \a^-}^0]
        \end{equation}
    wherein $\calr_{\a^+, \a^-}^0$ is the canonical element of the pairing $(\cdot, \cdot)_{\Dr(\a^+)}$.
\end{example}
\begin{remark}[Non-canonical coboundary structures]
    Suppose that $(\a^+, \delta^+, \calr)$ is a coboundary Lie bialgebra. Then, the classical double $\Dr(\a^+) \cong \a$ possesses an induced coboundary structure given by the same fomrmula, since:
        $$\delta_{\Dr(\a^+)} = [\Box, \calr]$$
    as maps $\Dr(\a^+) \to \Dr(\a^+) \hattensor \Dr(\a^+)$, per equation \ref{equation: classical_double_cobrackets}.
    
    We would like to note, that in general, the coboundary structure $\calr$ on the classical double $\Dr(\a^+)$ may not coincide with the canonical quasi-triangular structure $\calr_{\a^+, \a^-}^0$ from example \ref{example: canonical_quasi_triangular_structures_on_classical_doubles}. Thus in general, even though $\Dr(\a^+)$ is always quasi-triangular via $\calr_{\a^+, \a^-}^0$, this quasi-triangularity may not be inherited by the isotropic Lie subalgebra $\a^+ \subset \a$. Otherwise, we would always have that $\calr = \calr_{\a^+, \a^-}^0$ inside $Z^0_{\Lie}\left(\a, \a \hattensor \a\right) \cong \a \hattensor \a$, which is clearly nonsensical. 
\end{remark}

\begin{definition}[Gauge equivalences] \label{def: gauge_equivalences}

\end{definition}

\todo[inline]{Classical r-matrices with spectral parameters}