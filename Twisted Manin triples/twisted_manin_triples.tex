\section{Twisted Manin triples and quantisation of Lie coideals}
    \subsection{Twisted Manin triples and Lie coideals}
        For a recollection of basic facts about Manin triples, we refer the reader to subsection \ref{subsection: manin_triples}. We freely employ the notations therein.
    
        \begin{definition}[Twisted Manin triple] \label{def: twisted_manin_triples}
            Suppose that $(\g, \h)$ is a weighted Lie algebra, that $(\g, \g^+, \g^-)$ is a Manin triple, and consider some pseudo-involution\footnote{In principle, definition \ref{def: twisted_manin_triples} can be written down without the automorphism $\vartheta$ being pseudo-involutive, but such scenarios are beyond the scope of our purposes.} $\vartheta \in \PseudoInv(\g, \h)$. Also, denote the bilinear form associated to the Manin triple above by $(\cdot, \cdot)_{\g}$.
            \begin{itemize}
                \item Said pseudo-involution $\vartheta$ is then said to \textbf{twist} the aforementioned Manin triple if either $\vartheta(\g^{\pm}) = \g^{\pm}$ or $\vartheta(\g^{\pm}) = -\g^{\mp}$.
                \item The $\vartheta$-twisting above is said to be \textbf{invariant} (respectively, \textbf{anti-invariant}) if $( \vartheta(x), y )_{\g} = ( x, \vartheta(y) )_{\g}$ (respectively, if $( \vartheta(x), y )_{\g} = -( x, \vartheta(y) )_{\g}$).
            \end{itemize}
        \end{definition}

        \begin{theorem}[Lie bialgebras and coideals from invariantly twisted Manin triples] \label{theorem: lie_bialgebras_from_invariantly_twisted_manin_triples}
            Suppose that $(\g, \h, \n^{\pm}, \omega)$ is a root-decomposable Lie algebra, that $(\g, \g^+, \g^-)$ is a Manin triple twisted by some pseudo-involution $\vartheta \in \PseudoInv(\g, \h)$; also, let us write $\delta^{\pm}$ for the Lie bialgebra structures on $\g^{\pm}$. Next, let:
                $$\k$$
            be the $\vartheta$-pseudo-fixed-point subalgebra of $\g$ constructed in proposition \ref{prop: constructing_pseudo_fixed_point_subalgebras}, and let:
                $$\k^{\pm} := \k \cap \g^{\pm}$$
            \begin{enumerate}
                \item There is a topological Lie bialgebra structure $\delta_{\k^+}: \k^+ \to \k^+ \hattensor \k^+$, given by $\delta_{\k^+} := [\cdot, \cdot]_{\k^-}^*$, if and only if the $\vartheta$-twisting is \textit{invariant}. This is \textit{not} a Lie sub-bialgebra of $(\g^+, \delta^+)$.
                \item The restriction $\delta_{\k^+} := \delta^+|_{\k^+}$ is a Lie right-coideal structure $\delta_{\k^+}: \k^+ \to \k^+ \hattensor \g^+$ if and only if the $\vartheta$-twisting is \textit{anti-invariant}.
            \end{enumerate}
        \end{theorem}
        \begin{remark}
            Before we begin the proof, let us remark that we are forced to work with a root-decomposable Lie algebra instead of a general weighted one because so that we can make use of the pseudo-fixed-point subalgebra $\vartheta$ from proposition \ref{prop: constructing_pseudo_fixed_point_subalgebras}. It would be interesting to know if this assumption can be relaxed, or indeed, even done away with entirely.
        \end{remark}
            \begin{proof}
                Let $\g^{\pm} := \k^{\pm} \oplus \m^{\pm}$ be a splitting of vector spaces.
                \begin{enumerate}
                    \item 
                    \item 
                \end{enumerate}
            \end{proof}

        \begin{lemma}[Root-decomposable Lie bialgebras] \label{lemma: root_decomposable_lie_bialgebras}
            Suppose that $(\g, \h, \n^{\pm}, \omega)$ is a root-decomposable Lie algebra, and consider the Borel subalgebras $\b^{\pm} := \h \oplus \n^{\mp}$. Next, let us form the Lie algebra:
                $$\a := \b^+ \oplus \b^-$$
            into which $\b^{\pm}$ embed by means of the maps $\eta^{\pm}: \b^{\pm} \hookrightarrow \a$ given by:
                \begin{equation} \label{equation: root_decomposable_borel_subalgebra_embeddings}
                    \eta^{\pm}(x) := x \oplus ( \pm x_{\h} ) \quad, \quad x \in \b^{\pm}
                \end{equation}
            (cf. equation \eqref{equation: borel_lie_sub_bialgebra_embeddings}), wherein $x_{\h}$ is the image of $x \in \b^{\pm}$ under the canonical quotient map $\b^{\pm} \to \b^{\pm}/\n^{\pm} \cong \h$. Moreover, this larger Lie algebra shall be equipped with the symmetric, non-degenerate, and invariant bilinear form given by:
                \begin{equation} \label{equation: extended_root_decomposable_pairings}
                    (\cdot, \cdot)_{\a} := (\cdot, \cdot)_{\g} - (\cdot, \cdot)_{\h}
                \end{equation}
            Then, with respect to the bilinear form above, we have that:
                \begin{equation} \label{equation: root_decomposable_manin_triples}
                    (\a, \eta^+(\b^+), \eta^-(\b^-))
                \end{equation}
            is a Manin triple.
        \end{lemma}
            \begin{proof}
                First of all, via lemma \ref{lemma: direct_sums_of_weighted_lie_algebras}, we see that $\a$ is weighted by the abelian Lie subalgebra $\eta^+(\h) \oplus \eta^-(\h)$. Next, from remark \ref{remark: isotropic_subalgebras_of_root_decomopsable_lie_algebras}, it is clear that the Lie subalgebras $\eta^{\pm}(\b^{\pm}) \subset \a$ are isotropic with respect to $(\cdot, \cdot)_{\a}$. The rest follows from the construction of the triple $(\a, \eta^+(\b^+), \eta^-(\b^-))$.
            \end{proof}
        \begin{proposition}[Twisted Manin triples of types I and II] \label{prop: twisted_manin_triples_of_types_I_and_II}
            Suppose that $(\g, \h, \n^{\pm}, \omega)$ is a symmetrisable Kac-Moody algebra, and consider a pseudo-involution $\vartheta \in \PseudoInv(\g, \h)$. Suppose also that the Manin triple \eqref{equation: root_decomposable_manin_triples} is twisted by $\vartheta$, invariantly or anti-invariantly. Then:
            \begin{enumerate}
                \item $\vartheta(\n^{\pm}) = \n^{\pm}$ if and only if $\vartheta$ is of type I.
                \item $\vartheta(\n^{\pm}) = -\n^{\mp}$ if and only if $\vartheta$ is of type II.
            \end{enumerate}
        \end{proposition}
            \begin{proof}
                First of all, by definition \ref{def: weighted_automorphisms_of_weighted_lie_algebras}, we know that $\vartheta|_{\h} \in \Inv_{\Lie\Alg}(\h)$. The map $\vartheta|_{\h}$ thus extends to an involution on the weighing subalgebra $\eta^+(\h) \oplus \eta^-(\h) \subset \a$, and hence a pseudo-involution on $\a$, which we denote by the same symbol $\vartheta$.
            
                Let $\frakG$ be the minimal pro-algebraic group associated to $(\g, \h, \n^{\pm}, \omega)$, which exists because this Lie algebra is integrable. Since $(\cdot, \cdot)_{\g}$ is $\g$-invariant, it is also $\frakG$-invariant, i.e. $(\Ad(g) \cdot x, y)_{\g} = (x, \Ad(g) \cdot y)$ for all $x, y \in \g$ and all $g \in \frakG$. This means that for all $g \in \frakG$, we have in the $\vartheta$-invariant case that $( \Ad(g) \cdot \vartheta(x), y )_{\g} = ( x, \Ad(g) \cdot \vartheta(y) )_{\g}$ as a result of the fact that $( \vartheta(x), y )_{\g} = ( x, \vartheta(y) )_{\g}$ (cf. definition \ref{def: twisted_manin_triples}), and similarly in the anti-invariant case. By definition, $\vartheta$ is of type II if and only if we can choose $g \in \frakG$ such that $\omega = \Ad(g) \cdot \vartheta$.  As the bilinear form $(\cdot, \cdot)_{\g}$ is $\omega$-invariant by definition \ref{def: root_decomposable_lie_algebras}, and as $\omega(\n^{\pm}) = -\n^{\mp}$ by definition \ref{def: triangular_decomposable_lie_algebras}, we are done.
            \end{proof}

    \subsection{Deformation quantisation}

    \subsection{Twisted classical r-matrices, classical k-matrices, and their quantisations}