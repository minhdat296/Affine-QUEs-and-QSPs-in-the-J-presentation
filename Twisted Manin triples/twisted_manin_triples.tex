\section{Twisted Manin triples and quantisation of Lie coideals}
    \subsection{Twisted Manin triples and Lie coideals}
        For a recollection of basic facts about Manin triples, we refer the reader to subsection \ref{subsection: manin_triples}. We freely employ the notation scheme therein.
    
        \begin{definition}[Twisted Manin triple] \label{def: twisted_manin_triples}
            Suppose that $(\a, \t)$ is a weighted Lie algebra, that $(\a, \a^+, \a^-)$ is a Manin triple constructed using a symmetric, non-degenerate, and invariant bilinear form $(\cdot, \cdot)_{\a}$ on $\a$, and consider some pseudo-involution\footnote{In principle, definition \ref{def: twisted_manin_triples} can be written down without the automorphism $\vartheta$ being pseudo-involutive, but such scenarios are beyond the scope of our purposes.} $\vartheta \in \PseudoInv(\a, \t)$.
            \begin{itemize}
                \item Said pseudo-involution $\vartheta$ is then said to \textbf{twist} the aforementioned Manin triple if either $\vartheta(\a^{\pm}) = \a^{\pm}$ or $\vartheta(\a^{\pm}) = -\a^{\mp}$.
                \item The $\vartheta$-twist above is said to be \textbf{invariant} (respectively, \textbf{anti-invariant}) if $( \vartheta(x), y )_{\a} = ( x, \vartheta(y) )_{\a}$ (respectively, if $( \vartheta(x), y )_{\a} = -( x, \vartheta(y) )_{\a}$).
            \end{itemize}
        \end{definition}
        \begin{definition}[Twisted topological Lie bialgebras] \label{def: twisted_lie_bialgebras}
            Let $\a$ be a Lie algebra and let $\delta: \a \to \a \hattensor \a$ be a topological Lie bialgebra, and consider a Lie algebra automorphism $\vartheta \in \Aut_{\Lie\Alg}(\a)$ (which may not be a Lie bialgebra automorphism). Let us write:
                $$\delta^{\vartheta} := \delta \circ \vartheta$$
            We say that $\delta^{\vartheta}$ is an \textbf{invariant twist} (respectively, \textbf{anti-invariant twist}) of the Lie bialgebra structure $\delta$ if:
                $$\delta^{\vartheta} = (\vartheta \hattensor \vartheta) \circ \delta$$
            (respectively, if $\delta^{\vartheta} = -(\vartheta \hattensor \vartheta) \circ \delta$).
        \end{definition}
        \begin{remark}
            We caution the reader that in general, for a topological Lie bialgebra $(\a, \delta)$, twists thereof by Lie algebra automorphisms $\vartheta \in \Aut_{\Lie\Alg}(\a)$ may not be a Lie bialgebra structure on $\vartheta(\a)$ at all. In theorem \ref{theorem: lie_bialgebras_from_twisted_manin_triples}, for instance, we will see that at least for twists by pseudo-involutions on root-decomposable Lie algebras, whether said twists are invariant or anti-invariant shall determined if the resulting structure is a topological Lie bialgebra or not.
        \end{remark}
        \begin{example}[Yangian twisted current algebras] \label{example: yangian_twisted_current_algebras}
            An example of an anti-invariant twist of a topological Lie bialgebra structure that results not in a new topological Lie bialgebra structure, but rather a topological Lie coideal subalgebra structure, can be found in \cite[Subsection 3.5]{guay_regelskis_twisted_yangians_for_symmetric_pairs_of_types_BCD}. There, the authors were concerned with the standard Yangian topological Lie bialgebra structure $(\bar{\g}[t], \delta^+)$ associated to a finite-dimensional simple Lie algebra $\bar{\g}$, and twists given by certain involutions on $\bar{\g}[t]$.
        \end{example}

        Before we are able to see what sort of Lie-bialgebraic structures can be constructed using twisted Manin triples, let us make the following preliminary observation. Namely, there is a canonical Lie bialgebra structure associated to any root-decomposable Lie algebra, integrable or not, i.e. we are making note of a (mild) generalisation of the standard Lie bialgebra structure on symmetrisable Kac-Moody algebras (as given by equation \eqref{equation: standard_kac_moody_lie_bialgebra_structure}), instead of merely recovering it. Our claim, later on, shall be that depending on whether the Manin triple in question is being twisted invariantly or anti-invariantly, the resulting Lie-bialgebraic structure shall either be a twisted Lie bialgebra (that is \textit{not} a Lie sub-bialgebra of the original one!) or a Lie coideal.
        \begin{lemma}[Root-decomposable Lie bialgebras] \label{lemma: root_decomposable_lie_bialgebras}
            Suppose that $(\g, \h, \n^{\pm}, \omega)$ is a root-decomposable Lie algebra, and consider the Borel subalgebras $\b^{\pm} := \h \oplus \n^{\mp}$. Next, let us form the Lie algebra:
                $$\a := \b^+ \oplus \b^-$$
            into which $\b^{\pm}$ embed by means of the maps $\eta^{\pm}: \b^{\pm} \hookrightarrow \a$ given by:
                \begin{equation} \label{equation: root_decomposable_borel_subalgebra_embeddings}
                    \eta^{\pm}(x) := x \oplus ( \pm x_{\h} ) \quad, \quad x \in \b^{\pm}
                \end{equation}
            (cf. equation \eqref{equation: borel_lie_sub_bialgebra_embeddings}), wherein $x_{\h}$ is the image of $x \in \b^{\pm}$ under the canonical quotient map $\b^{\pm} \to \b^{\pm}/\n^{\pm} \cong \h$. Moreover, this larger Lie algebra shall be equipped with the symmetric, non-degenerate, and invariant bilinear form given by\footnote{We are implicitly identifying $\a \cong \g \oplus \h$ as vector spaces.}:
                \begin{equation} \label{equation: extended_root_decomposable_pairings}
                    (\cdot, \cdot)_{\a} := (\cdot, \cdot)_{\g} - (\cdot, \cdot)_{\h}
                \end{equation}
            Then, with respect to the bilinear form above, we have that:
                \begin{equation} \label{equation: root_decomposable_manin_triples}
                    (\a, \a^+, \a^-)
                \end{equation}
            wherein $\a^{\pm} := \eta^{\pm}(\b^{\pm})$, is a Manin triple.
        \end{lemma}
            \begin{proof}
                First of all, via lemma \ref{lemma: direct_sums_of_weighted_lie_algebras}, we see that $\a$ is weighted by the abelian Lie subalgebra $\eta^+(\h) \oplus \eta^-(\h)$. Next, from remark \ref{remark: isotropic_subalgebras_of_root_decomopsable_lie_algebras}, it is clear that the Lie subalgebras $\a^{\pm} \subset \a$ are isotropic with respect to $(\cdot, \cdot)_{\a}$. The rest follows from the construction of the triple $(\a, \a^+, \a^-)$.
            \end{proof}

        \begin{lemma}[Doubling weighted automorphisms] \label{lemma: doubling_weighted_automorphisms}
            Suppose that $(\g, \h, \n^{\pm}, \omega)$ is a root-decomposable Lie algebra, and consider a pseudo-involution $\vartheta \in \PseudoInv(\g, \h)$. Consider also the standard Manin triple $(\a, \a^+, \a^-)$ \eqref{equation: root_decomposable_manin_triples}. Then, $\vartheta$ extends to a pseudo-involution:
                \begin{equation} \label{equation: doubled_pseudo_involutions}
                    \Theta \in \PseudoInv(\a, \t)
                \end{equation}
            of the weighted Lie algebra $(\a, \t)$ with $\t := \eta^+(\h) \oplus \eta^-(\h)$, given by:
                $$\Theta( \eta^{\pm}(x) ) := \vartheta(x) \oplus \vartheta( \pm x_{\h} ) \quad, \quad x \in \b^{\pm}$$
            wherein $x_{\h}$ is the image of $x \in \b^{\pm}$ under the canonical quotient map $\b^{\pm} \to \b^{\pm}/\n^{\pm} \cong \h$.
        \end{lemma}
            \begin{proof}
                First of all, let us check that $\t$ is $\vartheta$-stable. To this end, note that for any $h \in \h$, we have $\Theta( \eta^{\pm}(h) ) = \vartheta(h) \oplus \vartheta(\pm h)$, which is certainly still an element of $\t$, since $\h$ is $\vartheta$-stable. Secondly, to show that $\Theta|_{\t}^2 = \id_{\t}$, simply consider the fact that $\Theta^2( \eta^{\pm}(h) ) = \vartheta^2(h) \oplus \vartheta^2(\pm h) = h \oplus (\pm h) = \eta^{\pm}(h)$, which is true for all $h \in \h$. We have thus shown that $\Theta$ is a pseudo-involution of the weighted Lie algebra $(\a, \t)$.
            \end{proof}
        \begin{convention}
            Often, we shall abuse notations slightly and say that the standard Manin triple $(\a, \a^+, \a^-)$ \eqref{equation: root_decomposable_manin_triples} is twisted by $\vartheta \in \PseudoInv(\g, \h)$, even though strictly speaking, this Manin triple has to be twisted by the \say{doubled} pseudo-involution $\Theta$ \eqref{equation: doubled_pseudo_involutions} instead.
        \end{convention}

        \begin{lemma}[Chevalley involutions twist anti-invariantly] \label{lemma: chevalley_involutions_twist_anti_invariantly}
            Suppose that $(\g, \h, \n^{\pm}, \omega)$ is a root-decomposable Lie algebra. Consider also the standard Manin triple $(\a, \a^+, \a^-)$ \eqref{equation: root_decomposable_manin_triples}. Then, the Chevalley involution $\omega$ extends - in the manner explained in lemma \ref{lemma: doubling_weighted_automorphisms} - to an \underline{anti-invariant} twist of the Manin triple $(\a, \a^+, \a^-)$. 
        \end{lemma}
            \begin{proof}
                \todo[inline]{There are some typos I need to fix this first before writing this proof.}
            \end{proof}
            
        \begin{theorem}[Lie bialgebras and coideals from twisted Manin triples] \label{theorem: lie_bialgebras_from_twisted_manin_triples}
            Suppose that $(\g, \h, \n^{\pm}, \omega)$ is a root-decomposable Lie algebra, and consider a pseudo-involution $\vartheta \in \PseudoInv(\g, \h)$. Consider also the standard Manin triple $(\a, \a^+, \a^-)$ \eqref{equation: root_decomposable_manin_triples}; denote the topological Lie bialgebra structures on $\a^{\pm}$ by $\delta^{\pm}: \a^{\pm} \to \a^{\pm} \hattensor \a^{\pm}$. Let $\k \subset \g$ be a $\vartheta$-pseudo-fixed-point subalgebra, and abbreviate $\k^{\pm} := \k \cap \a^{\pm}$.
            \begin{enumerate}
                \item There is a topological Lie bialgebra structure $\delta_{\k^+}: \k^+ \to \k^+ \hattensor \k^+$, given by $\delta_{\k^+} := [\cdot, \cdot]_{\k^-}^*$, if and only if the $\vartheta$-twist is \textit{invariant}. This is \textit{not} a Lie sub-bialgebra of $(\a^+, \delta^+)$.
                \item The restriction $\delta_{\k^+} := \delta^+|_{\k^+}$ is a topological Lie right-coideal structure $\delta_{\k^+}: \k^+ \to \k^+ \hattensor \a^+$ if and only if the $\vartheta$-twist is \textit{anti-invariant}.
            \end{enumerate}
        \end{theorem}
        \begin{remark}
            Before we begin the proof, let us remark that we are forced to work with a root-decomposable Lie algebra instead of a general weighted one because so that we can make use of the pseudo-fixed-point subalgebra $\vartheta$ from proposition \ref{prop: constructing_pseudo_fixed_point_subalgebras}. It would be interesting to know if this assumption can be relaxed, or indeed, even done away with entirely.
        \end{remark}
            \begin{proof}
                Let $\a^{\pm} := \k^{\pm} \oplus \m^{\pm}$ be a splitting of vector spaces. 
                \begin{enumerate}
                    \item 
                    \item 
                \end{enumerate}
            \end{proof}

        Regarding twists of the standard Kac-Moody Manin triple (that gives rise to the topological Lie bialgebra structure \eqref{equation: standard_kac_moody_lie_bialgebra_structure}), more can be said, especially when the underlying symmetrisable Kac-Moody algebra is either of finite or affine type.
        \begin{theorem} \label{theorem: twisted_forms_and_coideal_subalgebras_of_kac_moody_lie_bialgebras}
            Suppose that $(\g, \h, \n^{\pm}, \omega)$ is a symmetrisable Kac-Moody algebra, and consider a pseudo-involution $\vartheta \in \PseudoInv(\g, \h)$. Consider also the standard Manin triple $(\a, \a^+, \a^-)$ \eqref{equation: root_decomposable_manin_triples}.
            \begin{enumerate}
                \item If $\vartheta$ twists the Manin triple $(\a, \a^+, \a^-)$ invariantly, then it will be of type I\footnote{By remark \ref{remark: finite_type_kac_moody_algebras_only_have_type_II_weighted_automorphisms}, this forces $\g$ to be of an infinite (i.e. affine or indefinite) type in Kac's classification.}.
                \item $\vartheta$ is of type II, i.e. $\vartheta \in \Ad(\frakG) \cdot \omega$, then it will twist the Manin triple $(\a, \a^+, \a^-)$ anti-invariantly as a result of the Chevalley involution $\omega$ twisting anti-invariantly (lemma \ref{lemma: chevalley_involutions_twist_anti_invariantly}).
            \end{enumerate}
            Moreover, the converse to both of the statements above hold when $\g$ is of a finite or an affine type in Kac's classification.
        \end{theorem}
            \begin{proof}
                Let $\frakG$ be the minimal pro-algebraic group associated to $(\g, \h, \n^{\pm}, \omega)$, which exists because this Lie algebra is integrable.
                
                Since $(\cdot, \cdot)_{\a}$ is $\a$-invariant, it is also $\frakG$-invariant, i.e. $(\Ad(g) \cdot x, y)_{\a} = (x, \Ad(g) \cdot y)_{\a}$ for all $x, y \in \a$ and all $g \in \frakG$. This means that for all $g \in \frakG$, we have in the $\vartheta$-invariant case that $( \Ad(g) \cdot \vartheta(x), y )_{\a} = ( x, \Ad(g) \cdot \vartheta(y) )_{\a}$ as a result of the fact that $( \vartheta(x), y )_{\a} = ( x, \vartheta(y) )_{\a}$ (cf. definition \ref{def: twisted_manin_triples}), and similarly in the anti-invariant case. At the same time, we have seen through lemma \ref{lemma: chevalley_involutions_twist_anti_invariantly} that the Chevalley involution $\omega$ twists the Manin triple $(\a, \a^+, \a^-)$ \textit{anti-invariantly}. By definition \ref{def: weighted_automorphisms_of_types_I_and_II}, $\vartheta$ is of type II if and only if we can choose $g \in \frakG$ such that $\vartheta = \Ad(g) \cdot \omega$. We conclude the proof by combining all of the observations above.
            \end{proof}
        \begin{corollary}[Twisted forms and coideal subalgebras of Kac-Moody Lie bialgebras] \label{coro: twisted_forms_and_coideal_subalgebras_of_kac_moody_lie_bialgebras}
            Let $(\g, \h, \n^{\pm}, \omega)$ be a Kac-Moody algebra of either a finite or an affine type, and let $\vartheta \in \PseudoInv(\g, \h)$ be a pseudo-involution. Consider also a $\vartheta$-pseudo-fixed-point subalgebra $\k \subset \g$.
            \begin{enumerate}
                \item \todo[inline]{Twisted forms} 
                \item \todo[inline]{Coideal subalgebras}
            \end{enumerate}
        \end{corollary}
            \begin{proof}
                Combine the results of theorems \ref{theorem: lie_bialgebras_from_twisted_manin_triples} and \ref{theorem: twisted_forms_and_coideal_subalgebras_of_kac_moody_lie_bialgebras}.
            \end{proof}
            
        \begin{remark}[Type-I anti-invariant twists ?]
            It is possible that the converse to the two statements in theorem \ref{theorem: twisted_forms_and_coideal_subalgebras_of_kac_moody_lie_bialgebras} can hold when $(\g, \h, \n^{\pm}, \omega)$ is of an indefinite type, or even when it is a general root-decomposable Lie algebra. For instance, in the indefinite case, there may exist pseudo-involutions that twist the Manin triple $(\a, \a^+, \a^-)$ anti-invariantly yet are of type I, or in other words, there may exist pseudo-involutions that twist invariantly yet lie \textit{outside} of the orbit $\Ad(\frakG) \cdot \omega$. We are, however, unable to confirm nor deny this speculation, due to our lack of understanding of the structure of indefinite-type Kac-Moody groups.
            
            It is impossible, though, for an invariant twist to be of type II, and this is because by definition, all such pseudo-involutions are defined to be in the orbit $\Ad(\frakG) \cdot \omega$, and $\omega$ itself is already an anti-invariant twist.

            We would also like to remark, that theorem \ref{theorem: twisted_forms_and_coideal_subalgebras_of_kac_moody_lie_bialgebras} does not contradict the result mentioned in example \ref{example: yangian_twisted_current_algebras}, which showcases a topological Lie algebra \textit{anti-invariantly} twisted by a type-I involution. This is because the Manin triples at play are different in the two situations. Ultimately, this is a matter of which non-degenerate and invariant bilinear form we are using: in the former case, the pairing is of degree $0$, while in the latter case, it is of degree $-1$.
        \end{remark}

    \subsection{Deformation quantisation}

    \subsection{Twisted classical r-matrices, classical k-matrices, and their quantisations}