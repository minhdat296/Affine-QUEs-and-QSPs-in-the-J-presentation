\section{Triangular-decomposable Lie algebras and their automorphisms}
    \subsection{Weighted Lie algebras and (pre)triangular-decomposable Lie algebras}
        Ideally, we would like to develop a theory of Manin triples $(\a, \a^+, \a^-)$ that are somehow twisted by general Lie algebra automorphisms $\vartheta \in \Aut_{\Lie\Alg}(\a)$, but at present, we are unfortunately at a loss for how to write down a description of such twisted Manin triples. Guided by the theory of quantum symmetric pairs (QSPs) that arise from so-called \say{pseudo-involutions} on affine Kac-Moody algebras (see e.g. \cite{regelskis_vlaar_reflection_matrices_coideal_subalgebras}, \cite{kolb_kac_moody_QSPs}, and references therein), we have chosen to restrict our consideration down to the class of Lie algebras that admit so-called \say{triangular decompositions}. The structure and representation theory of this class of Lie algebras has been studied extensively, notably in \cite{moody_pianzola_lie_algebras_with_triangular_decompositions}, and thanks to the fact that they admit triangular decompositions, one can speak of Borel subalgebras; the automorphisms that we shall consider shall be the ones that either stabilise or permute pairs of Borel subalgebras.
        
        Let us begin by recalling the construction of (pre)triangular-decomposable\footnote{In \textit{loc. cit.}, such Lie algebras are called \say{Lie algebras admitting triangular decompositions}, but we find the terminology somewhat cumbersome.} Lie algebras from \cite[Section 2.1]{moody_pianzola_lie_algebras_with_triangular_decompositions}. Their construction starts with what we shall refer to as their \say{weighing subalgebras}: this is a pair:
            $$(\h, V)$$
        consisting of a non-zero abelian Lie algebra $\h \not = 0$ and a choice of a representation $\h \to \gl(V)$. Then, for any linear functional $\lambda \in \h^*$, a \say{weight vector} of weight $\lambda$ shall be an element $v \in V$ that is a simultaneous eigenvector of all the scalars in the set $\{ \lambda(h) \}_{h \in \h}$, i.e.:
            $$h \cdot v = \lambda(h) v \quad, \quad h \in \h$$
        For any $\lambda \in \h^*$, the vector subspace of $V$ spanned by weight-$\lambda$ vectors, i.e.:
            $$V_{\lambda} := \{ v \in V \mid \forall h \in \h: h \cdot v = \lambda(h) v \}$$
        is called the \say{subspace of weight $\lambda$}. If $V = \sum_{\lambda \in \h^*} V_{\lambda}$ as vector subspaces of $V$, then we shall say that $V$ admits a \say{weight space decomposition} with respect to $\h$; \textit{a priori}, this sum is necessarily direct if it exists, i.e.:
            $$V = \bigoplus_{\lambda \in \h^*} V_{\lambda}$$
        (see \cite[Section 2.1, Proposition 1]{moody_pianzola_lie_algebras_with_triangular_decompositions}). In order to avoid redundancy, let us refer to only the functionals $\lambda \in \h^*$ such that $V_{\lambda} \not = 0$ as \say{weights} of $V$, and the subset of $\h^*$ consisting of weights of $V$ is denoted by $\weight(V, \h)$.

        Next, consider another Lie algebra $\g$, on which our weighing algebra $\h$ acts via derivations, i.e. there exists a Lie algebra homomorphism $\h \to \der(\g)$, such that there exists a weight space decomposition:
            $$\g = \bigoplus_{\alpha \in \h^*} \g_{\alpha}$$
        It is then possible to show that:
            $$[\g_{\alpha}, \g_{\beta}] = \g_{\alpha + \beta} \quad, \quad \alpha, \beta \in \h^*$$
        and consequently, the Lie algebra $\g$ is graded by the abelian group $\rootlattice(\g, \h) := \Z\weight(\g, \h)$. Conversely, it is also possible to show that if there exists a set of weights $P \subset \weight(\g, \h)$ and a set of weight vectors $\{ e_{\alpha} \}_{\alpha \in P}$ such that $\g$ is generated as a Lie algebra by $\h \cup \{ e_{\alpha} \}_{\alpha \in P}$, then $\g = \bigoplus_{\alpha \in \h^*} \g_{\alpha}$ wherein $e_{\alpha} \in \g_{\alpha}$ for all $\alpha \in P$. For details on these facts, see \cite[Subsection 2.1, Proposition 2]{moody_pianzola_lie_algebras_with_triangular_decompositions}. Therefore, it makes sense to refer to a Lie algebra $\g$ generated by weight vectors and together with a weighing algebra $\h$ in the manner above as a Lie algebra \say{weighted} by $\h$. For compactness, let us denote such a \textbf{weighted Lie algebra} as a pair:
            $$(\g, \h)$$
        \begin{remark}
            For a Lie algebra $\g$ weighted by $\h$, we have that $\g_0 \cong \h$ if $\h$ acts by inner derivations.
        \end{remark}

        \begin{definition}[(Pre)triangular-decomposable Lie algebras] \label{def: (pre)triangular_decomposable_lie_algebras}
            A \textbf{pretriangular-decomposable Lie algebra} is a weighted Lie algebra $(\g, \h)$ satisfying the following additional conditions.
            \begin{itemize}
                \item There exists non-zero Lie subalgebras $\n^{\pm} \subset \g$ such that $\g \cong \n^- \oplus \h \oplus \n^+$ as Lie algebras.
                \item There exists an involution $\omega: \g \xrightarrow[]{\cong} -\g$ (commonly called the \textbf{Cartan involution}) such that $\omega(\n^{\pm}) = -\n^{\mp}$ and $\omega|_{\h} = \id_{\h}$.
                \item $\n^{\pm}$ are $\h$-submodules of $\g$, i.e. $[\h, \n^{\pm}] \subseteq \n^{\pm}$. The Lie subalgebras $\n^{\pm}$ thus admit the induced weight space decompositions $\n^{\pm} = \bigoplus_{\alpha \in \weight(\g, \h)} \n^{\pm} \cap \g_{\alpha}$, and let us require furthermore that:
                    $$\rootlattice^{\pm}(\g) := \{ \alpha \in \weight(\g, \h) \setminus \{0\} \mid \n^{\pm} \cap \g_{\alpha} \not = 0 \}$$
                are free (additive) sub-semigroups of $\rootlattice(\g, \h) := \Z\weight(\g, \h)$.
                \item Finally, we require that there is a semi-group basis\footnote{... which exists because we are pre-supposing that $\rootlattice^+(\g, \h)$ is free.} $\{\alpha_i\}_{i \in \simpleroots} \subset \rootlattice^+(\g, \h)$.
            \end{itemize}
            A pretriangular-decomposable Lie algebra as above shall be denoted as a quintuple:
                $$(\g, \h, \n^+, \n^-, \omega)$$
            (though we shall usually abbreviate the signs and write $(\g, \h, \n^{\pm}, \omega)$ instead).

            If the subalgebras $\n^{\pm} \subset \g$ are nilpotent, then $(\g, \h, \n^{\pm}, \omega)$ will be called a \textbf{triangular-decomposable} Lie algebra.
        \end{definition}
        \begin{remark}
            Note that by requiring that $0 \not \in \rootlattice^{\pm}(\g)$, we have that:
                $$\rootlattice^-(\g, \h) \cap \rootlattice^+(\g, \h) = \varnothing$$
                
            Also, we should point out that the \say{upper/lower-triangular} subalgebras $\n^{\pm} \subset \g$ of a general pretriangular-decomposable Lie algebra $(\g, \h, \n^{\pm}, \omega)$ are \textit{not} necessarily even locally nilpotent, let alone nilpotent! In more combinatorial terms, in this general of a setting, it is entirely possible for the so-called \say{root strings} to be infinitely long. This is the reason for singling out the triangular-decomposable Lie algebras from the larger class of pretriangular-decomposable ones.

            Lastly, there is no guarantee in general that the weight spaces $\g_{\alpha} \subset \g$ are finite-dimensional. Pretriangular-decomposable Lie algebras with this property are said to be \textbf{regular}. Note that this property needs not imply that the Lie subalgebras $\n^{\pm}$ are nilpotent: for instance, extended Kac-Moody algebras (given by \eqref{equation: extended_kac_moody_relations}; cf. also \cite[Theorem 1.2]{kac_infinite_dimensional_lie_algebras}) are regular in the above sense, but because their generators are not subjected to the Serre relations, those algebras are not triangular-decomposable, only pretriangular-decomposable. 
        \end{remark}
        \begin{example}
            By their very construction (e.g. as explained in \cite[Theorem 1.2]{kac_infinite_dimensional_lie_algebras}), Kac-Moody algebras are pretriangular-decomposable. Via the Serre relations, one sees also that symmetrisable Kac-Moody algebras are actually triangular-decomposable.
            
            Finite-dimensional but non-Kac-Moody examples of pretriangular-decomposable Lie algebras include reductive Lie algebras, while infinite-dimensional non-Kac-Moody examples include Heisenberg algebras and Virasoro algebras, extended affine Lie algebras (EALAs) in the sense of \cite{neher_lectures_on_EALAs}, as well as the higher-nullity analogues of the Heisenberg and Virasoro algebras that arise via EALAs. Triangular-decomposable Lie algebras form a very large class.

            On the other hand, solvable Lie algebras for instance - and thus also the Lie algebras that are nilpotent, and especially the abelian ones - are \textit{not} pretriangular decomposable.
        \end{example}

    \subsection{Pseudo-involutions}
        \begin{definition}[Weighted automorphisms and pseudo-involutions] \label{def: weighted_automorphisms_of_weighted_lie_algebras}
            Consider a weighted Lie algebra $(\g, \h)$. Next, consider a Lie algebra automorphism $\vartheta \in \Aut_{\Lie\Alg}(\g)$ under which $\h$ is stable, i.e. it is such that $\vartheta|_{\hbar} \in \Aut_{\Lie\Alg}(\h)$. In such a situation, we say that this is an automorphism of $\g$ \textbf{weighted} by $\h$.
            
            A weighted automorphism $\vartheta \in \Aut(\g, \h)$ that is an involution, i.e. $\vartheta^2 = \id_{\g}$, is said to be a \textbf{weighted involution}. If we only have that $\vartheta|_{\h}^2 = \id_{\h}$, then $\vartheta$ will be called a \textbf{pseudo-involution}. 
        \end{definition}
        It is clear that weight automorphisms, pseudo-involutions, weighted involutions, and involutions of a given weighted Lie algebra $\g$ form subgroups of $\Aut_{\Lie\Alg}(\g)$. We denote these subgroups, respectively, by:
            $$\Aut(\g, \h) \quad \supset \quad \PseudoInv(\g, \h) \quad \supset \quad \Inv(\g, \h) \quad \subset \quad \Inv_{\Lie\Alg}(\g)$$
        and we note that $\Inv(\g, \h) := \Inv_{\Lie\Alg}(\g) \cap \Aut(\g, \h)$ just by definition.
        \begin{remark}
            The notion of pseudo-involution above coincides with \cite[Definition 1.1]{regelskis_vlaar_kac_moody_pseudo_symmetric_pairs}. Additionally, note that the Cartan involution $\omega \in \Aut(\g, \h)$ of a given pretriangular-decomposable Lie algebra $(\g, \h, \n^{\pm}, \omega)$ is in fact a weighted involution by definition, and the notion below is given \textit{relatively} to this choice of an involution on $\g$.
        \end{remark}

        Now, recall that any involutive endomorphism $\vartheta \in \End(V)$ on a finite-dimensional vector space $V$ has eigenvalues $\pm 1$ (and hence is automatically an automorphism too)\footnote{As long as the characteristic of the ground field is not $2$, but this is not a concern for us.}, and hence induces the following eigenspace decomposition:
            $$V = V_{\vartheta[1]} \oplus V_{\vartheta[-1]}$$
        Through this, we infer that $\a$ is a finite-dimensional Lie algebra, $\vartheta \in \Inv_{\Lie\Alg}(\a)$ , and $V := \a$ is the adjoint $\a$-module, then we can decompose the Lie algebra $\a$ in the following manner:
            \begin{equation} \label{equation: symmetric_space_decomposition}
                \a = \a_{\vartheta[1]} \oplus \a_{\vartheta[-1]}
            \end{equation}
        This is commonly called the \say{Cartan decomposition} or the \say{symmetric pair decomposition}, and we note that:
            $$\a_{\vartheta[1]} = \a^{\vartheta}$$
        i.e. the eigenspace of weight $1$ coincides - as a Lie subalgebra of $\a$ - with the fixed-point Lie subalgebra $\a^{\vartheta} := \{x \in \a \mid \vartheta(x) = x\}$.

        Non-involutive Lie algebra automorphisms, even pseudo-involutive ones, may have eigenvalues other than $\pm 1$, so a decomposition in the form \eqref{equation: symmetric_space_decomposition} is not generally available for such a Lie algebra automorphism. In order to circumvent this difficulty, the following generalisation of the notion of fixed-point subalgebras was proposed in \cite[Definition 1.2]{regelskis_vlaar_kac_moody_pseudo_symmetric_pairs}. 
        \begin{definition}[Pseudo-fixed-points subalgebras] \label{def: pseudo_fixed_point_subalgebras}
            Let $(\g, \h)$ be a weighted Lie algebra, in which the weighing subalgebra $\h$ is finite-dimensional, and fix some pseudo-involution $\vartheta \in \PseudoInv(\g, \h)$. A \textbf{pseudo-fixed-point subalgebra} of $\g$ is then a Lie subalgebra $\k$ such that:
            \begin{itemize}
                \item $\k \cap \h = \h^{\vartheta}$, and
                \item $\dim\left( \k \cap ( \g_{\alpha} + \vartheta(\g_{\alpha}) ) \right) = \dim \g_{\alpha}$ for each weight $\alpha \in \weight(\g, \h)$.
            \end{itemize}
        \end{definition}
        \begin{remark}
            To be strict about technicalities, we must point out that our definition \eqref{def: pseudo_fixed_point_subalgebras} and \cite[Definition 1.2]{regelskis_vlaar_kac_moody_pseudo_symmetric_pairs} do not coincide in full generality, though they do tend to in practice. For them, $\h$ needs to be maximal amongst the Lie subalgebras of $\g$ that act semi-simply via the adjoint representation. For us, in contrast, $\h$ merely has to act semi-simply via derivations on $\g$, which is a much weaker requirement. We would also like to remark that definition \ref{def: pseudo_fixed_point_subalgebras} (as well as \cite[Definition 1.2]{regelskis_vlaar_kac_moody_pseudo_symmetric_pairs}) makes no mention of a possibility of uniqueness for pseudo-fixed-point subalgebras.
        \end{remark}

        Now, given a weighted Lie algebra $(\g, \h)$, a pseudo-involution $(\g, \h) \in \PseudoInv(\g, \h)$ thereon, and a $\vartheta$-pseudo-fixed-point subalgebra $\k \subset \g$, let us attempt to compute the elements in:
            $$[\k, \k] \quad, \quad [\k, \g]$$
        At the level of generality described above, we do not believe that much can be said, but if we restrict our attention to the pretriangular-decomposable Lie algebras amongst the weighted ones, then the problem becomes more tractable. As such, let us fix a pretriangular-decomposable Lie algebra $(\g, \h, \n^{\pm}, \omega)$, and to be clear, we recall that by definition, such a Lie algebra admits a weight space decomposition:
            $$\g = \h \oplus \bigoplus_{\alpha \in \weight(\g, \h) \setminus \{0\}} \g_{\alpha}$$
        via the semi-simple action via derivations of $\h$ on $\g$ (see definition \ref{def: (pre)triangular_decomposable_lie_algebras}). We recall also that as a part of the definition, $\g$ is generated the weighing subalgebra $\h$ along with weight vectors $e_{\alpha} \in \g_{\alpha}$, which is what shall allow us to compute the elements in $[\k, \k]$ and in $[\k, \g]$ (again, see definition \ref{def: (pre)triangular_decomposable_lie_algebras}).
        \begin{lemma}[Adjoint actions of pseudo-fixed-point subalgebras] \label{lemma: adjoint_actions_of_pseudo_fixed_point_subalgebras}
            Consider a pretriangular-decomposable Lie algebra $(\g, \h, \n^{\pm}, \omega)$, generated by $\h \cup \{e_{\alpha}\}_{\alpha \in \rootlattice(\g, \h)}$ wherein $e_{\alpha} \in \g_{\alpha}$ are weight vectors, and consider a pseudo-involution $\vartheta \in \PseudoInv(\g, \h)$ along with a pseudo-fixed-point subalgebra $\k \subset \g$. Additionally, for each weight $\alpha \in \weight(\g, \h)$, suppose that:
                $$\dim \g_{\alpha} = |I_{\alpha}|$$
            for some set $I_{\alpha}$, and let us fix once and for all a basis:
                $$\{e_{\alpha}[i]\}_{i \in I_{\alpha}} \subset \g_{\alpha}$$
            \begin{enumerate}
                \item For each weight $\alpha \in \weight(\g, \h)$, we have:
                    \begin{equation}
                        \k \cap ( \g_{\alpha} + \vartheta(\g_{\alpha}) ) = \sum_{i \in I_{\alpha}} \k \cap \bbC b_{\alpha}[i] \quad, \quad b_{\alpha}[i] := e_{\alpha}[i] + \vartheta( e_{\alpha}[i] )
                    \end{equation}
                Consequently, $\k$ is generated by:
                    \begin{equation} \label{equation: pseudo_fixed_generators}
                        \h^{\vartheta} \cup \{ b_{\alpha} \}_{\alpha \in \weight(\g, \h) \setminus \{0\}} \quad, \quad b_{\alpha} := e_{\alpha} + \vartheta( e_{\alpha} )
                    \end{equation}
                whose elements satisfy the following relations:
                    \begin{equation} \label{equation: pseudo_fixed_relations}
                        \begin{gathered}
                            [h, b_{\alpha}] = \alpha(h) b_{\alpha} \quad, \quad h \in \h^{\vartheta}
                            \\
                            [b_{\alpha}, b_{\beta}] = b_{\alpha + \beta} + 
                        \end{gathered}
                    \end{equation}
                \item 
            \end{enumerate}
        \end{lemma}
            \begin{proof}
                
            \end{proof}

    \subsection{Pro-algebraic groups associated to triangular-decomposable Lie algebras}
        \begin{definition}[Integrable modules] \label{def: integrable_modules_over_triangular_decomposable_lie_algebras}
            Consider a pretriangular-decomposable Lie algebra $(\g, \h, \n^{\pm}, \omega)$. A $\g$-module given by $\pi: \g \to \End(V)$ is said to be \textbf{integrable} if for any $\alpha \in \rootlattice^+(\g, \h)$ and for any corresponding weight vector $e_{\alpha} \in \g_{\alpha}$, the operator $\pi(e_{\alpha}) \in \End(V)$ is locally nilpotent, which is to say that for all $v \in V$, there exists some $N_v > 0$ such that $\pi(e_{\alpha})^n \cdot v = 0$ for all $n \geq N_v$. More succinctly, we say that $V$ is integrable if $\n^+$ acts locally nilpotently on it.
        \end{definition}
        In the Kac-Moody case, this notion reduces to the same notion by the same name, as defined in \cite[Chapter 3]{kac_infinite_dimensional_lie_algebras}.

        Given a pretriangular-decomposable Lie algebra $(\g, \h, \n^{\pm}, \omega)$, it is typical to refer to the subalgebras $\b^{\pm} := \h \oplus \n^{\pm}$ as the \textbf{positive/negative Borel subalgebras} of $\g$. More generally, Borel subalgebras of an arbitrary Lie algebra are those Lie subalgebras which are maximal amongst the solvable ones.

        \todo[inline]{Minimal pro-algebraic groups associated to triangular-decomposable Lie algebras}
        \begin{example}[Kac-Moody groups] \label{example: kac_moody_groups}
            
        \end{example}
        \begin{remark}[Conjugacy of Borel subalgebras of Kac-Moody algebras]
            By a result of Kac and Peterson from \cite{kac_peterson_infinite_flag_varieties_and_conjugacy_of_cartan_subalgebras} (cf. also \cite{chernousov_egorov_gille_pianzola_cohomological_proof_of_peterson_kac_theorem} and \cite{chernousov_neher_pianzola_conjugacy_of_cartan_subalgebras_in_EALAs_with_non_fgc_centreless_cores}), which builds upon a classical result of Borel, we know that when $\g$ either of finite or affine type, all Borel subalgebras are $\frakG$-conjugate.
        \end{remark}