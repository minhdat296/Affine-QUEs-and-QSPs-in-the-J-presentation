\documentclass[a4paper, 11pt]{article}

%\usepackage[center]{titlesec}

\usepackage{amsfonts, amssymb, amsmath, amsthm, amsxtra}

%\usepackage{foekfont}

\usepackage{MnSymbol}

\usepackage{pdfrender, xcolor}
%\pdfrender{StrokeColor=black,LineWidth=.4pt,TextRenderingMode=2}

%\usepackage{minitoc}
%\setcounter{section}{-1}
%\setcounter{tocdepth}{}
%\setcounter{minitocdepth}{}
%\setcounter{secnumdepth}{}

\usepackage{graphicx}

\usepackage[english]{babel}
\usepackage[utf8]{inputenc}
%\usepackage{mathpazo}
\usepackage{dutchcal}
%\usepackage{eucal}
\usepackage{eufrak}
\usepackage{bbm}
\usepackage{bm}
\usepackage{csquotes}
\usepackage[nottoc]{tocbibind}
\usepackage{appendix}
\usepackage{float}
\usepackage[T1]{fontenc}
\usepackage[
    left = \flqq{},% 
    right = \frqq{},% 
    leftsub = \flq{},% 
    rightsub = \frq{} %
]{dirtytalk}

\usepackage{imakeidx}
\makeindex

%\usepackage[dvipsnames]{xcolor}
\usepackage{hyperref}
    \hypersetup{
        colorlinks=true,
        linkcolor=teal,
        filecolor=pink,      
        urlcolor=teal,
        citecolor=magenta
    }
\usepackage{comment}

% You would set the PDF title, author, etc. with package options or
% \hypersetup.

\usepackage[backend=biber, style=alphabetic, sorting=nty]{biblatex}
    \addbibresource{bibliography.bib}
\renewbibmacro{in:}{}

\raggedbottom

\usepackage{mathrsfs}
\usepackage{mathtools} 
\mathtoolsset{showonlyrefs} 
%\usepackage{amsthm}
\renewcommand\qedsymbol{$\blacksquare$}

\usepackage{tikz-cd}
\tikzcdset{scale cd/.style={every label/.append style={scale=#1},
    cells={nodes={scale=#1}}}}
\usepackage{tikz}
\usepackage{setspace}
\usepackage[version=3]{mhchem}
\parskip=0.1in
\usepackage[margin=25mm]{geometry}

\usepackage{listings, lstautogobble}
\lstset{
	language=matlab,
	basicstyle=\scriptsize\ttfamily,
	commentstyle=\ttfamily\itshape\color{gray},
	stringstyle=\ttfamily,
	showstringspaces=false,
	breaklines=true,
	frameround=ffff,
	frame=single,
	rulecolor=\color{black},
	autogobble=true
}

\usepackage{todonotes,tocloft,xpatch,hyperref}

% This is based on classicthesis chapter definition
\let\oldsec=\section
\renewcommand*{\section}{\secdef{\Sec}{\SecS}}
\newcommand\SecS[1]{\oldsec*{#1}}%
\newcommand\Sec[2][]{\oldsec[\texorpdfstring{#1}{#1}]{#2}}%

\newcounter{istodo}[section]

% http://tex.stackexchange.com/a/61267/11984
\makeatletter
%\xapptocmd{\Sec}{\addtocontents{tdo}{\protect\todoline{\thesection}{#1}{}}}{}{}
\newcommand{\todoline}[1]{\@ifnextchar\Endoftdo{}{\@todoline{#1}}}
\newcommand{\@todoline}[3]{%
	\@ifnextchar\todoline{}
	{\contentsline{section}{\numberline{#1}#2}{#3}{}{}}%
}
\let\l@todo\l@subsection
\newcommand{\Endoftdo}{}

\AtEndDocument{\addtocontents{tdo}{\string\Endoftdo}}
\makeatother

\usepackage{lipsum}

%   Reduce the margin of the summary:
\def\changemargin#1#2{\list{}{\rightmargin#2\leftmargin#1}\item[]}
\let\endchangemargin=\endlist 

%   Generate the environment for the abstract:
%\newcommand\summaryname{Abstract}
%\newenvironment{abstract}%
    %{\small\begin{center}%
    %\bfseries{\summaryname} \end{center}}

\newtheorem{theorem}{Theorem}[section]
    \numberwithin{theorem}{subsection}
\newtheorem{proposition}{Proposition}[section]
    \numberwithin{proposition}{subsection}
\newtheorem{lemma}{Lemma}[section]
    \numberwithin{lemma}{subsection}
\newtheorem{claim}{Claim}[section]
    \numberwithin{claim}{subsection}
\newtheorem{question}{Question}[section]
    \numberwithin{question}{subsection}

\theoremstyle{definition}
    \newtheorem{definition}{Definition}[section]
        \numberwithin{definition}{subsection}

\theoremstyle{remark}
    \newtheorem{remark}{Remark}[section]
        \numberwithin{remark}{subsection}
    \newtheorem{example}{Example}[section]
        \numberwithin{example}{subsection}    
    \newtheorem{convention}{Convention}[section]
        \numberwithin{convention}{subsection}
    \newtheorem{corollary}{Corollary}[section]
        \numberwithin{corollary}{subsection}

\numberwithin{equation}{section}

\setcounter{section}{-1}

\renewcommand{\implies}{\Rightarrow}
\renewcommand{\cong}{\simeq}
\newcommand{\codim}{\operatorname{codim}}
\newcommand{\ladjoint}{\dashv}
\newcommand{\radjoint}{\vdash}
\newcommand{\<}{\langle}
\renewcommand{\>}{\rangle}
\newcommand\bra[2][]{#1\langle {#2} #1\rvert}
\newcommand\ket[2][]{#1\lvert {#2} #1\rangle}
\newcommand\abs[2][]{#1\left\lvert {#2} #1\right\rvert}
\newcommand\norm[2][]{#1\left\| {#2} #1\right\|}
\newcommand\order[2][]{#1\: \mathbf{:} \: {#2} #1 \: \mathbf{:} \:}
\newcommand{\ndiv}{\hspace{-2pt}\not|\hspace{5pt}}
\newcommand{\cond}{\blacktriangle}
\newcommand{\decond}{\triangle}
\newcommand{\solid}{\blacksquare}
\newcommand{\ot}{\leftarrow}
\renewcommand{\-}{\text{-}}
\renewcommand{\mapsto}{\leadsto}
\renewcommand{\leq}{\leqslant}
\renewcommand{\geq}{\geqslant}
\renewcommand{\setminus}{\smallsetminus}
\newcommand{\punc}{\overset{\circ}}
\renewcommand{\div}{\operatorname{div}}
\newcommand{\grad}{\operatorname{grad}}
\newcommand{\curl}{\operatorname{curl}}
\makeatletter
\DeclareRobustCommand{\cev}[1]{%
  {\mathpalette\do@cev{#1}}%
}
\newcommand{\do@cev}[2]{%
  \vbox{\offinterlineskip
    \sbox\z@{$\m@th#1 x$}%
    \ialign{##\cr
      \hidewidth\reflectbox{$\m@th#1\vec{}\mkern4mu$}\hidewidth\cr
      \noalign{\kern-\ht\z@}
      $\m@th#1#2$\cr
    }%
  }%
}
\makeatother

\newcommand{\N}{\mathbb{N}}
\newcommand{\Z}{\mathbb{Z}}
\newcommand{\Q}{\mathbb{Q}}
\newcommand{\R}{\mathbb{R}}
\newcommand{\bbC}{\mathbb{C}}
\newcommand{\bbK}{\mathbb{K}}
\NewDocumentCommand{\x}{e{_^}}{%
  \mathbin{\mathop{\times}\displaylimits
    \IfValueT{#1}{_{#1}}
    \IfValueT{#2}{^{#2}}
  }%
}
\NewDocumentCommand{\pushout}{e{_^}}{%
  \mathbin{\mathop{\sqcup}\displaylimits
    \IfValueT{#1}{_{#1}}
    \IfValueT{#2}{^{#2}}
  }%
}
\NewDocumentCommand{\ordprod}{e{_^}}{%
  \mathbin{\mathop{\overrightarrow{\prod}}\displaylimits
    \IfValueT{#1}{_{#1}}
    \IfValueT{#2}{^{#2}}
  }%
} %ordered product
\newcommand{\simpleroots}{\mathsf{\Delta}}
\newcommand{\rootsystem}{\mathsf{\Phi}}
\newcommand{\rootlattice}{\mathsf{Q}}
\newcommand{\weight}{\mathsf{\Pi}}
\newcommand{\weightlattice}{\mathsf{\Lambda}}
\newcommand{\Weylgroup}{\mathsf{W}}
\newcommand{\supp}{\operatorname{supp}}
\newcommand{\domain}{\operatorname{dom}}
\newcommand{\codomain}{\operatorname{codom}}
\newcommand{\im}{\operatorname{im}}
\newcommand{\coim}{\operatorname{coim}}
\newcommand{\coker}{\operatorname{coker}}
\newcommand{\id}{\mathrm{id}}
\newcommand{\chara}{\operatorname{char}}
\newcommand{\trdeg}{\operatorname{trdeg}}
\newcommand{\rank}{\operatorname{rank}}
\newcommand{\trace}{\operatorname{tr}}
\newcommand{\qdet}{\operatorname{qdet}}
\newcommand{\sklyanindet}{\operatorname{sdet}} %sklyanin determinant
\newcommand{\quasidet}{\operatorname{quasidet}}
\newcommand{\pfaff}{\operatorname{Pf}} %pfaffian
\newcommand{\qpfaff}{\operatorname{qPf}} %quantum pfaffian
\newcommand{\sklyaninpfaff}{\operatorname{sPf}} %twisted quantum pfaffian
\newcommand{\length}{\operatorname{length}}
\newcommand{\height}{\operatorname{ht}}
\renewcommand{\span}{\operatorname{span}}
\newcommand{\e}{\epsilon}
\newcommand{\p}{\mathfrak{p}}
\newcommand{\q}{\mathfrak{q}}
\newcommand{\m}{\mathfrak{m}}
\newcommand{\n}{\mathfrak{n}}
\newcommand{\calF}{\mathcal{F}}
\newcommand{\calG}{\mathcal{G}}
\newcommand{\calO}{\mathcal{O}}
\newcommand{\F}{\mathbb{F}}
\DeclareMathOperator{\lcm}{lcm}
\newcommand{\gr}{\operatorname{gr}}
\newcommand{\vol}{\mathrm{vol}}
\newcommand{\ord}{\operatorname{ord}}
\newcommand{\projdim}{\operatorname{proj.dim}}
\newcommand{\injdim}{\operatorname{inj.dim}}
\newcommand{\flatdim}{\operatorname{flat.dim}}
\newcommand{\globdim}{\operatorname{glob.dim}}
\renewcommand{\Re}{\operatorname{Re}}
\renewcommand{\Im}{\operatorname{Im}}
\newcommand{\sgn}{\operatorname{sgn}}
\newcommand{\coad}{\operatorname{coad}}
\newcommand{\ch}{\operatorname{ch}} %characters of representations
\newcommand{\dist}{\operatorname{dist}} %distance
\newcommand{\level}{\boldsymbol{c}} %level
\newcommand{\dualcoxeter}{\boldsymbol{h}^{\vee}} %dual coxeter number

\newcommand{\Ad}{\mathrm{Ad}}
\newcommand{\GL}{\mathrm{GL}}
\newcommand{\SL}{\mathrm{SL}}
\newcommand{\PGL}{\mathrm{PGL}}
\newcommand{\PSL}{\mathrm{PSL}}
\newcommand{\Sp}{\mathrm{Sp}}
\newcommand{\GSp}{\mathrm{GSp}}
\newcommand{\GSpin}{\mathrm{GSpin}}
\newcommand{\rmO}{\mathrm{O}}
\newcommand{\SO}{\mathrm{SO}}
\newcommand{\SU}{\mathrm{SU}}
\newcommand{\rmU}{\mathrm{U}}
\newcommand{\rmH}{\mathrm{H}}
\newcommand{\rmY}{\mathrm{Y}} %yangian
\newcommand{\rmX}{\mathrm{X}} %extended yangian
\newcommand{\rmu}{\mathrm{u}}
\newcommand{\rmV}{\mathrm{V}}
\newcommand{\gl}{\mathfrak{gl}}
\renewcommand{\sl}{\mathfrak{sl}}
\newcommand{\diag}{\mathfrak{diag}}
\newcommand{\pgl}{\mathfrak{pgl}}
\newcommand{\psl}{\mathfrak{psl}}
\newcommand{\fraksp}{\mathfrak{sp}}
\newcommand{\gsp}{\mathfrak{gsp}}
\newcommand{\gspin}{\mathfrak{gspin}}
\newcommand{\frako}{\mathfrak{o}}
\newcommand{\so}{\mathfrak{so}}
\newcommand{\su}{\mathfrak{su}}
\newcommand{\Spec}{\operatorname{Spec}}
\newcommand{\Spf}{\operatorname{Spf}}
\newcommand{\Spm}{\operatorname{Spm}}
\newcommand{\Spv}{\operatorname{Spv}}
\newcommand{\Spa}{\operatorname{Spa}}
\newcommand{\Spd}{\operatorname{Spd}}
\newcommand{\Proj}{\operatorname{Proj}}
\newcommand{\Gr}{\mathrm{Gr}}
\newcommand{\Hecke}{\mathrm{Hecke}}
\newcommand{\Sht}{\mathrm{Sht}}
\newcommand{\Quot}{\mathrm{Quot}}
\newcommand{\Hilb}{\mathrm{Hilb}}
\newcommand{\Pic}{\mathrm{Pic}}
\newcommand{\Div}{\mathrm{Div}}
\newcommand{\Jac}{\mathrm{Jac}}
\newcommand{\Alb}{\mathrm{Alb}} %albanese variety
\newcommand{\Bun}{\mathrm{Bun}}
\newcommand{\loopspace}{\boldsymbol{\Omega}}
\newcommand{\suspension}{\boldsymbol{\Sigma}}
\newcommand{\tangent}{\mathrm{T}} %tangent space
\newcommand{\Eig}{\mathrm{Eig}}
\newcommand{\Cox}{\mathrm{Cox}} %coxeter functors
\newcommand{\rmK}{\mathrm{K}} %Killing form
\newcommand{\km}{\mathfrak{km}} %kac-moody algebras
\newcommand{\Dyn}{\mathrm{Dyn}} %associated Dynkin quivers
\newcommand{\Car}{\mathrm{Car}} %cartan matrices of finite quivers
\newcommand{\uce}{\mathfrak{uce}} %universal central extension of lie algebras

\newcommand{\Ring}{\mathrm{Ring}}
\newcommand{\Cring}{\mathrm{CRing}}
\newcommand{\Bool}{\mathrm{Bool}} %boolean algebras
\newcommand{\Alg}{\mathrm{Alg}}
\newcommand{\Leib}{\mathrm{Leib}} %leibniz algebras
\newcommand{\Fld}{\mathrm{Fld}}
\newcommand{\Sets}{\mathrm{Sets}}
\newcommand{\Equiv}{\mathrm{Equiv}} %equivalence relations
\newcommand{\Cat}{\mathrm{Cat}}
\newcommand{\Grp}{\mathrm{Grp}}
\newcommand{\Ab}{\mathrm{Ab}}
\newcommand{\Sch}{\mathrm{Sch}}
\newcommand{\Coh}{\mathrm{Coh}}
\newcommand{\QCoh}{\mathrm{QCoh}}
\newcommand{\Perf}{\mathrm{Perf}} %perfect complexes
\newcommand{\Sing}{\mathrm{Sing}} %singularity categories
\newcommand{\Desc}{\mathrm{Desc}}
\newcommand{\Sh}{\mathrm{Sh}}
\newcommand{\Psh}{\mathrm{PSh}}
\newcommand{\Fib}{\mathrm{Fib}}
\renewcommand{\mod}{\-\mathrm{mod}}
\newcommand{\comod}{\-\mathrm{comod}}
\newcommand{\bimod}{\-\mathrm{bimod}}
\newcommand{\Vect}{\mathrm{Vect}}
\newcommand{\Rep}{\mathrm{Rep}}
\newcommand{\Grpd}{\mathrm{Grpd}}
\newcommand{\Arr}{\mathrm{Arr}}
\newcommand{\Esp}{\mathrm{Esp}}
\newcommand{\Ob}{\mathrm{Ob}}
\newcommand{\Mor}{\mathrm{Mor}}
\newcommand{\Mfd}{\mathrm{Mfd}}
\newcommand{\Riem}{\mathrm{Riem}}
\newcommand{\RS}{\mathrm{RS}}
\newcommand{\LRS}{\mathrm{LRS}}
\newcommand{\TRS}{\mathrm{TRS}}
\newcommand{\TLRS}{\mathrm{TLRS}}
\newcommand{\LVRS}{\mathrm{LVRS}}
\newcommand{\LBRS}{\mathrm{LBRS}}
\newcommand{\Spc}{\mathrm{Spc}}
\newcommand{\Top}{\mathrm{Top}}
\newcommand{\Topos}{\mathrm{Topos}}
\newcommand{\Nil}{\operatorname{Nil}}
\newcommand{\Rad}{\operatorname{Rad}}
\newcommand{\Stk}{\mathrm{Stk}}
\newcommand{\Pre}{\mathrm{Pre}}
\newcommand{\simp}{\mathbf{\Delta}}
\newcommand{\Res}{\mathrm{Res}}
\newcommand{\Ind}{\mathrm{Ind}}
\newcommand{\Pro}{\mathrm{Pro}}
\newcommand{\Mon}{\mathrm{Mon}}
\newcommand{\Comm}{\mathrm{Comm}}
\newcommand{\Fin}{\mathrm{Fin}}
\newcommand{\Assoc}{\mathrm{Assoc}}
\newcommand{\Semi}{\mathrm{Semi}}
\newcommand{\Co}{\mathrm{Co}}
\newcommand{\Loc}{\mathrm{Loc}}
\newcommand{\Ringed}{\mathrm{Ringed}}
\newcommand{\Haus}{\mathrm{Haus}} %hausdorff spaces
\newcommand{\Comp}{\mathrm{Comp}} %compact hausdorff spaces
\newcommand{\Stone}{\mathrm{Stone}} %stone spaces
\newcommand{\Extr}{\mathrm{Extr}} %extremely disconnected spaces
\newcommand{\Ouv}{\mathrm{Ouv}}
\newcommand{\Str}{\mathrm{Str}}
\newcommand{\Func}{\mathrm{Func}}
\newcommand{\Crys}{\mathrm{Crys}}
\newcommand{\LocSys}{\mathrm{LocSys}}
\newcommand{\Sieves}{\mathrm{Sieves}}
\newcommand{\pt}{\mathrm{pt}}
\newcommand{\Graphs}{\mathrm{Graphs}}
\newcommand{\Lie}{\mathrm{Lie}}
\newcommand{\Env}{\mathrm{Env}}
\newcommand{\Ho}{\mathrm{Ho}}
\newcommand{\rmD}{\mathrm{D}}
\newcommand{\Cov}{\mathrm{Cov}}
\newcommand{\Frames}{\mathrm{Frames}}
\newcommand{\Locales}{\mathrm{Locales}}
\newcommand{\Span}{\mathrm{Span}}
\newcommand{\Corr}{\mathrm{Corr}}
\newcommand{\Monad}{\mathrm{Monad}}
\newcommand{\Var}{\mathrm{Var}}
\newcommand{\sfN}{\mathrm{N}} %nerve
\newcommand{\Diam}{\mathrm{Diam}} %diamonds
\newcommand{\co}{\mathrm{co}}
\newcommand{\ev}{\mathrm{ev}}
\newcommand{\LA}{\mathrm{LA}} %category of lie algebras
\newcommand{\LBA}{\mathrm{LBA}} %category of lie bialgebras
\newcommand{\bi}{\mathrm{bi}} %bi-, for bialgebras, bimodules, etc.
\newcommand{\Hopf}{\mathrm{Hopf}}
\newcommand{\QUE}{\mathrm{QUE}} %quantised enveloping algebras, i.e. hopf algebras over C[[h]] that are cocommutative mod h
\newcommand{\Nat}{\mathrm{Nat}}
\newcommand{\Dmod}{\mathrm{D}\mod}
\newcommand{\Perv}{\mathrm{Perv}}
\newcommand{\Sph}{\mathrm{Sph}}
\newcommand{\Moduli}{\mathrm{Moduli}}
\newcommand{\Pseudo}{\mathrm{Pseudo}}
\newcommand{\Lax}{\mathrm{Lax}}
\newcommand{\Strict}{\mathrm{Strict}}
\newcommand{\Opd}{\mathrm{Opd}} %operads
\newcommand{\Shv}{\mathrm{Shv}}
\newcommand{\Char}{\mathrm{Char}} %CharShv = character sheaves
\newcommand{\Huber}{\mathrm{Huber}}
\newcommand{\Tate}{\mathrm{Tate}}
\newcommand{\Affd}{\mathrm{Affd}} %affinoid algebras
\newcommand{\Adic}{\mathrm{Adic}} %adic spaces
\newcommand{\Rig}{\mathrm{Rig}}
\newcommand{\An}{\mathrm{An}}
\newcommand{\Perfd}{\mathrm{Perfd}} %perfectoid spaces
\newcommand{\Sub}{\mathrm{Sub}} %subobjects
\newcommand{\Ideals}{\mathrm{Ideals}}
\newcommand{\Isoc}{\mathrm{Isoc}} %isocrystals
\newcommand{\Ban}{\mathrm{Ban}} %Banach spaces
\newcommand{\Fre}{\mathrm{Fr\acute{e}}} %Frechet spaces
\newcommand{\Ch}{\mathrm{Ch}} %chain complexes
\newcommand{\Pure}{\mathrm{Pure}}
\newcommand{\Mixed}{\mathrm{Mixed}}
\newcommand{\Hodge}{\mathrm{Hodge}} %Hodge structures
\newcommand{\Mot}{\mathrm{Mot}} %motives
\newcommand{\KL}{\mathrm{KL}} %category of Kazhdan-Lusztig modules
\newcommand{\Pres}{\mathrm{Pr}} %presentable categories
\newcommand{\Noohi}{\mathrm{Noohi}} %category of Noohi groups
\newcommand{\Inf}{\mathrm{Inf}}
\newcommand{\LPar}{\mathrm{LPar}} %Langlands parameters
\newcommand{\ORig}{\mathrm{ORig}} %overconvergent sites
\newcommand{\Quiv}{\mathrm{Quiv}} %quivers
\newcommand{\Def}{\mathrm{Def}} %deformation functors
\newcommand{\Higgs}{\mathrm{Higgs}}
\newcommand{\BGG}{\mathrm{BGG}}
\newcommand{\Poiss}{\mathrm{Poiss}}
\newcommand{\Fact}{\mathrm{Fact}} %factorisation
\newcommand{\Chr}{\mathrm{Chr}} %chiral
\newcommand{\Smooth}{\mathrm{Sm}}

\newcommand{\Aut}{\operatorname{Aut}}
\newcommand{\Inv}{\operatorname{Inv}}
\newcommand{\PseudoInv}{\widetilde{\Inv}}
\newcommand{\Inn}{\operatorname{Inn}}
\newcommand{\Out}{\operatorname{Out}}
\newcommand{\der}{\mathfrak{der}} %derivations on Lie algebras
\newcommand{\frakend}{\mathfrak{end}}
\newcommand{\aut}{\mathfrak{aut}}
\newcommand{\inn}{\mathfrak{inn}} %inner derivations
\newcommand{\out}{\mathfrak{out}} %outer derivations
\newcommand{\Stab}{\operatorname{Stab}}
\newcommand{\Cent}{\operatorname{Cent}}
\newcommand{\Norm}{\operatorname{Norm}}
\newcommand{\Core}{\operatorname{Core}}
\newcommand{\cent}{\mathfrak{cent}}
\newcommand{\core}{\mathfrak{core}}
\newcommand{\Transporter}{\operatorname{Transp}} %transporter between two subsets of a group
\newcommand{\Conj}{\operatorname{Conj}}
\newcommand{\Diag}{\operatorname{Diag}}
\newcommand{\Gal}{\operatorname{Gal}}
\newcommand{\bfG}{\mathbf{G}} %absolute Galois group
\newcommand{\Frac}{\operatorname{Frac}}
\newcommand{\Ann}{\operatorname{Ann}}
\newcommand{\Val}{\operatorname{Val}}
\newcommand{\Chow}{\operatorname{Chow}}
\newcommand{\Sym}{\operatorname{Sym}}
\newcommand{\Alt}{\operatorname{Alt}}
\newcommand{\End}{\operatorname{End}}
\newcommand{\Mat}{\operatorname{Mat}}
\newcommand{\Diff}{\operatorname{Diff}}
\newcommand{\Autom}{\operatorname{Autom}}
\newcommand{\Artin}{\operatorname{Artin}} %artin maps
\newcommand{\sk}{\operatorname{sk}} %skeleton of a category
\newcommand{\eqv}{\operatorname{eqv}} %functor that maps groups $G$ to $G$-sets
\newcommand{\Inert}{\operatorname{Inert}}
\newcommand{\Fil}{\operatorname{Fil}}
\newcommand{\prim}{\mathfrak{prim}}
\newcommand{\Nerve}{\operatorname{N}}
\newcommand{\Hol}{\operatorname{Hol}} %holomorphic functions %holonomy groups
\newcommand{\Bi}{\operatorname{Bi}} %Bi for biholomorphic functions
\newcommand{\chev}{\operatorname{chev}}
\newcommand{\bfLie}{\mathbf{Lie}} %non-reduced lie algebra associated to generalised cartan matrices
\newcommand{\frakLie}{\mathfrak{Lie}} %reduced lie algebra associated to generalised cartan matrices
\newcommand{\frakChev}{\mathfrak{Chev}} 
\newcommand{\Rees}{\operatorname{Rees}}
\newcommand{\Dr}{\operatorname{Dr}} %Drinfeld's quantum double 
\newcommand{\frakDr}{\mathfrak{Dr}} %classical double of lie bialgebras
\newcommand{\Tot}{\operatorname{Tot}} %total complexes

\renewcommand{\projlim}{\varprojlim}
\newcommand{\indlim}{\varinjlim}
%\newcommand{\colim}{\operatorname{colim}}
%\renewcommand{\lim}{\operatorname{lim}}
\newcommand{\toto}{\rightrightarrows}
%\newcommand{\tensor}{\otimes}

\NewDocumentCommand{\tensor}{e{_^}}{%
  \mathbin{\mathop{\otimes}\displaylimits
    \IfValueT{#1}{_{#1}}
    \IfValueT{#2}{^{#2}}
  }%
}
\NewDocumentCommand{\singtensor}{e{_^}}{%
  \mathbin{\mathop{\odot}\displaylimits
    \IfValueT{#1}{_{#1}}
    \IfValueT{#2}{^{#2}}
  }%
}
\NewDocumentCommand{\hattensor}{e{_^}}{%
  \mathbin{\mathop{\hat{\otimes}}\displaylimits
    \IfValueT{#1}{_{#1}}
    \IfValueT{#2}{^{#2}}
  }%
}
\NewDocumentCommand{\brevetensor}{e{_^}}{%
  \mathbin{\mathop{\breve{\otimes}}\displaylimits
    \IfValueT{#1}{_{#1}}
    \IfValueT{#2}{^{#2}}
  }%
}
\NewDocumentCommand{\semidirect}{e{_^}}{%
  \mathbin{\mathop{\rtimes}\displaylimits
    \IfValueT{#1}{_{#1}}
    \IfValueT{#2}{^{#2}}
  }%
}
\newcommand{\eq}{\operatorname{eq}}
\newcommand{\coeq}{\operatorname{coeq}}
\newcommand{\Hom}{\operatorname{Hom}}
\newcommand{\Bil}{\operatorname{Bil}} %bilinear maps
\newcommand{\Maps}{\operatorname{Maps}}
\newcommand{\Tor}{\operatorname{Tor}}
\newcommand{\Ext}{\operatorname{Ext}}
\newcommand{\Isom}{\operatorname{Isom}}
\newcommand{\stalk}{\operatorname{stalk}}
\newcommand{\RKE}{\operatorname{RKE}}
\newcommand{\LKE}{\operatorname{LKE}}
\newcommand{\oblv}{\operatorname{oblv}}
\newcommand{\const}{\operatorname{const}}
\newcommand{\free}{\operatorname{free}}
\newcommand{\adrep}{\operatorname{ad}} %adjoint representation
\newcommand{\NL}{\mathbb{NL}} %naive cotangent complex
\newcommand{\pr}{\operatorname{pr}}
\newcommand{\Der}{\operatorname{Der}}
\newcommand{\Frob}{\operatorname{Fr}} %Frobenius
\newcommand{\frob}{\operatorname{f}} %trace of Frobenius
\newcommand{\bfpt}{\mathbf{pt}}
\newcommand{\bfloc}{\mathbf{loc}}
\DeclareMathAlphabet{\mymathbb}{U}{BOONDOX-ds}{m}{n}
\newcommand{\0}{\boldsymbol{ 0 }}
\newcommand{\1}{\boldsymbol{ 1 }}
\newcommand{\2}{\boldsymbol{ 2 }}
\newcommand{\Jet}{\operatorname{Jet}}
\newcommand{\Split}{\mathrm{Split}}
\newcommand{\Sq}{\mathrm{Sq}}
\newcommand{\Zero}{\mathrm{Z}}
\newcommand{\SqZ}{\Sq\Zero}
\newcommand{\lie}{\mathfrak{lie}}
\newcommand{\y}{\operatorname{y}} %yoneda
\newcommand{\Sm}{\mathrm{Sm}}
\newcommand{\AJ}{\phi} %abel-jacobi map
\newcommand{\act}{\mathrm{act}}
\newcommand{\ram}{\mathrm{ram}} %ramification index
\newcommand{\inv}{\mathrm{inv}}
\newcommand{\Spr}{\mathrm{Spr}} %the Springer map/sheaf
\newcommand{\Refl}{\mathrm{Refl}} %reflection functor]
\newcommand{\HH}{\mathrm{HH}} %Hochschild (co)homology
\newcommand{\HC}{\mathrm{HC}} %cyclic (co)homology
\newcommand{\Poinc}{\mathrm{Poinc}}
\newcommand{\Simpson}{\mathrm{Simpson}}
\newcommand{\Section}{\operatorname{Sect}}
\newcommand{\Ran}{\operatorname{Ran}} %Ran space
\newcommand{\quantumfields}{\operatorname{QFld}}

\newcommand{\bbU}{\mathbb{U}}
\newcommand{\bbV}{\mathbb{V}}
\newcommand{\W}{\mathbb{W}}
\newcommand{\calU}{\mathcal{U}}
\newcommand{\calu}{\mathcal{u}}
\newcommand{\calW}{\mathcal{W}}
\newcommand{\rmI}{\mathrm{I}} %augmentation ideal
\newcommand{\bfV}{\mathbf{V}}
\newcommand{\C}{\mathcal{C}}
\newcommand{\D}{\mathcal{D}}
\newcommand{\scrD}{\mathscr{D}}
\newcommand{\T}{\mathscr{T}} %Tate modules
\newcommand{\calM}{\mathcal{M}}
\newcommand{\calN}{\mathcal{N}}
\newcommand{\calP}{\mathcal{P}}
\newcommand{\calQ}{\mathcal{Q}}
\newcommand{\A}{\mathbb{A}}
\renewcommand{\P}{\mathbb{P}}
\newcommand{\calL}{\mathcal{L}}
\newcommand{\scrL}{\mathscr{L}}
\newcommand{\E}{\mathcal{E}}
\renewcommand{\H}{\mathbf{H}}
\newcommand{\scrS}{\mathscr{S}}
\newcommand{\calX}{\mathcal{X}}
\newcommand{\calY}{\mathcal{Y}}
\newcommand{\caly}{\mathcal{y}}
\newcommand{\calZ}{\mathcal{Z}}
\newcommand{\calS}{\mathcal{S}}
\newcommand{\calR}{\mathcal{R}}
\newcommand{\calr}{\mathcal{r}}
\newcommand{\calK}{\mathcal{K}}
\newcommand{\calk}{\mathcal{k}}
\newcommand{\scrX}{\mathscr{X}}
\newcommand{\scrY}{\mathscr{Y}}
\newcommand{\scrZ}{\mathscr{Z}}
\newcommand{\calA}{\mathcal{A}}
\newcommand{\calB}{\mathcal{B}}
\renewcommand{\S}{\mathcal{S}}
\newcommand{\B}{\mathbb{B}}
\newcommand{\bbD}{\mathbb{D}}
\newcommand{\G}{\mathbb{G}}
\newcommand{\horn}{\mathbf{\Lambda}}
\renewcommand{\L}{\mathbb{L}}
\renewcommand{\a}{\mathfrak{a}}
\renewcommand{\b}{\mathfrak{b}}
\renewcommand{\c}{\mathfrak{c}}
\renewcommand{\d}{\mathfrak{d}}
\renewcommand{\t}{\mathfrak{t}}
\renewcommand{\r}{\mathfrak{r}}
\renewcommand{\o}{\mathfrak{o}}
\renewcommand{\sp}{\mathfrak{sp}}
\newcommand{\fraku}{\mathfrak{u}}
\newcommand{\frakl}{\mathfrak{l}}
\newcommand{\fraky}{\mathfrak{y}}
\newcommand{\frakv}{\mathfrak{v}}
\newcommand{\frakw}{\mathfrak{w}}
\newcommand{\frake}{\mathfrak{e}}
\newcommand{\bbX}{\mathbb{X}}
\newcommand{\frakG}{\mathfrak{G}}
\newcommand{\frakH}{\mathfrak{H}}
\newcommand{\frakE}{\mathfrak{E}}
\newcommand{\frakF}{\mathfrak{F}}
\newcommand{\g}{\mathfrak{g}}
\newcommand{\h}{\mathfrak{h}}
\renewcommand{\k}{\mathfrak{k}}
\newcommand{\z}{\mathfrak{z}}
\newcommand{\fraki}{\mathfrak{i}}
\newcommand{\frakj}{\mathfrak{j}}
\newcommand{\del}{\partial}
\newcommand{\bbE}{\mathbb{E}}
\newcommand{\scrO}{\mathscr{O}}
\newcommand{\bbO}{\mathbb{O}}
\newcommand{\scrA}{\mathscr{A}}
\newcommand{\scrB}{\mathscr{B}}
\newcommand{\scrE}{\mathscr{E}}
\newcommand{\scrF}{\mathscr{F}}
\newcommand{\scrG}{\mathscr{G}}
\newcommand{\scrM}{\mathscr{M}}
\newcommand{\scrN}{\mathscr{N}}
\newcommand{\scrP}{\mathscr{P}}
\newcommand{\frakS}{\mathfrak{S}}
\newcommand{\frakT}{\mathfrak{T}}
\newcommand{\calI}{\mathcal{I}}
\newcommand{\calJ}{\mathcal{J}}
\newcommand{\scrI}{\mathscr{I}}
\newcommand{\scrJ}{\mathscr{J}}
\newcommand{\scrH}{\mathscr{H}}
\newcommand{\calH}{\mathcal{H}}
\newcommand{\scrK}{\mathscr{K}}
\newcommand{\scrV}{\mathscr{V}}
\newcommand{\scrW}{\mathscr{W}}
\newcommand{\bbS}{\mathbb{S}}
\newcommand{\bfA}{\mathbf{A}}
\newcommand{\bfB}{\mathbf{B}}
\newcommand{\bfC}{\mathbf{C}}
\renewcommand{\O}{\mathbb{O}}
\newcommand{\calV}{\mathcal{V}}
\newcommand{\scrR}{\mathscr{R}} %radical
\newcommand{\sfR}{\mathsf{R}} %quantum R-matrices
\newcommand{\sfr}{\mathsf{r}} %classical R-matrices
\newcommand{\rmZ}{\mathrm{Z}} %centre of algebra
\newcommand{\rmC}{\mathrm{C}} %centralisers in algebras
\newcommand{\bfGamma}{\mathbf{\Gamma}}
\newcommand{\scrU}{\mathscr{U}}
\newcommand{\rmW}{\mathrm{W}} %Weil group
\newcommand{\frakM}{\mathfrak{M}}
\newcommand{\frakN}{\mathfrak{N}}
\newcommand{\frakB}{\mathfrak{B}}
\newcommand{\frakX}{\mathfrak{X}}
\newcommand{\frakY}{\mathfrak{Y}}
\newcommand{\frakZ}{\mathfrak{Z}}
\newcommand{\frakU}{\mathfrak{U}}
\newcommand{\frakR}{\mathfrak{R}}
\newcommand{\frakP}{\mathfrak{P}}
\newcommand{\frakQ}{\mathfrak{Q}}
\newcommand{\sfX}{\mathsf{X}}
\newcommand{\sfY}{\mathsf{Y}}
\newcommand{\sfZ}{\mathsf{Z}}
\newcommand{\sfS}{\mathsf{S}}
\newcommand{\sfT}{\mathsf{T}}
\newcommand{\sfOmega}{\mathsf{\Omega}} %drinfeld p-adic upper-half plane
\newcommand{\rmA}{\mathrm{A}}
\newcommand{\rmB}{\mathrm{B}}
\newcommand{\calT}{\mathcal{T}}
\newcommand{\sfA}{\mathsf{A}}
\newcommand{\sfB}{\mathsf{B}}
\newcommand{\sfC}{\mathsf{C}}
\newcommand{\sfD}{\mathsf{D}}
\newcommand{\sfE}{\mathsf{E}}
\newcommand{\sfF}{\mathsf{F}}
\newcommand{\sfG}{\mathsf{G}}
\newcommand{\sfh}{\mathsf{h}} %coxeter number
\newcommand{\frakL}{\mathfrak{L}}
\newcommand{\K}{\mathrm{K}}
\newcommand{\rmT}{\mathrm{T}}
\newcommand{\bfv}{\mathbf{v}}
\newcommand{\bfg}{\mathbf{g}}
\newcommand{\frakV}{\mathfrak{V}}
\newcommand{\bfn}{\mathbf{n}}
\newcommand{\Alpha}{\mathrm{A}}
\newcommand{\Beta}{\mathrm{B}}

%special lie modules
\newcommand{\standard}{\boldsymbol{M}}
\newcommand{\simple}{\boldsymbol{L}}
\newcommand{\vacuum}{ \mathcal{Vac} }
\newcommand{\weyl}{\mathcal{W}}
\newcommand{\boson}{ \mathcal{Bos} }
\newcommand{\fermion}{ \mathcal{Fer} }
\newcommand{\KR}{ \mathcal{KR} }

\newcommand{\frakC}{\mathfrak{C}}
\newcommand{\frakD}{\mathfrak{D}}
\newcommand{\rmi}{\mathrm{i}}
\newcommand{\bfH}{\mathbf{H}}
\newcommand{\bfX}{\mathbf{X}}
\newcommand{\coxeter}{\mathrm{h}}

\newcommand{\aff}{\mathrm{aff}}
\newcommand{\ft}{\mathrm{ft}} %finite type
\newcommand{\fp}{\mathrm{fp}} %finite presentation
\newcommand{\fr}{\mathrm{fr}} %free
\newcommand{\tft}{\mathrm{tft}} %topologically finite type
\newcommand{\tfp}{\mathrm{tfp}} %topologically finite presentation
\newcommand{\tfr}{\mathrm{tfr}} %topologically free
\newcommand{\aft}{\mathrm{aft}}
\newcommand{\lft}{\mathrm{lft}}
\newcommand{\laft}{\mathrm{laft}}
\newcommand{\cpt}{\mathrm{cpt}}
\newcommand{\cproj}{\mathrm{cproj}}
\newcommand{\qc}{\mathrm{qc}}
\newcommand{\qs}{\mathrm{qs}}
\newcommand{\lcmpt}{\mathrm{lcmpt}}
\newcommand{\red}{\mathrm{red}}
\newcommand{\fin}{\mathrm{fin}}
\newcommand{\fd}{\mathrm{fd}} %finite-dimensional
\newcommand{\gen}{\mathrm{gen}}
\newcommand{\petit}{\mathrm{petit}}
\newcommand{\gros}{\mathrm{gros}}
\newcommand{\loc}{\mathrm{loc}}
\newcommand{\glob}{\mathrm{glob}}
%\newcommand{\ringed}{\mathrm{ringed}}
%\newcommand{\qcoh}{\mathrm{qcoh}}
\newcommand{\cl}{\mathrm{cl}}
\newcommand{\et}{\mathrm{\acute{e}t}}
\newcommand{\fet}{\mathrm{f\acute{e}t}}
\newcommand{\profet}{\mathrm{prof\acute{e}t}}
\newcommand{\proet}{\mathrm{pro\acute{e}t}}
\newcommand{\Zar}{\mathrm{Zar}}
\newcommand{\fppf}{\mathrm{fppf}}
\newcommand{\fpqc}{\mathrm{fpqc}}
\newcommand{\orig}{\mathrm{orig}} %overconvergent topology
\newcommand{\smooth}{\mathrm{sm}}
\newcommand{\sh}{\mathrm{sh}}
\newcommand{\op}{\mathrm{op}}
\newcommand{\cop}{\mathrm{cop}}
\newcommand{\open}{\mathrm{open}}
\newcommand{\closed}{\mathrm{closed}}
\newcommand{\geom}{\mathrm{geom}}
\newcommand{\alg}{\mathrm{alg}}
\newcommand{\sober}{\mathrm{sober}}
\newcommand{\dR}{\mathrm{dR}}
\newcommand{\rad}{\mathfrak{rad}}
\newcommand{\discrete}{\mathrm{discrete}}
%\newcommand{\add}{\mathrm{add}}
%\newcommand{\lin}{\mathrm{lin}}
\newcommand{\Krull}{\mathrm{Krull}}
\newcommand{\qis}{\mathrm{qis}} %quasi-isomorphism
\newcommand{\ho}{\mathrm{ho}} %homotopy equivalence
\newcommand{\sep}{\mathrm{sep}}
\newcommand{\insep}{\mathrm{insep}}
\newcommand{\unr}{\mathrm{unr}}
\newcommand{\tame}{\mathrm{tame}}
\newcommand{\wild}{\mathrm{wild}}
\newcommand{\nil}{\mathrm{nil}}
\newcommand{\defm}{\mathrm{defm}}
\newcommand{\Art}{\mathrm{Art}}
\newcommand{\Noeth}{\mathrm{Noeth}}
\newcommand{\affd}{\mathrm{affd}}
%\newcommand{\adic}{\mathrm{adic}}
\newcommand{\pre}{\mathrm{pre}}
\newcommand{\coperf}{\mathrm{coperf}}
\newcommand{\perf}{\mathrm{perf}}
\newcommand{\perfd}{\mathrm{perfd}}
\newcommand{\rat}{\mathrm{rat}}
\newcommand{\cont}{\mathrm{cont}}
\newcommand{\dg}{\mathrm{dg}}
\newcommand{\almost}{\mathrm{a}}
%\newcommand{\stab}{\mathrm{stab}}
\newcommand{\heart}{\heartsuit}
\newcommand{\proj}{\mathrm{proj}}
\newcommand{\rot}{\mathrm{rot}}
\newcommand{\qproj}{\mathrm{qproj}} %quasi-projective
\newcommand{\pd}{\mathrm{pd}}
\newcommand{\crys}{\mathrm{crys}}
\newcommand{\prisma}{\mathrm{prisma}}
\newcommand{\FF}{\mathrm{FF}}
\newcommand{\sph}{\mathrm{sph}}
\newcommand{\lax}{\mathrm{lax}}
\newcommand{\weak}{\mathrm{weak}}
\newcommand{\strict}{\mathrm{strict}}
\newcommand{\mon}{\mathrm{mon}}
\newcommand{\sym}{\mathrm{sym}}
\newcommand{\lisse}{\mathrm{lisse}}
\newcommand{\an}{\mathrm{an}}
\newcommand{\ad}{\mathrm{ad}}
\newcommand{\sch}{\mathrm{sch}}
\newcommand{\rig}{\mathrm{rig}}
\newcommand{\pol}{\mathrm{pol}}
\newcommand{\plat}{\mathrm{flat}}
\newcommand{\proper}{\mathrm{proper}}
\newcommand{\compl}{\mathrm{compl}}
\newcommand{\non}{\mathrm{non}}
\newcommand{\access}{\mathrm{access}}
\newcommand{\comp}{\mathrm{comp}}
\newcommand{\tstructure}{\mathrm{t}} %t-structures
\newcommand{\pure}{\mathrm{pure}} %pure motives
\newcommand{\mixed}{\mathrm{mixed}} %mixed motives
\newcommand{\num}{\mathrm{num}} %numerical motives
\newcommand{\ess}{\mathrm{ess}}
\newcommand{\topological}{\mathrm{top}}
\newcommand{\convex}{\mathrm{cvx}}
\newcommand{\locconvex}{\mathrm{lcvx}}
\newcommand{\ab}{\mathrm{ab}} %abelian extensions
\newcommand{\inj}{\mathrm{inj}}
\newcommand{\surj}{\mathrm{surj}} %coverage on sets generated by surjections
\newcommand{\eff}{\mathrm{eff}} %effective Cartier divisors
\newcommand{\Weil}{\mathrm{Weil}} %weil divisors
\newcommand{\lex}{\mathrm{lex}}
\newcommand{\rex}{\mathrm{rex}}
\newcommand{\AR}{\mathrm{A\-R}}
\newcommand{\cons}{\mathrm{c}} %constructible sheaves
\newcommand{\tor}{\mathrm{tor}} %tor dimension
\newcommand{\connected}{\mathrm{connected}}
\newcommand{\cg}{\mathrm{cg}} %compactly generated
\newcommand{\nilp}{\mathrm{nilp}}
\newcommand{\isg}{\mathrm{isg}} %isogenous
\newcommand{\qisg}{\mathrm{qisg}} %quasi-isogenous
\newcommand{\irr}{\mathrm{irr}} %irreducible represenations
\newcommand{\indecomp}{\mathrm{indecomp}}
\newcommand{\preproj}{\mathrm{preproj}}
\newcommand{\preinj}{\mathrm{preinj}}
\newcommand{\reg}{\mathrm{reg}}
\newcommand{\sing}{\mathrm{sing}}
\newcommand{\crit}{\mathrm{crit}}
\newcommand{\semisimple}{\mathrm{ss}}
\newcommand{\integrable}{\mathrm{int}}
\newcommand{\s}{\mathfrak{s}}
\newcommand{\stab}{\mathrm{stab}}
\newcommand{\disj}{\mathrm{disj}}
\newcommand{\positive}{\mathrm{pos}} 
\newcommand{\negative}{\mathrm{neg}} 
\newcommand{\up}{\mathrm{up}} %upper
\newcommand{\low}{\mathrm{low}} %lower
\newcommand{\locallyconstant}{\mathrm{lconst}}
\newcommand{\complete}{\mathrm{compl}}
\newcommand{\chiral}{\mathrm{ch}}
\newcommand{\fact}{\mathrm{fact}}
\newcommand{\tw}{\mathrm{tw}} %twisted
\newcommand{\ext}{\mathrm{ext}} %extended
\newcommand{\elliptic}{\mathrm{ell}}
\newcommand{\trigonometric}{\mathrm{trig}}
\newcommand{\rational}{\mathrm{rat}}
\newcommand{\trianglular}{\mathrm{tr}} %triangular
\newcommand{\quasitriangular}{\mathrm{qtr}} %quasi-triangular

%prism custom command
\usepackage{relsize}
\usepackage[bbgreekl]{mathbbol}
\usepackage{amsfonts}
\DeclareSymbolFontAlphabet{\mathbb}{AMSb} %to ensure that the meaning of \mathbb does not change
\DeclareSymbolFontAlphabet{\mathbbl}{bbold}
\newcommand{\prism}{{\mathlarger{\mathbbl{\Delta}}}}

%symmetric pairs; twisted yangians and reflection algebras
\newcommand{\romanzero}{\mathsf{0}}
\newcommand{\romanone}{\mathsf{I}}
\newcommand{\romantwo}{\mathsf{II}}
\newcommand{\romanthree}{\mathsf{III}}
\newcommand{\Aone}{\sfA\romanone}
\newcommand{\BCD}{\sfB, \sfC, \sfD}
\newcommand{\BCDzero}{\sfB\sfC\sfD\romanzero}
\newcommand{\BCDone}{\sfB\sfC\sfD\romanone}
\newcommand{\XB}{\mathcal{XB}} %reflection equation
\newcommand{\UXB}{\mathcal{UXB}} %reflection equation and unitary condition
\newcommand{\UB}{\mathcal{UB}} %reflection equation, unitary condition, and symmetry relation
\newcommand{\ZB}{\mathcal{ZB}} 
\newcommand{\DB}{\mathcal{DB}} %doubled reflection algebra 
\newcommand{\DY}{\mathcal{DY}} %double yangian
\newcommand{\dy}{\mathcal{dy}} %classical limit of double yangian
\newcommand{\DX}{\mathcal{DX}} %extended double yangian
\newcommand{\dx}{\mathcal{dx}} %classical limit of extended double yangian
\newcommand{\CYBE}{\operatorname{CYBE}} %classical yang-baxter equation
\newcommand{\QYBE}{\operatorname{QYBE}} %quantum yang-baxter equation

\newcommand{\Loop}{\mathrm{L}}

\renewcommand{\i}{\imath}

%superalgebras
\newcommand{\even}{\bar{0}}
\newcommand{\odd}{\bar{1}}
\newcommand{\super}{\mathrm{s}}

%schur weyl duality
\newcommand{\heckealgebra}{\mathsf{He}}
\newcommand{\degenerateheckealgebra}{\overline{\heckealgebra}}
\newcommand{\braueralgebra}{\mathsf{Br}}
\newcommand{\walledbraueralgebra}{\overline{\braueralgebra}}
\newcommand{\VW}{\mathsf{VW}}

\newcommand{\Heis}{\mathcal{H}}

\newcommand{\KM}{\mathrm{KM}} %kac-moody presentation/drinfeld-jimbo presentation
\newcommand{\current}{\mathrm{Dr}} %drinfeld current presentation

\newcommand{\freeboson}{\boldsymbol{\phi}}
\newcommand{\freefermion}{\boldsymbol{\psi}}
\newcommand{\DF}{\boldsymbol{\Phi}_{\calR \to \current}} %ding-frenkel isomorphism
\newcommand{\Beck}{\boldsymbol{\Phi}_{\current \to \KM}} %ding-frenkel isomorphism

\begin{document}

	\title{Infinite-dimensional twisted Manin triples}
	
	\author{Dat Minh Ha}
	\maketitle
	
	\begin{abstract}
	    We generalise the results from \cite{belliard_crampe_coideal_subalgebras_from_twisted_manin_triples} to the cases wherein Manin triples are twisted by the \say{pseudo-involutions}, in the sense of \cite{regelskis_vlaar_kac_moody_pseudo_symmetric_pairs}. This serves as a first step in formulating quantum symmetric pairs in terms similar to Drinfeld's J-presentation for Yangian, following the work \cite{belliard_regelskis_J_presentation_for_twisted_yangians} which established twisted Yangians in such a presentation.
	\end{abstract}
	
	{
      \hypersetup{} 
      %\dominitoc
      \tableofcontents %sort sections alphabetically
    }

    \section{Introduction}
    \subsection{Notations}

    \subsection{Context}

    \subsection{Overview}

    \section{Triangular-decomposable Lie algebras and their automorphisms}
    \subsection{Weighted Lie algebras and (pre)triangular-decomposable Lie algebras}
        Ideally, we would like to develop a theory of Manin triples $(\a, \a^+, \a^-)$ that are somehow twisted by general Lie algebra automorphisms $\vartheta \in \Aut_{\Lie\Alg}(\a)$, but at present, we are unfortunately at a loss for how to write down a description of such twisted Manin triples. Guided by the theory of quantum symmetric pairs (QSPs) that arise from so-called \say{pseudo-involutions} on affine Kac-Moody algebras (see e.g. \cite{regelskis_vlaar_reflection_matrices_coideal_subalgebras}, \cite{kolb_kac_moody_QSPs}, and references therein), we have chosen to restrict our consideration down to the class of Lie algebras that admit so-called \say{triangular decompositions}. The structure and representation theory of this class of Lie algebras has been studied extensively, notably in \cite{moody_pianzola_lie_algebras_with_triangular_decompositions}, and thanks to the fact that they admit triangular decompositions, one can speak of Borel subalgebras; the automorphisms that we shall consider shall be the ones that either stabilise or permute pairs of Borel subalgebras.
        
        Let us begin by recalling the construction of (pre)triangular-decomposable\footnote{In \textit{loc. cit.}, such Lie algebras are called \say{Lie algebras admitting triangular decompositions}, but we find the terminology somewhat cumbersome.} Lie algebras from \cite[Section 2.1]{moody_pianzola_lie_algebras_with_triangular_decompositions}. Their construction starts with what we shall refer to as their \say{weighing subalgebras}: this is a pair:
            $$(\h, V)$$
        consisting of a non-zero abelian Lie algebra $\h \not = 0$ and a choice of a representation $\h \to \gl(V)$. Then, for any linear functional $\lambda \in \h^*$, a \say{weight vector} of weight $\lambda$ shall be an element $v \in V$ that is a simultaneous eigenvector of all the scalars in the set $\{ \lambda(h) \}_{h \in \h}$, i.e.:
            $$h \cdot v = \lambda(h) v \quad, \quad h \in \h$$
        For any $\lambda \in \h^*$, the vector subspace of $V$ spanned by weight-$\lambda$ vectors, i.e.:
            $$V_{\lambda} := \{ v \in V \mid \forall h \in \h: h \cdot v = \lambda(h) v \}$$
        is called the \say{subspace of weight $\lambda$}. If $V = \sum_{\lambda \in \h^*} V_{\lambda}$ as vector subspaces of $V$, then we shall say that $V$ admits a \say{weight space decomposition} with respect to $\h$; \textit{a priori}, this sum is necessarily direct if it exists, i.e.:
            $$V = \bigoplus_{\lambda \in \h^*} V_{\lambda}$$
        (see \cite[Section 2.1, Proposition 1]{moody_pianzola_lie_algebras_with_triangular_decompositions}). In order to avoid redundancy, let us refer to only the functionals $\lambda \in \h^*$ such that $V_{\lambda} \not = 0$ as \say{weights} of $V$, and the subset of $\h^*$ consisting of weights of $V$ is denoted by $\weight(V, \h)$.

        Next, consider another Lie algebra $\g$, on which our weighing algebra $\h$ acts via derivations, i.e. there exists a Lie algebra homomorphism $\h \to \der(\g)$, such that there exists a weight space decomposition:
            $$\g = \bigoplus_{\alpha \in \h^*} \g_{\alpha}$$
        It is then possible to show that:
            $$[\g_{\alpha}, \g_{\beta}] = \g_{\alpha + \beta} \quad, \quad \alpha, \beta \in \h^*$$
        and consequently, the Lie algebra $\g$ is graded by the abelian group $\rootlattice(\g, \h) := \Z\weight(\g, \h)$. Conversely, it is also possible to show that if there exists a set of weights $P \subset \weight(\g, \h)$ and a set of weight vectors $\{ e_{\alpha} \}_{\alpha \in P}$ such that $\g$ is generated as a Lie algebra by $\h \cup \{ e_{\alpha} \}_{\alpha \in P}$, then $\g = \bigoplus_{\alpha \in \h^*} \g_{\alpha}$ wherein $e_{\alpha} \in \g_{\alpha}$ for all $\alpha \in P$. For details on these facts, see \cite[Subsection 2.1, Proposition 2]{moody_pianzola_lie_algebras_with_triangular_decompositions}. Therefore, it makes sense to refer to a Lie algebra $\g$ generated by weight vectors and together with a weighing algebra $\h$ in the manner above as a Lie algebra \say{weighted} by $\h$. For compactness, let us denote such a \textbf{weighted Lie algebra} as a pair:
            $$(\g, \h)$$
        \begin{remark}
            For a Lie algebra $\g$ weighted by $\h$, we have that $\g_0 \cong \h$ if $\h$ acts by inner derivations.
        \end{remark}

        \begin{definition}[(Pre)triangular-decomposable Lie algebras] \label{def: (pre)triangular_decomposable_lie_algebras}
            A \textbf{pretriangular-decomposable Lie algebra} is a weighted Lie algebra $(\g, \h)$ satisfying the following additional conditions.
            \begin{itemize}
                \item There exists non-zero Lie subalgebras $\n^{\pm} \subset \g$ such that $\g \cong \n^- \oplus \h \oplus \n^+$ as Lie algebras.
                \item There exists an involution $\omega: \g \xrightarrow[]{\cong} -\g$ (commonly called the \textbf{Cartan involution}) such that $\omega(\n^{\pm}) = -\n^{\mp}$ and $\omega|_{\h} = \id_{\h}$.
                \item $\n^{\pm}$ are $\h$-submodules of $\g$, i.e. $[\h, \n^{\pm}] \subseteq \n^{\pm}$. The Lie subalgebras $\n^{\pm}$ thus admit the induced weight space decompositions $\n^{\pm} = \bigoplus_{\alpha \in \weight(\g, \h)} \n^{\pm} \cap \g_{\alpha}$, and let us require furthermore that:
                    $$\rootlattice^{\pm}(\g) := \{ \alpha \in \weight(\g, \h) \setminus \{0\} \mid \n^{\pm} \cap \g_{\alpha} \not = 0 \}$$
                are free (additive) sub-semigroups of $\rootlattice(\g, \h) := \Z\weight(\g, \h)$.
                \item Finally, we require that there is a semi-group basis\footnote{... which exists because we are pre-supposing that $\rootlattice^+(\g, \h)$ is free.} $\{\alpha_i\}_{i \in \simpleroots} \subset \rootlattice^+(\g, \h)$.
            \end{itemize}
            A pretriangular-decomposable Lie algebra as above shall be denoted as a quintuple:
                $$(\g, \h, \n^+, \n^-, \omega)$$
            (though we shall usually abbreviate the signs and write $(\g, \h, \n^{\pm}, \omega)$ instead).

            If the subalgebras $\n^{\pm} \subset \g$ are nilpotent, then $(\g, \h, \n^{\pm}, \omega)$ will be called a \textbf{triangular-decomposable} Lie algebra.
        \end{definition}
        \begin{remark}
            Note that by requiring that $0 \not \in \rootlattice^{\pm}(\g)$, we have that:
                $$\rootlattice^-(\g, \h) \cap \rootlattice^+(\g, \h) = \varnothing$$
                
            Also, we should point out that the \say{upper/lower-triangular} subalgebras $\n^{\pm} \subset \g$ of a general pretriangular-decomposable Lie algebra $(\g, \h, \n^{\pm}, \omega)$ are \textit{not} necessarily even locally nilpotent, let alone nilpotent! In more combinatorial terms, in this general of a setting, it is entirely possible for the so-called \say{root strings} to be infinitely long. This is the reason for singling out the triangular-decomposable Lie algebras from the larger class of pretriangular-decomposable ones.

            Lastly, there is no guarantee in general that the weight spaces $\g_{\alpha} \subset \g$ are finite-dimensional. Pretriangular-decomposable Lie algebras with this property are said to be \textbf{regular}. Note that this property needs not imply that the Lie subalgebras $\n^{\pm}$ are nilpotent: for instance, extended Kac-Moody algebras (given by \eqref{equation: extended_kac_moody_relations}; cf. also \cite[Theorem 1.2]{kac_infinite_dimensional_lie_algebras}) are regular in the above sense, but because their generators are not subjected to the Serre relations, those algebras are not triangular-decomposable, only pretriangular-decomposable. 
        \end{remark}
        \begin{example}
            By their very construction (e.g. as explained in \cite[Theorem 1.2]{kac_infinite_dimensional_lie_algebras}), Kac-Moody algebras are pretriangular-decomposable. Via the Serre relations, one sees also that symmetrisable Kac-Moody algebras are actually triangular-decomposable.
            
            Finite-dimensional but non-Kac-Moody examples of pretriangular-decomposable Lie algebras include reductive Lie algebras, while infinite-dimensional non-Kac-Moody examples include Heisenberg algebras and Virasoro algebras, extended affine Lie algebras (EALAs) in the sense of \cite{neher_lectures_on_EALAs}, as well as the higher-nullity analogues of the Heisenberg and Virasoro algebras that arise via EALAs. Triangular-decomposable Lie algebras form a very large class.

            On the other hand, solvable Lie algebras for instance - and thus also the Lie algebras that are nilpotent, and especially the abelian ones - are \textit{not} pretriangular decomposable.
        \end{example}

    \subsection{Pseudo-involutions}
        \begin{definition}[Weighted automorphisms and pseudo-involutions] \label{def: weighted_automorphisms_of_weighted_lie_algebras}
            Consider a weighted Lie algebra $(\g, \h)$. Next, consider a Lie algebra automorphism $\vartheta \in \Aut_{\Lie\Alg}(\g)$ under which $\h$ is stable, i.e. it is such that $\vartheta|_{\hbar} \in \Aut_{\Lie\Alg}(\h)$. In such a situation, we say that this is an automorphism of $\g$ \textbf{weighted} by $\h$.
            
            A weighted automorphism $\vartheta \in \Aut(\g, \h)$ that is an involution, i.e. $\vartheta^2 = \id_{\g}$, is said to be a \textbf{weighted involution}. If we only have that $\vartheta|_{\h}^2 = \id_{\h}$, then $\vartheta$ will be called a \textbf{pseudo-involution}. 
        \end{definition}
        It is clear that weight automorphisms, pseudo-involutions, weighted involutions, and involutions of a given weighted Lie algebra $\g$ form subgroups of $\Aut_{\Lie\Alg}(\g)$. We denote these subgroups, respectively, by:
            $$\Aut(\g, \h) \quad \supset \quad \PseudoInv(\g, \h) \quad \supset \quad \Inv(\g, \h) \quad \subset \quad \Inv_{\Lie\Alg}(\g)$$
        and we note that $\Inv(\g, \h) := \Inv_{\Lie\Alg}(\g) \cap \Aut(\g, \h)$ just by definition.
        \begin{remark}
            The notion of pseudo-involution above coincides with \cite[Definition 1.1]{regelskis_vlaar_kac_moody_pseudo_symmetric_pairs}. Additionally, note that the Cartan involution $\omega \in \Aut(\g, \h)$ of a given pretriangular-decomposable Lie algebra $(\g, \h, \n^{\pm}, \omega)$ is in fact a weighted involution by definition, and the notion below is given \textit{relatively} to this choice of an involution on $\g$.
        \end{remark}

        Now, recall that any involutive endomorphism $\vartheta \in \End(V)$ on a finite-dimensional vector space $V$ has eigenvalues $\pm 1$ (and hence is automatically an automorphism too)\footnote{As long as the characteristic of the ground field is not $2$, but this is not a concern for us.}, and hence induces the following eigenspace decomposition:
            $$V = V_{\vartheta[1]} \oplus V_{\vartheta[-1]}$$
        Through this, we infer that $\a$ is a finite-dimensional Lie algebra, $\vartheta \in \Inv_{\Lie\Alg}(\a)$ , and $V := \a$ is the adjoint $\a$-module, then we can decompose the Lie algebra $\a$ in the following manner:
            \begin{equation} \label{equation: symmetric_space_decomposition}
                \a = \a_{\vartheta[1]} \oplus \a_{\vartheta[-1]}
            \end{equation}
        This is commonly called the \say{Cartan decomposition} or the \say{symmetric pair decomposition}, and we note that:
            $$\a_{\vartheta[1]} = \a^{\vartheta}$$
        i.e. the eigenspace of weight $1$ coincides - as a Lie subalgebra of $\a$ - with the fixed-point Lie subalgebra $\a^{\vartheta} := \{x \in \a \mid \vartheta(x) = x\}$.

        Non-involutive Lie algebra automorphisms, even pseudo-involutive ones, may have eigenvalues other than $\pm 1$, so a decomposition in the form \eqref{equation: symmetric_space_decomposition} is not generally available for such a Lie algebra automorphism. In order to circumvent this difficulty, the following generalisation of the notion of fixed-point subalgebras was proposed in \cite[Definition 1.2]{regelskis_vlaar_kac_moody_pseudo_symmetric_pairs}. 
        \begin{definition}[Pseudo-fixed-points subalgebras] \label{def: pseudo_fixed_point_subalgebras}
            Let $(\g, \h)$ be a weighted Lie algebra, in which the weighing subalgebra $\h$ is finite-dimensional, and fix some pseudo-involution $\vartheta \in \PseudoInv(\g, \h)$. A \textbf{pseudo-fixed-point subalgebra} of $\g$ is then a Lie subalgebra $\k$ such that:
            \begin{itemize}
                \item $\k \cap \h = \h^{\vartheta}$, and
                \item $\dim\left( \k \cap ( \g_{\alpha} + \vartheta(\g_{\alpha}) ) \right) = \dim \g_{\alpha}$ for each weight $\alpha \in \weight(\g, \h)$.
            \end{itemize}
        \end{definition}
        \begin{remark}
            To be strict about technicalities, we must point out that our definition \eqref{def: pseudo_fixed_point_subalgebras} and \cite[Definition 1.2]{regelskis_vlaar_kac_moody_pseudo_symmetric_pairs} do not coincide in full generality, though they do tend to in practice. For them, $\h$ needs to be maximal amongst the Lie subalgebras of $\g$ that act semi-simply via the adjoint representation. For us, in contrast, $\h$ merely has to act semi-simply via derivations on $\g$, which is a much weaker requirement. We would also like to remark that definition \ref{def: pseudo_fixed_point_subalgebras} (as well as \cite[Definition 1.2]{regelskis_vlaar_kac_moody_pseudo_symmetric_pairs}) makes no mention of a possibility of uniqueness for pseudo-fixed-point subalgebras.
        \end{remark}

        Now, given a weighted Lie algebra $(\g, \h)$, a pseudo-involution $(\g, \h) \in \PseudoInv(\g, \h)$ thereon, and a $\vartheta$-pseudo-fixed-point subalgebra $\k \subset \g$, let us attempt to compute the elements in:
            $$[\k, \k] \quad, \quad [\k, \g]$$
        At the level of generality described above, we do not believe that much can be said, but if we restrict our attention to the pretriangular-decomposable Lie algebras amongst the weighted ones, then the problem becomes more tractable. As such, let us fix a pretriangular-decomposable Lie algebra $(\g, \h, \n^{\pm}, \omega)$, and to be clear, we recall that by definition, such a Lie algebra admits a weight space decomposition:
            $$\g = \h \oplus \bigoplus_{\alpha \in \weight(\g, \h) \setminus \{0\}} \g_{\alpha}$$
        via the semi-simple action via derivations of $\h$ on $\g$ (see definition \ref{def: (pre)triangular_decomposable_lie_algebras}). We recall also that as a part of the definition, $\g$ is generated the weighing subalgebra $\h$ along with weight vectors $e_{\alpha} \in \g_{\alpha}$, which is what shall allow us to compute the elements in $[\k, \k]$ and in $[\k, \g]$ (again, see definition \ref{def: (pre)triangular_decomposable_lie_algebras}).
        \begin{lemma}[Adjoint actions of pseudo-fixed-point subalgebras] \label{lemma: adjoint_actions_of_pseudo_fixed_point_subalgebras}
            Consider a pretriangular-decomposable Lie algebra $(\g, \h, \n^{\pm}, \omega)$, generated by $\h \cup \{e_{\alpha}\}_{\alpha \in \rootlattice(\g, \h)}$ wherein $e_{\alpha} \in \g_{\alpha}$ are weight vectors, and consider a pseudo-involution $\vartheta \in \PseudoInv(\g, \h)$ along with a pseudo-fixed-point subalgebra $\k \subset \g$. Additionally, for each weight $\alpha \in \weight(\g, \h)$, suppose that:
                $$\dim \g_{\alpha} = |I_{\alpha}|$$
            for some set $I_{\alpha}$, and let us fix once and for all a basis:
                $$\{e_{\alpha}[i]\}_{i \in I_{\alpha}} \subset \g_{\alpha}$$
            \begin{enumerate}
                \item For each weight $\alpha \in \weight(\g, \h)$, we have:
                    \begin{equation}
                        \k \cap ( \g_{\alpha} + \vartheta(\g_{\alpha}) ) = \sum_{i \in I_{\alpha}} \k \cap \bbC b_{\alpha}[i] \quad, \quad b_{\alpha}[i] := e_{\alpha}[i] + \vartheta( e_{\alpha}[i] )
                    \end{equation}
                Consequently, $\k$ is generated by:
                    \begin{equation} \label{equation: pseudo_fixed_generators}
                        \h^{\vartheta} \cup \{ b_{\alpha} \}_{\alpha \in \weight(\g, \h) \setminus \{0\}} \quad, \quad b_{\alpha} := e_{\alpha} + \vartheta( e_{\alpha} )
                    \end{equation}
                whose elements satisfy the following relations:
                    \begin{equation} \label{equation: pseudo_fixed_relations}
                        \begin{gathered}
                            [h, b_{\alpha}] = \alpha(h) b_{\alpha} \quad, \quad h \in \h^{\vartheta}
                            \\
                            [b_{\alpha}, b_{\beta}] = b_{\alpha + \beta} + 
                        \end{gathered}
                    \end{equation}
                \item 
            \end{enumerate}
        \end{lemma}
            \begin{proof}
                
            \end{proof}

    \subsection{Pro-algebraic groups associated to triangular-decomposable Lie algebras}
        \begin{definition}[Integrable modules] \label{def: integrable_modules_over_triangular_decomposable_lie_algebras}
            Consider a pretriangular-decomposable Lie algebra $(\g, \h, \n^{\pm}, \omega)$. A $\g$-module given by $\pi: \g \to \End(V)$ is said to be \textbf{integrable} if for any $\alpha \in \rootlattice^+(\g, \h)$ and for any corresponding weight vector $e_{\alpha} \in \g_{\alpha}$, the operator $\pi(e_{\alpha}) \in \End(V)$ is locally nilpotent, which is to say that for all $v \in V$, there exists some $N_v > 0$ such that $\pi(e_{\alpha})^n \cdot v = 0$ for all $n \geq N_v$. More succinctly, we say that $V$ is integrable if $\n^+$ acts locally nilpotently on it.
        \end{definition}
        In the Kac-Moody case, this notion reduces to the same notion by the same name, as defined in \cite[Chapter 3]{kac_infinite_dimensional_lie_algebras}.

        Given a pretriangular-decomposable Lie algebra $(\g, \h, \n^{\pm}, \omega)$, it is typical to refer to the subalgebras $\b^{\pm} := \h \oplus \n^{\pm}$ as the \textbf{positive/negative Borel subalgebras} of $\g$. More generally, Borel subalgebras of an arbitrary Lie algebra are those Lie subalgebras which are maximal amongst the solvable ones.

        \todo[inline]{Minimal pro-algebraic groups associated to triangular-decomposable Lie algebras}
        \begin{example}[Kac-Moody groups] \label{example: kac_moody_groups}
            
        \end{example}
        \begin{remark}[Conjugacy of Borel subalgebras of Kac-Moody algebras]
            By a result of Kac and Peterson from \cite{kac_peterson_infinite_flag_varieties_and_conjugacy_of_cartan_subalgebras} (cf. also \cite{chernousov_egorov_gille_pianzola_cohomological_proof_of_peterson_kac_theorem} and \cite{chernousov_neher_pianzola_conjugacy_of_cartan_subalgebras_in_EALAs_with_non_fgc_centreless_cores}), which builds upon a classical result of Borel, we know that when $\g$ either of finite or affine type, all Borel subalgebras are $\frakG$-conjugate.
        \end{remark}

    \section{Twisted Manin triples and quantisation of Lie coideals}
    \subsection{Twisted Manin triples and Lie coideals}
        For a recollection of basic facts about Manin triples, we refer the reader to subsection \ref{subsection: manin_triples}. We freely employ the notations therein.
    
        \begin{definition}[Twisted Manin triple] \label{def: twisted_manin_triples}
            Suppose that $(\g, \h)$ is a weighted Lie algebra, that $(\g, \g^+, \g^-)$ is a Manin triple, and consider some pseudo-involution\footnote{In principle, definition \ref{def: twisted_manin_triples} can be written down without the automorphism $\vartheta$ being pseudo-involutive, but such scenarios are beyond the scope of our purposes.} $\vartheta \in \PseudoInv(\g, \h)$. Also, denote the bilinear form associated to the Manin triple above by $(\cdot, \cdot)_{\g}$.
            \begin{itemize}
                \item Said pseudo-involution $\vartheta$ is then said to \textbf{twist} the aforementioned Manin triple if either $\vartheta(\g^{\pm}) = \g^{\pm}$ or $\vartheta(\g^{\pm}) = -\g^{\mp}$.
                \item The $\vartheta$-twisting above is said to be \textbf{invariant} (respectively, \textbf{anti-invariant}) if $( \vartheta(x), y )_{\g} = ( x, \vartheta(y) )_{\g}$ (respectively, if $( \vartheta(x), y )_{\g} = -( x, \vartheta(y) )_{\g}$).
            \end{itemize}
        \end{definition}

        \begin{theorem}[Lie bialgebras and coideals from invariantly twisted Manin triples] \label{theorem: lie_bialgebras_from_invariantly_twisted_manin_triples}
            Suppose that $(\g, \h, \n^{\pm}, \omega)$ is a root-decomposable Lie algebra, that $(\g, \g^+, \g^-)$ is a Manin triple twisted by some pseudo-involution $\vartheta \in \PseudoInv(\g, \h)$; also, let us write $\delta^{\pm}$ for the Lie bialgebra structures on $\g^{\pm}$. Next, let:
                $$\k$$
            be the $\vartheta$-pseudo-fixed-point subalgebra of $\g$ constructed in proposition \ref{prop: constructing_pseudo_fixed_point_subalgebras}, and let:
                $$\k^{\pm} := \k \cap \g^{\pm}$$
            \begin{enumerate}
                \item There is a topological Lie bialgebra structure $\delta_{\k^+}: \k^+ \to \k^+ \hattensor \k^+$, given by $\delta_{\k^+} := [\cdot, \cdot]_{\k^-}^*$, if and only if the $\vartheta$-twisting is \textit{invariant}. This is \textit{not} a Lie sub-bialgebra of $(\g^+, \delta^+)$.
                \item The restriction $\delta_{\k^+} := \delta^+|_{\k^+}$ is a Lie right-coideal structure $\delta_{\k^+}: \k^+ \to \k^+ \hattensor \g^+$ if and only if the $\vartheta$-twisting is \textit{anti-invariant}.
            \end{enumerate}
        \end{theorem}
        \begin{remark}
            Before we begin the proof, let us remark that we are forced to work with a root-decomposable Lie algebra instead of a general weighted one because so that we can make use of the pseudo-fixed-point subalgebra $\vartheta$ from proposition \ref{prop: constructing_pseudo_fixed_point_subalgebras}. It would be interesting to know if this assumption can be relaxed, or indeed, even done away with entirely.
        \end{remark}
            \begin{proof}
                Let $\g^{\pm} := \k^{\pm} \oplus \m^{\pm}$ be a splitting of vector spaces.
                \begin{enumerate}
                    \item 
                    \item 
                \end{enumerate}
            \end{proof}

        \begin{lemma}[Root-decomposable Lie bialgebras] \label{lemma: root_decomposable_lie_bialgebras}
            Suppose that $(\g, \h, \n^{\pm}, \omega)$ is a root-decomposable Lie algebra, and consider the Borel subalgebras $\b^{\pm} := \h \oplus \n^{\mp}$. Next, let us form the Lie algebra:
                $$\a := \b^+ \oplus \b^-$$
            into which $\b^{\pm}$ embed by means of the maps $\eta^{\pm}: \b^{\pm} \hookrightarrow \a$ given by:
                \begin{equation} \label{equation: root_decomposable_borel_subalgebra_embeddings}
                    \eta^{\pm}(x) := x \oplus ( \pm x_{\h} ) \quad, \quad x \in \b^{\pm}
                \end{equation}
            (cf. equation \eqref{equation: borel_lie_sub_bialgebra_embeddings}), wherein $x_{\h}$ is the image of $x \in \b^{\pm}$ under the canonical quotient map $\b^{\pm} \to \b^{\pm}/\n^{\pm} \cong \h$. Moreover, this larger Lie algebra shall be equipped with the symmetric, non-degenerate, and invariant bilinear form given by:
                \begin{equation} \label{equation: extended_root_decomposable_pairings}
                    (\cdot, \cdot)_{\a} := (\cdot, \cdot)_{\g} - (\cdot, \cdot)_{\h}
                \end{equation}
            Then, with respect to the bilinear form above, we have that:
                \begin{equation} \label{equation: root_decomposable_manin_triples}
                    (\a, \eta^+(\b^+), \eta^-(\b^-))
                \end{equation}
            is a Manin triple.
        \end{lemma}
            \begin{proof}
                First of all, via lemma \ref{lemma: direct_sums_of_weighted_lie_algebras}, we see that $\a$ is weighted by the abelian Lie subalgebra $\eta^+(\h) \oplus \eta^-(\h)$. Next, from remark \ref{remark: isotropic_subalgebras_of_root_decomopsable_lie_algebras}, it is clear that the Lie subalgebras $\eta^{\pm}(\b^{\pm}) \subset \a$ are isotropic with respect to $(\cdot, \cdot)_{\a}$. The rest follows from the construction of the triple $(\a, \eta^+(\b^+), \eta^-(\b^-))$.
            \end{proof}
        \begin{proposition}[Twisted Manin triples of types I and II] \label{prop: twisted_manin_triples_of_types_I_and_II}
            Suppose that $(\g, \h, \n^{\pm}, \omega)$ is a symmetrisable Kac-Moody algebra, and consider a pseudo-involution $\vartheta \in \PseudoInv(\g, \h)$. Suppose also that the Manin triple \eqref{equation: root_decomposable_manin_triples} is twisted by $\vartheta$, invariantly or anti-invariantly. Then:
            \begin{enumerate}
                \item $\vartheta(\n^{\pm}) = \n^{\pm}$ if and only if $\vartheta$ is of type I.
                \item $\vartheta(\n^{\pm}) = -\n^{\mp}$ if and only if $\vartheta$ is of type II.
            \end{enumerate}
        \end{proposition}
            \begin{proof}
                First of all, by definition \ref{def: weighted_automorphisms_of_weighted_lie_algebras}, we know that $\vartheta|_{\h} \in \Inv_{\Lie\Alg}(\h)$. The map $\vartheta|_{\h}$ thus extends to an involution on the weighing subalgebra $\eta^+(\h) \oplus \eta^-(\h) \subset \a$, and hence a pseudo-involution on $\a$, which we denote by the same symbol $\vartheta$.
            
                Let $\frakG$ be the minimal pro-algebraic group associated to $(\g, \h, \n^{\pm}, \omega)$, which exists because this Lie algebra is integrable. Since $(\cdot, \cdot)_{\g}$ is $\g$-invariant, it is also $\frakG$-invariant, i.e. $(\Ad(g) \cdot x, y)_{\g} = (x, \Ad(g) \cdot y)$ for all $x, y \in \g$ and all $g \in \frakG$. This means that for all $g \in \frakG$, we have in the $\vartheta$-invariant case that $( \Ad(g) \cdot \vartheta(x), y )_{\g} = ( x, \Ad(g) \cdot \vartheta(y) )_{\g}$ as a result of the fact that $( \vartheta(x), y )_{\g} = ( x, \vartheta(y) )_{\g}$ (cf. definition \ref{def: twisted_manin_triples}), and similarly in the anti-invariant case. By definition, $\vartheta$ is of type II if and only if we can choose $g \in \frakG$ such that $\omega = \Ad(g) \cdot \vartheta$.  As the bilinear form $(\cdot, \cdot)_{\g}$ is $\omega$-invariant by definition \ref{def: root_decomposable_lie_algebras}, and as $\omega(\n^{\pm}) = -\n^{\mp}$ by definition \ref{def: triangular_decomposable_lie_algebras}, we are done.
            \end{proof}

    \subsection{Deformation quantisation}

    \subsection{Twisted classical r-matrices, classical k-matrices, and their quantisations}

    \section{Examples of twisted Kac-Moody Manin triples and their quantisations}
    \subsection{(Quantum) affine Kac-Moody Lie bialgebras and their twisted forms}

    \subsection{\texorpdfstring{Classical and $q$-Onsager algebras}{}}
        \todo[inline]{Onsager Lie bialgebras arise as twists of the standard Kac-Moody Lie bialgebra structure by the Chevalley involution. These twists are necessarily anti-invariant according to theorem \ref{theorem: twisted_forms_and_coideal_subalgebras_of_kac_moody_lie_bialgebras}.}

    \subsection{\texorpdfstring{Type-II twisted affine algebras and $q$-Yangians with centres}{}}
        \todo[inline]{Twist affine Kac-Moody Manin triples by type-II pseudo-involutions $\vartheta$. Such twists are necessarily anti-invariant, by theorem \ref{theorem: twisted_forms_and_coideal_subalgebras_of_kac_moody_lie_bialgebras}, so the resulting pseudo-fixed-point subalgebras $\k$ are Lie coideals of the standard (affine) Kac-Moody Lie bialgebra structure. Let us denote the standard Kac-Moody Lie cobracket by:
            $$\delta: \g \to \g \hattensor \g$$
        It is known that the restriction $\delta|_{\g'}$ is a Lie sub-bialgebra structure on the derivation-less affine Lie algebras, which coincides with the derived subalgebra $\g'$. The pseudo-fixed-point subalgebras $\k' = \k \cap \g'$ thus have structures of Lie coideals of $(\g', \delta|_{\g'})$. One of Lie coideal structures should agree with the classical limits of the twisted $q$-Yangians with centres, say $\calU_q(\g', \vartheta)$, from \cite{molev_ragoucy_sorba_twisted_q_yangians_type_A}, and this is to be expected, as these twisted $q$-Yangians with centres are coideal subalgebras of the quantum affine algebra $\calU_q(\g')$. The centre-less twisted $q$-Yangians $\calU_q(\Loop \bar{\g}, \vartheta)$ is obtained as the quotient of $\calU_q(\g', \vartheta)$ by the ideal generated by $\sklyanindet = 1$, so their classical limits should be the image of $\k'$ under the canonical quotient map:
            $$\g' \to \Loop \bar{\g}$$
        obtained by identifying $\Loop \bar{\g} \cong \g'/\bbC \level$. Note that $\Loop \bar{\g}$ is well-defined as a quotient Lie bialgebra induced by $\delta|_{\g'}$, and this is because $\delta(\level) = 0$, and so $\bbC \level \subset \g'$ is a Lie bi-ideal.
        \\
        Onsager algebras are special cases of classical limits of twisted $q$-Yangians.}

    \subsection{Outlook: quantum (pseudo-)symmetric pairs of twisted affine types}

    \begin{appendices}
        \section{Manin triples and quantisation of Lie bialgebras} 
            \subsection{Manin triples} \label{subsection: manin_triples}
                Let us recall the following definition from \cite[Subsection 2.6]{appel_laredo_2_categorical_etingof_kazhdan_quantisation} (see also \cite[Subsection 7.4]{etingof_kazhdan_quantisation_1}).
\begin{definition}[Manin triples] \label{def: manin_triples}
    A \textbf{Manin triple} is a triple of Lie algebras $(\a, \a^+, \a^-)$ such that $\a = \a^- \oplus \a^+$, along with a non-degenerate invariant pairing $(\cdot, \cdot) \in \Hom( \Sym^2(\a)^{\a}, \bbC )$, which are to satisfy the following conditions.
    \begin{itemize}
        \item With respect to $(\cdot, \cdot)$, the Lie subalgebras $\a^{\pm} \subset \a$ are to be isotropic. 
        \item The non-degenerate pairing $(\cdot, \cdot)$ induces an isomorphism of topological vector spaces $\a^- \xrightarrow[]{\cong} (\a^+)^*$, with $\a^+$ equipped with the discrete topology and $(\a^+)^*$ equipped with the weak topology.
        \item The commutator on $\a = \a^- \oplus \a^+$ is continuous with respect to the topologies chosen above. 
    \end{itemize}
\end{definition}
Now, if $(\a, \a^+, \a^-)$ is a Manin triple, then we can construct a Lie bialgebra structure on $\a = \a^+ \oplus \a^-$ in the following manner. Let:
    $$\calr \in \a^+ \tensor \a^- \subset \a \tensor \a$$
be the canonical element, corresponding to $\id_{\a}$ via the non-degenerate pairing on $\a$. Then, one can check that the following construction defines a topological Lie cobracket $\delta: \a \to \a \hattensor \a$:
    \begin{equation}
        \delta(x) := [\Box(x), \calr] \quad, \quad x \in \a
    \end{equation}
wherein $\Box(x) := x \tensor 1 + 1 \tensor x$. This is compatible with the Lie bracket on $\a$ in such a manner that $\a$ is a Lie bialgebra. Also, it can be shown that:
    $$\pm \delta^{\pm} := \delta|_{\a^{\pm}}: \a^{\pm} \to \a^{\pm} \tensor \a^{\pm}$$
are well-defined Lie sub-bialgebra structures, which are such that each of the Lie cobrackets $\delta^{\pm}$ is dual to the Lie bracket on $\a^{\pm}$, respectively. Additionally, $(\a, \delta)$ is isomorphic to Drinfeld's classical double of either $(\a^{\pm}, \delta^{\pm})$ (more on this shortly); indeed, these doubles are isomorphic to one another.
\begin{remark}
    It can be checked that $\calr$ satisfies the \say{classical Yang-Baxter equation}:
        \begin{equation} \label{equation: CYBE}
            [\calr_{1, 2}, \calr_{1, 3}] + [\calr_{1, 2}, \calr_{2, 3}] + [\calr_{1, 3}, \calr_{2, 3}] = 0
        \end{equation}
    and for this reason, Lie bialgebras that arise in this manner are known as being \say{quasi-triangular}.
\end{remark}

Conversely, given a topological Lie bialgebra structure $\delta^+: \a^+ \to \a^+ \hattensor \a^+$, one can construct a Manin triple $(\a, \a^+, \a^-)$ with $\a^- := (\a^+)^*, \a := \a^+ \oplus \a^-$, and the non-degenerate and invariant pairing on $\a$ is the canonical one between $\a^+$ and $\a^-$, given by:
    $$(x, \varphi) := \varphi(x) \quad, \quad x \in \a^+, \varphi \in \a^-$$
Moreover, $\a^-$ automatically carries an induced Lie bialgebra structure $\delta^-$ given by dualising the (continuous) Lie bracket on $\a^+$; there is thus also a Lie bialgebra structure on $\a$ given by $\delta := \delta^+ \oplus (-\delta^-)$. As such, the Manin triple constructed above is in fact a triple of Lie bialgebras; the procedure above that outputs the Lie bialgebra $(\a, \delta)$ from the Lie sub-bialgebra $(\a^+, \delta^+)$ is commonly known as Drinfeld's \textbf{classical double} construction, and we write:
    $$\a \cong \Dr(\a^+)$$
It is also easy to see that $\a \cong \Dr(\a^-)$.

In short, the procedure described above yields us a bijective correspondence:
    \begin{equation} \label{equation: manin_triple_lie_bialgebra_correspondence}
        \left\{ \text{Manin triples $(\a, \a^+, \a^-)$} \right\} \leftrightarrows \left\{ \text{Lie bialgebra structures $(\a^+, \delta^+)$} \right\}
    \end{equation}
wherein the forward map is given by $\delta^+ := [\Box, \calr]|_{\a^+}$ while the backward map sends $(\a^+, \delta^+)$ to $(\Dr(\a^+) := \a^+ \oplus (\a^+)^*, \a^+, (\a^+)^*)$.

            \subsection{Quantisation of Lie bialgebras}
                \todo[inline]{Generalities about quantisation of topological Lie bialgebras}
    
        \section{Kac-Moody Lie (bi)algebras}
            \subsection{Kac-Moody algebras} \label{subsection: setup_kac_moody_algebras}
                Let us recall some basic features of Kac-Moody algebras. The canonical reference is the book \cite{kac_infinite_dimensional_lie_algebras}.

Following \cite[Chapter 1]{kac_infinite_dimensional_lie_algebras}, let $C := ( C_{i, j} )_{1 \leq i, j \leq n}$ be a (generalised) Cartan matrix and choose for it a realisation:
    $$(\h, \simpleroots, \simpleroots^{\vee})$$
This consists of a vector space $\h$ of dimension $l := 2n - \rank C$ along with linearly independent subsets $\simpleroots^{\vee} := \{ \alpha_i^{\vee} \}_{1 \leq i \leq n} \subset \h$ and $\simpleroots := \{ \alpha_i \}_{1 \leq i \leq n} \subset \h^*$, whose elements are known as \say{simple coroots} and \say{simple roots}, and are such that:
    \begin{equation} \label{equation: cartan_matrix_entries}
        \alpha_i( \alpha_j^{\vee} ) = C_{i, j}
    \end{equation}
\begin{remark}
    Because the Cartan matrix $C$ is not necessarily symmetric in general, the pairing \eqref{equation: cartan_matrix_entries} between the simple coroots and simple roots - by evaluating functionals - is generally \textit{not symmetric}.
\end{remark}

We can then define a Lie algebra $\tilde{\g}$ to be the one generated by the set:
    \begin{equation} \label{equation: kac_moody_generators}
        \h \cup \{ e_i^{\pm} \}_{1 \leq i \leq n}
    \end{equation}
whose elements are subjected to the following relations:
    \begin{equation} \label{equation: extended_kac_moody_relations}
        \begin{gathered}
            [h, h'] = 0 \quad, \quad h, h' \in \h
            \\
            [h, e_i^{\pm}] = \pm \alpha_i(h) e_i^{\pm} \quad, \quad [e_i^+, e_j^-] = \delta_{i, j} \alpha_i^{\vee} \quad, \quad h \in \h, 1 \leq i \leq n
        \end{gathered}
    \end{equation}
It can be shown (see e.g. \cite[Theorem 1.2]{kac_infinite_dimensional_lie_algebras}) that $\tilde{\g}$ admits a \say{triangular decomposition}:
    \begin{equation} \label{equation: extended_kac_moody_triangular_decomposition}
        \tilde{\g} \cong \tilde{\n}^- \oplus \h \oplus \tilde{\n}^+
    \end{equation}
wherein $\tilde{\n}^{\pm}$ are the free Lie algebras generated by the sets $\{ e_i^{\pm} \}_{1 \leq i \leq n}$.
    
Next, if we let $\r \subset \tilde{\g}$ be the Lie ideal that is the sum of all ideals with zero intersection with the Lie ideal $\h \subset \tilde{\g}$, then we shall obtain the \say{Kac-Moody algebra} $\g$ associated to the previously fixed Cartan matrix as the quotient:
    $$\g := \tilde{\g}/\r$$
\textit{A priori} - and this is a somewhat non-trivial fact (see \cite[Theorem 9.11]{kac_infinite_dimensional_lie_algebras}) - the Lie algebra $\g$ is generated by the same set \eqref{equation: kac_moody_generators}\footnote{Even though it is technically an abuse of notations, we shall use the same symbols to denote the elements of \eqref{equation: kac_moody_generators} and their images under the quotient map $\tilde{\g} \to \g$.}, and in addition to the relations \eqref{equation: extended_kac_moody_relations}, the generators now satisfy also the so-called \say{Serre relations}, which take the following form in the adjoint representation of $\g$:
    \begin{equation} \label{equation: kac_moody_serre_relations}
        ( \ad( e_i^{\pm} ) )^{1 - C_{i, j}} \cdot e_j \quad, \quad 1 \leq i \not = j \leq n
    \end{equation}
or in other words, $\r$ is generated by such relations. This leads to a triangular decomposition:
    \begin{equation} \label{equation: kac_moody_triangular_decomposition}
        \g \cong \n^- \oplus \h \oplus \n^+
    \end{equation}
wherein $\n^{\pm} := \tilde{\n}^{\pm}/( \tilde{\n}^{\pm} \cap \r )$

Due to the relations $[h, e_i^{\pm}] = \pm \alpha_i(h) e_i^{\pm}$, the Lie algebra $\tilde{\g}$ has a canonical \say{root grading} by the abelian group $\rootlattice := \Z \simpleroots$ (commonly called the \say{root lattice}), taking the form of a \say{root space decomposition}:
    $$\tilde{\g} \cong \bigoplus_{\alpha \in \rootlattice} \tilde{\g}_{\alpha}$$
wherein $\tilde{\g}_{\alpha} := \{ x \in \tilde{\g} \mid \forall h \in \h: [h, x] = \alpha(h) x \}$ are the \say{root spaces}. Any lattice element $\alpha := \sum_{1 \leq i \leq n} a_i \alpha_i \in \rootlattice$ has a \say{height} $\height(\alpha) := \sum_{1 \leq i \leq n} a_i$, which allows us to define $\deg x_{\alpha} := \height(\alpha)$ for all $x_{\alpha} \in \tilde{\g}_{\alpha}$. In particular, the degrees of the generators \eqref{equation: kac_moody_generators} are $\deg e_i^{\pm} = \pm 1$ and $\deg h = 0$ (for all $h \in \h$). Through the relations \eqref{equation: kac_moody_serre_relations}, one can also see that the graded components are all finite-dimensional.

Henceforth, let us assume moreover that the Cartan matrix $C$ is symmetrisable, i.e. that there exists an invertible diagonal $n \x n$ matrix $D$ and a symmetric $n \x n$ matrix $A$ (called the symmetrisation of $C$) such that:
    \begin{equation} \label{equation: symmetrising_cartan_matrices}
        C := DA
    \end{equation}
This allows us to define a symmetric and non-degenerate bilinear form $(\cdot, \cdot)_{\h} \in \Hom( \Sym^2(\h), \bbC )$ given by:
    \begin{equation} \label{equation: kac_moody_pairing_on_cartan_subalgebras}
        ( \alpha_i^{\vee}, h )_{\h} := \delta_{i, j} D_{i, i}^{-1} \alpha_i(h) \quad, \quad h \in \h, 1 \leq i \leq n
    \end{equation}
As an aside, we note that equations \eqref{equation: kac_moody_pairing_on_cartan_subalgebras} and \eqref{equation: cartan_matrix_entries} together imply that:
    $$A_{i, j} = (\alpha_i^{\vee}, \alpha_j^{\vee})_{\h} = \delta_{i, j} D_{i, i}^{-1} \alpha_i( \alpha_j^{\vee} ) = \delta_{i, j} D_{i, i}^{-1} C_{i, j}$$
Anyhow, the bilinear form $(\cdot, \cdot)_{\h}$ constructed above induces, via an induction process\footnote{For a less \textit{ad hoc} construction of $(\cdot, \cdot)_{\g}$, we refer the reader to \cite{neher_pianzola_prelat_sepp_invariant_bilinear_forms_via_fppf_descent}.}, a symmetric, non-degenerate, and \textit{invariant} bilinear form $(\cdot, \cdot)_{\g} \in \Hom( \Sym^2(\g)^{\g}, \bbC )$ given by:
    \begin{equation} \label{equation: kac_moody_pairing}
        ( e_i^-, e_j^+ )_{\g} = \delta_{i, j} D_{i, i}^{-1} \quad, \quad 1 \leq i, j \leq n
    \end{equation}
and is uniquely determined by $(\cdot, \cdot)_{\h}$. Moreover, the bilinear form $(\cdot, \cdot)_{\g}$ extends to $\tilde{\g}$; this extension is a degenerate bilinear form whose radical is precisely $\r \subset \tilde{\g}$. Additionally, and this is a rather important fact for us, the bilinear form $(\cdot, \cdot)_{\g}$ is of total degree $0$ with respect to the grading on $\Hom( \Sym^2(\g)^{\g}, \bbC )$ by the root lattice $\rootlattice$, induced by the one on $\g$ (and this is one reason why the root spaces $\g_{\alpha} \subset \g$ being finite-dimensional is important).
\begin{convention}
    For brevity, let us refer to symmetric (and often also non-degenerate) and invariant bilinear forms as \say{invariant pairings}.
\end{convention}

\todo[inline]{Loop realisations for affine Kac-Moody algbras (untwisted and twisted).}

            \subsection{The standard Lie bialgebra structure on Kac-Moody algebras} \label{subsection: setup_standard_kac_moody_lie_bialgebras}
                Let $\b^{\pm} := \n^{\pm} \oplus \h$ be the \say{Borel subalgebras} of the Kac-Moody algebra $\g$. It is clear that these Lie subalgebras of $\g$ are not isotropic with respect to the Kac-Moody pairing $(\cdot, \cdot)_{\g}$ given by equation \eqref{equation: kac_moody_pairing}. However, one can consider instead the larger Lie algebra $\a := \b^+ \oplus \b^-$, into which $\b^{\pm}$ embed by means of the maps $\eta^{\pm}: \b^{\pm} \hookrightarrow \a$ given by:
    \begin{equation} \label{equation: borel_lie_sub_bialgebra_embeddings}
        \eta^{\pm}(x) := x \oplus ( \pm x_{\h} ) \quad, \quad x \in \b^{\pm}
    \end{equation}
wherein $x_{\h}$ is the image of $x \in \b^{\pm}$ under the canonical quotient map $\b^{\pm} \to \b^{\pm}/\n^{\pm} \cong \h$. This larger Lie algebra shall be equipped with the non-degenerate and invariant pairing given by:
    $$(\cdot, \cdot)_{\a} := (\cdot, \cdot)_{\g} - (\cdot, \cdot)_{\h}$$
in which the Lie subalgebras $\eta^{\pm}(\b^{\pm})$ are clearly isotropic with respect to $(\cdot, \cdot)_{\a}$. As such, there is a Manin triple:
    $$(\a, \eta^+(\b^+), \eta^-(\b^-))$$
from which arises the topological Lie bialgebra structures $\delta^{\pm}: \eta^{\pm}(\b^{\pm}) \to \eta^{\pm}(\b^{\pm}) \hattensor \eta^{\pm}(\b^{\pm})$ given by:
    $$\delta^{\pm} = [\Box, \calr]$$
wherein $\calr$ is the Casimir tensor. On generators, these Lie cobrackets are given by:
    \begin{equation} \label{equation: standard_kac_moody_lie_bialgebra_structure}
        \begin{gathered}
            \delta^{\pm}(h) = 0 \quad, \quad h \in \h
            \\
            \delta^{\pm}(e_i^{\pm}) = \frac12 D_{i, i} e_i^{\pm} \wedge \alpha_i^{\vee} \quad, \quad 1 \leq i \leq n
        \end{gathered}
    \end{equation}
\begin{remark}
    Note also, that by construction, we have that:
        $$\a \cong \Dr( \eta^{\pm}( \b^{\pm} ) )$$
    as Lie bialgebras.
\end{remark}
    
As a consequence of this construction, the Lie subalgebra $\h \subset \a$ is a Lie coideal on top of being a Lie ideal (this is trivial, for it is abelian), and hence the quotient:
    $$\g \cong \a/\h$$
carries a Lie bialgebra structure given by the same formulae as in \eqref{equation: standard_kac_moody_lie_bialgebra_structure}.
\begin{remark}
    In fact, it is well-known that the formulae in \eqref{equation: standard_kac_moody_lie_bialgebra_structure} lift to topological Lie bialgebra structures $\tilde{\delta}^{\pm}: \tilde{\b}^{\pm} \to \tilde{\b}^{\pm} \hattensor \tilde{\b}^{\pm}$ on the Borel subalgebras $\tilde{\b}^{\pm} := \h \oplus \tilde{\n}^{\pm}$, and hence also on the extended Kac-Moody algebra $\tilde{\g}$. This is a \textit{post hoc} construction, inspired by the construction of the Lie bialgebra structures on $\b^{\pm}$ described above, and it can also be shown, without much difficulty, that the Lie algebra $\tilde{\a} := \tilde{\b}^+ \oplus \tilde{\b}^-$ with the Lie cobracket given by $\tilde{\delta} := \tilde{\delta}^+ \oplus (-\tilde{\delta}^-)$ is isomorphic to the classical doubles $\Dr(\tilde{\b}^{\pm})$; note that the embeddings of $\tilde{\b}^{\pm}$ into this larger Lie bialgebra are given by the same formulae as in \eqref{equation: borel_lie_sub_bialgebra_embeddings}. 

    That said, the existence of such a Lie bialgebra structure on the extended Kac-Moody algebra $\tilde{\g}$ is very useful for formulating a quantisation of $\g$. See, for instance, \cite{etingof_kazhdan_quantisation_6}. 
\end{remark}
    \end{appendices}
	
    \addcontentsline{toc}{section}{References}
    \printbibliography

\end{document}